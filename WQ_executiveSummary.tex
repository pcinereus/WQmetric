\section{Executive Summary}



The Reef 2050 Water Quality Improvement Plan (Reef Plan) guides how industry, government and the
community will work together to improve the quality of water flowing to the Great Barrier Reef
(GBR). Nested under the water quality theme of Reef 2050, it is a joint commitment of the Australian
and Queensland governments to address all land-based run-off flowing from the catchments adjacent to
the GBR. The plan sets the strategic priorities for the whole Reef catchment. Regional Water Quality
Improvement Plans, developed by regional natural resource management bodies, support the plan in
providing locally relevant information and guiding local priority actions within regions Progress
towards the goal and target is assessed and described through the annual Reef Report Card (Report
Card), which is based on a range of monitoring programs summarising improvements in land management
practices, progress towards pollutant targets, and the condition of the GBR and its catchments. The
information in this report determines the success of actions and identifies whether further measures
need to be taken to address water quality in the Great Barrier Reef. In previous Report Cards (until
2015), marine water quality was reported using a metric based on satellite remote sensing of near
surface concentrations of chlorophyll and total suspended solids.  This provided a wide spatial and
temporal coverage of marine water quality which cannot be achieved with in situ observations.

More specifically, in previous Report Cards, marine water quality was assessed using near-surface
concentrations of Chlorophyll-a (Chl-a) and non-algal particulates (NAP)\footnote{reported as Total                                                                                                   
Suspended Solids (TSS) which includes suspended solids and particulate nutrients.} as indicators
determined from satellite remote sensing. Index scores for these indicators were calculated based on
the relative area of the inshore water body that did or did not exceed the relevant GBRMPA Water
Quality Guidelines. Scores for Chl-a and NAP were aggregated (averaged) into a final metric value
subsequently converted into a final grade on a five-point uniform scale (very good, good, moderate,
poor, very poor) for each region. This final grade describes the overall water quality condition
across the Great Barrier Reef and within each individual region.  The water quality metric used
underpinning previous Report Cards (until 2015) presented a number of significant shortcomings:

\begin{itemize}
\item It was solely based on remote sensing-derived data. Concerns were raised about the
appropriateness of relying on remote sensing exclusively to evaluate inshore water quality,
considering well-documented challenges in obtaining accurate estimates from optically complex waters
and the fact that valid satellite observations are limited in the wet season due to cloud cover;
\item It was limited to reporting on two indicators and did not incorporate other water quality data
collected through the Marine Monitoring Program and IMOS;
\item It appeared relatively insensitive to large terrestrial inputs such as the impact of rainfall
on the volume and quality of water entering the GBR lagoon, most likely due to the binary assessment
of compliance relative to the water quality guidelines and aggregation and averaging over large
spatial and temporal scales;
\end{itemize}

In 2016, based on the limitations described above, the Reef Plan Independent Science Panel (ISP)
expressed a lack of confidence in the water quality metric used in Report Cards (until 2015) and
recommended that a new approach be identified for Report Card 2016 and future Report Cards. The ISP
also acknowledged substantial advancements in modelling water quality through the eReefs
biogeochemical models and the fact that recent research and method development\footnote{Such as a
Reef Rescue-funded project on data integration (Brando et al. 2013) and data aggregation methods
developed for and used in the recent Gladstone Healthy Harbour Partnership report card} Such as a
Reef Rescue-funded project on data integration (Brando et al. 2013) and data aggregation methods
developed for and used in the recent Gladstone Healthy Harbour Partnership report card had improved
our ability to construct report card metrics. To address the above shortcomings, the ISP requested
that:

\begin{itemize}
\item the e-Reefs marine biogeochemical model be tested for its ability to deliver a better water quality assessment than the current practice based on remote sensing;
\item the GBRMPA water quality guidelines be reviewed to incorporate new evidence collected over the last 6-8 years in understanding coral and seagrass responses to chronic and acute pressures, ecosystem health, recovery and resilience;
\item the utility of observational data streams from in-situ monitoring is analysed for potential inclusion in Report Card;
\item the current practice of scoring relative to water quality guidelines and aggregating data over fixed spatial and temporal scales be improved to incorporate the magnitude, frequency and duration of exceedance rather than using average annual exceedance counts;
\item the inclusion of photic depth, as derived from satellite data, into the metric be evaluated since light is the important driver for coral and seagrass productivity. The most appropriate measure of photic depth can be evaluated and related to seagrass and coral responses; and
\item options for combining indicator scores into a single metric are evaluated, including a statistical assessment of potential metrics.
\end{itemize}

These recommendations led to this NESP 3.2.5 project entitled "Testing and implementation of the
water quality metric for the 2017 and 2018 reef report cards". Run as a collaboration between
GBRMPA, AIMS, CSIRO and JCU, the high-level objectives for this project were to identify and assess
alternative strategies to integrate available monitoring and modelling data into an improved metric,
adopt these findings into Report Card 2016, and provide recommendations for further improvements to
the metric in subsequent report cards. To achieve these objectives and meet the timelines of Report
Card 2016, significant improvements had to be demonstrated by April 2017.
