\section{Exploratory data analysis}\label{sec:EDA}

Exploratory data analysis is vital for informing data processing and analysis as well as
establishing assumptions and limitations.  Of particular importance for the current project is the
spatial and temporal distribution and variability of the various data Measures and Sources.  As
such, a series of exploratory plots have been generated (see Appendix~\ref{appsec:EDA} beginning on
page \pageref{appsec:EDA}).  In the interest of keeping the main text free of copious graphics, we
have elected to present only a small fraction of the exploratory data analyses figures here.  The
figures presented will act as exemplars of general format and predominant features or patterns.

\subsection{All data}

Figures~\ref{fig:violin_chl_oc} -- \ref{fig:violin_NOx_oc} display the temporal distribution of
Chlorophyll-a, TSS, Secchi depth and NOx observations for the Wet Tropics Open Coastal Zone from
AIMS insitu, AIMS FLNTU, Satellite, eReefs and eReefs926 sources.

All of the figures are presented with log-transformed y-axes as the data are typically positively
skewed.  This is expected for parameters that have a natural minimum (zero), yet no theoretical
maximum.  It does however mean that these distributional properties should be considered during the
analyses.  In particular, for mean based aggregations, outliers and skewed distributions can impart
unrepresentative influence on outcomes.

Each of the data sources present different variability characteristics.  The scale of the range of
AIMS insitu data is predominantly and approximately less than or equal to the scale of the
half/twice the associated threshold value (Fig.~\ref{fig:violin_chl_oc}-\ref{fig:violin_NOx_oc}a).
The AIMS FLNTU logger data (Fig.~\ref{fig:violin_chl_oc}-\ref{fig:violin_NOx_oc}b) have a larger
range than the AIMS insitu data - presumably because the former data collection frequency captures
most of the peaks and trophs whereas the latter are unlikely to do so.  Furthermore, whilst the AIMS
insitu data are predominantly collected during the dry season, the AIMS FLNTU loggers collect data
across the entire year and are therefore likely to record a greater proportion of the full variation
in conditions.  Of course it is important when interpreting these diagnostic plots to focus mainly
on the violin plots and less on the dots (representing individual observations).  This is because
the dots do not provide an indication of the density and it is easy to allow outliers to distort out
impression of the variability of the data.

Similarly, the scale of the range eReefs and eReefs926 data
(Fig.~\ref{fig:violin_chl_oc}-\ref{fig:violin_NOx_oc}d-e) is approximately equal to the scale of the
range of the span from half/twice the threshold value.  This reflects both a more complete time
series and broader spatial extent represented in the data.  In contrast to the AIMS insitu and to a
lesser extent the AIMS FLNTU and eReefs data, the scale of the range of the Satellite is relatively
large - typically a greater span than the range of half/twice threshold value
(Fig.~\ref{fig:violin_chl_oc}-\ref{fig:violin_NOx_oc}c).

The Satellite, eReefs and eReefs926 data series all start and end part of the way through a water
year.  For annually aggregated data, this is likely to result in unrepresentative estimates and thus
only full water years will be analysed.
 
\subsection{Annual data}\label{sec:annualchunks}
\begin{figure}[ptbh] 
  a) AIMS insitu\\\includegraphics[align=t,width=0.95\linewidth]{{figures/Exploratory_Data_Analysis/Insitu/eda.year.chl_Wet Tropics__Open Coastal_niskin_log\res}.pdf}\\
  b) AIMS FLNTU\\\includegraphics[align=t, width=0.95\linewidth]{{figures/Exploratory_Data_Analysis/FLNTU/eda.year.chl_Wet Tropics__Open Coastal_flntu_log\res}.pdf}\\
  c) Satellite\\\includegraphics[align=t, width=0.95\linewidth]{{figures/Exploratory_Data_Analysis/Satellite/eda.year.chl_Wet Tropics__Open Coastal__log\res}.pdf}}\\
  d) eReefs\\\includegraphics[align=t, width=0.95\linewidth]{{figures/Exploratory_Data_Analysis/eReefs/eda.year.chl_Wet Tropics__Open Coastal_eReefs_log\res}.pdf}}\\
  e) eReefs926\\\includegraphics[align=t, width=0.95\linewidth]{{figures/Exploratory_Data_Analysis/eReefs926/eda.year.chl_Wet Tropics__Open Coastal_eReefs926_log\res}.pdf}}\\
  \caption[Observed Chlorophyll-a data for the Wet Tropics Open Coastal Zone (grouped annually)]{Observed (logarithmic axis with violin plot overlay) Chlorophyll-a data for the Wet
Tropics Open Coastal Zone from a) AIMS insitu, b) AIMS FLNTU, c) Satellite, d) eReefs and e)
eReefs926.  Observations are ordered over time and colored conditional on season as Wet (blue
symbols) and Dry (red symbols).  Blue smoother represents Generalized Additive Mixed Model within a
water year and purple line represents average within the water year.  Horizontal red, black and
green dashed lines denote the twice threshold, threshold and half threshold values respectively.
Red and green background shading indicates the range (10\% shade: x4,/4; 30\% shade: x2,/2)
above and below threshold respectively.}\label{fig:violin_chl_oc}
\end{figure}

\begin{figure}[ptbh]
  a) AIMS insitu\\\includegraphics[align=t,width=0.95\linewidth]{{figures/Exploratory_Data_Analysis/Insitu/eda.year.nap_Wet Tropics__Open Coastal_niskin_log\res}.pdf}\\
  b) AIMS FLNTU\\\includegraphics[align=t, width=0.95\linewidth]{{figures/Exploratory_Data_Analysis/FLNTU/eda.year.nap_Wet Tropics__Open Coastal_flntu_log\res}.pdf}\\
  c) Satellite\\\includegraphics[align=t, width=0.95\linewidth]{{figures/Exploratory_Data_Analysis/Satellite/eda.year.nap_Wet Tropics__Open Coastal__log\res}.pdf}\\
  d) eReefs\\\includegraphics[align=t, width=0.95\linewidth]{{figures/Exploratory_Data_Analysis/eReefs/eda.year.nap_Wet Tropics__Open Coastal_eReefs_log\res}.pdf}\\
  e) eReefs926\\\includegraphics[align=t, width=0.95\linewidth]{{figures/Exploratory_Data_Analysis/eReefs926/eda.year.nap_Wet Tropics__Open Coastal_eReefs926_log\res}.pdf}\\
  \caption[Observed TSS data for the Wet Tropics Open Coastal Zone (grouped annually)]{Observed (logarithmic axis with violin plot overlay) TSS data for the Wet Tropics Open
Coastal Zone from a) AIMS insitu, b) AIMS FLNTU, c) Satellite, d) eReefs and e) eReefs926.
Observations are ordered over time and colored conditional on season as Wet (blue symbols) and Dry
(red symbols).  Blue smoother represents Generalized Additive Mixed Model within a water year and
purple line represents average within the water year.  Horizontal red, black and green dashed lines
denote the twice threshold, threshold and half threshold values respectively.  Red and green
background shading indicates the range (10\% shade: x4,/4; 30\% shade: x2,/2)
above and below threshold respectively.}\label{fig:violin_nap_oc}
\end{figure}
 
\begin{figure}[ptbh]
  a) AIMS insitu\\\includegraphics[align=t,width=0.95\linewidth]{{figures/Exploratory_Data_Analysis/Insitu/eda.year.sd_Wet Tropics__Open Coastal_niskin_log\res}.pdf}\\
  b) AIMS FLNTU\\\includegraphics[align=t, width=0.95\linewidth]{{figures/Exploratory_Data_Analysis/FLNTU/eda.year.sd_Wet Tropics__Open Coastal_flntu_log\res}.pdf}\\
  c) Satellite\\\includegraphics[align=t, width=0.95\linewidth]{{figures/Exploratory_Data_Analysis/Satellite/eda.year.sd_Wet Tropics__Open Coastal__log\res}.pdf}\\
  d) eReefs\\\includegraphics[align=t, width=0.95\linewidth]{{figures/Exploratory_Data_Analysis/eReefs/eda.year.sd_Wet Tropics__Open Coastal_eReefs_log\res}.pdf}\\
  e) eReefs926\\\includegraphics[align=t, width=0.95\linewidth]{{figures/Exploratory_Data_Analysis/eReefs926/eda.year.sd_Wet Tropics__Open Coastal_eReefs926_log\res}.pdf}\\
  \caption[Observed Secchi depth data for the Wet Tropics Open Coastal Zone (grouped annually)]{Observed (logarithmic axis with violin plot overlay) Secchi depth data for the Wet
Tropics Open Coastal Zone from a) AIMS insitu, b) AIMS FLNTU, c) Satellite, d) eReefs and e)
eReefs926.  Observations are ordered over time and colored conditional on season as Wet (blue
symbols) and Dry (red symbols).  Blue smoother represents Generalized Additive Mixed Model within a
water year and purple line represents average within the water year.  Horizontal red, black and
green dashed lines denote the twice threshold, threshold and half threshold values respectively.
Red and green background shading indicates the range (10\% shade: x4,/4; 30\% shade: x2,/2)
above and below threshold respectively.}\label{fig:violin_sd_oc}
\end{figure}


\begin{figure}[ptbh]
  a) AIMS insitu\\\includegraphics[align=t,width=0.95\linewidth]{{figures/Exploratory_Data_Analysis/Insitu/eda.year.NOx_Wet Tropics__Open Coastal_niskin_log\res}.pdf}\\
  %b) AIMS FLNTU\\\includegraphics[align=t, width=0.95\linewidth]{{figures/Exploratory_Data_Analysis/FLNTU/eda.year.NOx_Wet Tropics__Open Coastal_flntu_log}.pdf}\\
  %c) Satellite\\\includegraphics[align=t, width=0.95\linewidth]{{figures/Exploratory_Data_Analysis/Satellite/eda.year.NOx_Wet Tropics__Open Coastal__log}.png}\\
  b) eReefs\\\includegraphics[align=t, width=0.95\linewidth]{{figures/Exploratory_Data_Analysis/eReefs/eda.year.NOx_Wet Tropics__Open Coastal_eReefs_log\res}.pdf}\\
  c) eReefs926\\\includegraphics[align=t,width=0.95\linewidth]{{figures/Exploratory_Data_Analysis/eReefs926/eda.year.NOx_Wet Tropics__Open Coastal_eReefs926_log\res}.pdf}\\
  \caption[Observed NOx data for the Wet Tropics Open Coastal Zone (grouped annually)]{Observed (logarithmic axis with violin plot overlay) NOx data for the Wet Tropics Open
  Coastal Zone from a) AIMS insitu, b) eReefs and c) eReefs926.  Observations are ordered over time
  and colored conditional on season as Wet (blue symbols) and Dry (red symbols).  Blue smoother
  represents Generalized Additive Mixed Model within a water year and purple line represents average
  within the water year.  Horizontal red, black and green dashed lines denote the twice threshold,
  threshold and half threshold values respectively.  Red and green background shading indicates the
  range (10\% shade: x4,/4; 30\% shade: x2,/2) above and below threshold
  respectively.}\label{fig:violin_NOx_oc}
\end{figure}

\clearpage

\subsection{Monthly data}

Figures~\ref{fig:violin_month_chl_oc_1} -- \ref{fig:violin_month_NOx_oc_1} provide finer temporal
resolution by displaying the temporal distribution of Chlorophyll-a, TSS, Secchi depth and NOx
observations for the each month within Wet Tropics Open Coastal Zone from AIMS insitu, AIMS FLNTU,
Satellite, eReefs and eReefs926 sources.

The monthly violin plots do not add any additional insights with respect to understanding the
characteristics of the underlying data to help guide the selection of appropriate indexation
formulation or perhaps even Measure/Source selection. Rather, they provide a less compacted view of
the underlying data from which patterns highlighted in Section~\ref{sec:annualchunks} might be more
easily appreciated.


\begin{figure}[ptbh]
  a) AIMS insitu\\\includegraphics[align=t,width=0.95\linewidth]{{figures/Exploratory_Data_Analysis/Insitu/eda.year.month.chl_Wet Tropics__Open Coastal_niskin_log\res.pdf}}\\[1em]
  b) AIMS FLNTU\\\includegraphics[align=t,width=0.95\linewidth]{{figures/Exploratory_Data_Analysis/FLNTU/eda.year.month.chl_Wet Tropics__Open Coastal_flntu_log\res.pdf}}\\[1em]
  \caption[Observed AIMS niskin and FLNTU Chlorophyll-a data for the Wet Tropics Open Coastal Zone (grouped monthly)]{Observed (logarithmic axis with violin plot overlay) Chlorophyll-a data for the Wet Tropics Open Coastal Zone from a) AIMS insitu, b) AIMS FLNTU.
    Observations grouped into months are ordered over time and colored conditional on season as Wet (blue symbols) and Dry (red symbols).
    Sample sizes represented as numbers above violins and horizontal black dashed line denotes threshold value.
    Red and green background shading indicates the range (10\% shade: x4,/4; 30\% shade: x2,/2) above and below threshold respectively.}\label{fig:violin_month_chl_oc_1}
\end{figure}

\begin{figure}[ptbh]
  a) Satellite\\\includegraphics[align=t,width=0.95\linewidth]{{figures/Exploratory_Data_Analysis/Satellite/eda.year.month.chl_Wet Tropics__Open Coastal__log\res.pdf}}\\[1em]
  b) eReefs\\\includegraphics[align=t,width=0.95\linewidth]{{figures/Exploratory_Data_Analysis/eReefs/eda.year.month.chl_Wet Tropics__Open Coastal_eReefs_log\res.pdf}}\\[1em]
  \caption[Observed Satellite and eReefs Chlorophll-a data for the Wet Tropics Open Coastal Zone (grouped monthly)]{Observed (logarithmic axis with violin plot overlay) Chlorophyll-a data for the Wet Tropics Open Coastal Zone from a) Satellite, b) eReefs.
    Observations grouped into months are ordered over time and colored conditional on season as Wet (blue symbols) and Dry (red symbols).
    Sample sizes represented as numbers above violins and horizontal black dashed line denotes threshold value.
    Red and green background shading indicates the range (10\% shade: x4,/4; 30\% shade: x2,/2) above and below threshold respectively.}\label{fig:violin_month_chl_oc_2}
\end{figure}


\begin{figure}[ptbh]
  a) AIMS insitu\\\includegraphics[align=t,width=0.95\linewidth]{{figures/Exploratory_Data_Analysis/Insitu/eda.year.month.nap_Wet Tropics__Open Coastal_niskin_log\res.pdf}}\\[1em]
  b) AIMS FLNTU\\\includegraphics[align=t,width=0.95\linewidth]{{figures/Exploratory_Data_Analysis/FLNTU/eda.year.month.nap_Wet Tropics__Open Coastal_flntu_log\res.pdf}}\\[1em]
  \caption[Observed AIMS niskin and FLNTU TSS data for the Wet Tropics Open Coastal Zone (grouped monthly)]{Observed (logarithmic axis with violin plot overlay) TSS data for the Wet Tropics Open Coastal Zone from a) AIMS insitu, b) AIMS FLNTU.
    Observations grouped into months are ordered over time and colored conditional on season as Wet (blue symbols) and Dry (red symbols).
    Sample sizes represented as numbers above violins and horizontal black dashed line denotes threshold value.
    Red and green background shading indicates the range (10\% shade: x4,/4; 30\% shade: x2,/2) above and below threshold respectively.}\label{fig:violin_month_nap_oc_1}
\end{figure}

\begin{figure}[ptbh]
  a) Satellite\\\includegraphics[align=t,width=0.95\linewidth]{{figures/Exploratory_Data_Analysis/Satellite/eda.year.month.nap_Wet Tropics__Open Coastal__log\res.pdf}}\\[1em]
  b) eReefs\\\includegraphics[align=t,width=0.95\linewidth]{{figures/Exploratory_Data_Analysis/eReefs/eda.year.month.nap_Wet Tropics__Open Coastal_eReefs_log\res.pdf}}\\[1em]
  \caption[Observed Satellite and eReefs TSS data for the Wet Tropics Open Coastal Zone (grouped monthly)]{Observed (logarithmic axis with violin plot overlay) TSS data for the Wet Tropics Open Coastal Zone from a) Satellite, b) eReefs.
    Observations grouped into months are ordered over time and colored conditional on season as Wet (blue symbols) and Dry (red symbols).
    Sample sizes represented as numbers above violins and horizontal black dashed line denotes threshold value.
    Red and green background shading indicates the range (10\% shade: x4,/4; 30\% shade: x2,/2) above and below threshold respectively.}\label{fig:violin_month_nap_oc_2}
\end{figure}
  
\begin{figure}[ptbh]
  a) AIMS insitu\\\includegraphics[align=t,width=0.8\linewidth]{{figures/Exploratory_Data_Analysis/Insitu/eda.year.month.sd_Wet Tropics__Open Coastal_niskin_log\res.pdf}}\\[1em]
  b) Satellite\\\includegraphics[align=t,width=0.8\linewidth]{{figures/Exploratory_Data_Analysis/Satellite/eda.year.month.sd_Wet Tropics__Open Coastal__log\res.pdf}}\\[1em]
  c) eReefs\\\includegraphics[align=t,width=0.8\linewidth]{{figures/Exploratory_Data_Analysis/eReefs/eda.year.month.sd_Wet Tropics__Open Coastal_eReefs_log\res.pdf}}\\[1em]
  \caption[Observed AIMS niskin, Satellite and eReefs Secchi depth data for the Wet Tropics Open Coastal Zone (grouped monthly)]{Observed (logarithmic axis with violin plot overlay) Secchi depth data for the Wet Tropics Open Coastal Zone from a) AIMS insitu, b) Satellite and c) eReefs.
    Observations grouped into months are ordered over time and colored conditional on season as Wet (blue symbols) and Dry (red symbols).
    Sample sizes represented as numbers above violins and horizontal black dashed line denotes threshold value.
    Red and green background shading indicates the range (10\% shade: x4,/4; 30\% shade: x2,/2) above and below threshold respectively.}\label{fig:violin_month_sd_oc_1}
\end{figure}
 
\begin{figure}[ptbh]
  a) AIMS insitu\\\includegraphics[align=t,width=0.8\linewidth]{{figures/Exploratory_Data_Analysis/Insitu/eda.year.month.NOx_Wet Tropics__Open Coastal_niskin_log\res.pdf}}\\[1em]
  %b) Satellite\\\includegraphics[align=t,width=0.8\linewidth]{{figures/Exploratory_Data_Analysis/Satellite/eda.year.month.NOx_Wet Tropics__Open Coastal__log\res.pdf}}\\[1em]
  b) eReefs\\\includegraphics[align=t,width=0.8\linewidth]{{figures/Exploratory_Data_Analysis/eReefs/eda.year.month.NOx_Wet Tropics__Open Coastal_eReefs_log\res.pdf}}\\[1em]
  c) eReefs926\\\includegraphics[align=t,width=0.8\linewidth]{{figures/Exploratory_Data_Analysis/eReefs926/eda.year.month.NOx_Wet Tropics__Open Coastal_eReefs926_log\res.pdf}}\\[1em]
  \caption[Observed AIMS niskin, Satellite and eReefs NOx data for the Wet Tropics Open Coastal Zone (grouped monthly)]{Observed (logarithmic axis with violin plot overlay) NOx data for the Wet Tropics Open Coastal Zone from a) AIMS insitu, b) eReefs c) eReefs926.
    Observations grouped into months are ordered over time and colored conditional on season as Wet (blue symbols) and Dry (red symbols).
    Sample sizes represented as numbers above violins and horizontal black dashed line denotes threshold value.
    Red and green background shading indicates the range (10\% shade: x4,/4; 30\% shade: x2,/2) above and below threshold respectively.}\label{fig:violin_month_NOx_oc_1}
\end{figure}

\clearpage

\subsection{Spatial data}

Figures~\ref{fig:eda.spatial_chl_oc} -- \ref{fig:eda.spatial_NOx_m} explore the spatio-temporal
patterns in observed data from a finer spatial perspective (again focussing on just the Wet Tropics
Open Coastal and Dry Tropics Midshelf Zones).  Importantly, the colour scales have been mapped to
a constant value range for each source for a given Measure.  The lower and upper bounds of the
constant range is 
respectively based on twice and half the threshold (see Table~\ref{tab:thresholds}) value
(except for Secchi depth which are half and twice respectively).  Half and double the threshold was
considered broadly appropriate for the Insitu data and thus, should also be broadly appropriate for the
other sources which could be considered 'proxies' for direct sampling.

% since they are intended to be indirect approximations of direct sampling.


These figures also highlight the disparity in resolution
between the different data sources. The AIMS insitu data is spatially very sparse \footnote{the AIMS
FLNTU logger data is even more sparse and thus is not shown.}.  The Satellite data has the most
extensive spatial resolution and notwithstanding the many gaps due to various optical interferences
(such as cloud cover), also has the greatest temporal coverage\footnote{The remote sensing Satellite data
span a temporal range of 2002 through to 2017, although only the range 2010-2016 is displayed}.

For the selected Zones and span of water years, there is little evidence of a major latitudinal
gradient in Satellite Chlorophyll-a with most of any change (if any) occurring across the shelf.
Indeed, Satellite parameters are relatively constant over space and time for the Dry Tropics
Midshelf Zone (see Figs.~\ref{fig:eda.spatial_chl_m}--\ref{fig:eda.spatial_sd_m}b).  Moreover, the
spatial patterns of Satellite derived Chlorophyll-a and TSS appear relatively invariant between
years (see Figs.\ref{fig:eda.spatial_chl_oc}--\ref{fig:eda.spatial_sd_m}b).

The eReefs and eReefs926 do show some variability in spatial and temporal Chlorophyll-a and Secchi
depth (see
Figs.~\ref{fig:eda.spatial_chl_oc}c-d,\ref{fig:eda.spatial_sd_oc}c-d,\ref{fig:eda.spatial_chl_m}c-d
and \ref{fig:eda.spatial_sd_m}c-d), yet relatively little for TSS and NOx (at least for Dry Tropics
Midshelf).  Whilst this apparent lack of variability is largely an artefact of the colour scale mapping,
the values of these Measures are constantly
substantially below the threshold value and thus invariant on the scale considered
appropriate for comparison against the associated thresholds..


%% Spatial maps (single zones)
%% Wet Tropics Open Coastal

\begin{landscape}
  
  \begin{figure}[ptbh]
    \begin{minipage}{0.5\linewidth}
      a) AIMS insitu\\\includegraphics[align=t,width=1\linewidth]{{figures/Exploratory_Data_Analysis/Insitu/eda.spatial.year.chl_Wet Tropics__Open Coastal_niskin_logA\res.png}}\\[1em]
      c) eReefs\\\includegraphics[align=t,width=1\linewidth]{{figures/Exploratory_Data_Analysis/eReefs/eda.spatial.year.chl_Wet Tropics__Open Coastal_eReefs_logA\res.png}}\\[1em]
    \end{minipage}
    \begin{minipage}{0.5\linewidth}
      b) Satellite\\\includegraphics[align=t,width=1\linewidth]{{figures/Exploratory_Data_Analysis/Satellite/eda.spatial.year.chl_Wet Tropics__Open Coastal__logA\res.png}}\\[1em]
      d) eReefs926\\\includegraphics[align=t,width=1\linewidth]{{figures/Exploratory_Data_Analysis/eReefs926/eda.spatial.year.chl_Wet Tropics__Open Coastal_eReefs926_logA\res.png}}\\[1em]
    \end{minipage}
  %b) AIMS FLNTU\\\includegraphics[align=t,width=0.95\linewidth]{{figures/Exploratory_Data_Analysis/FLNTU/eda.spatial.year.chl_Wet Tropics__Open Coastal_flntu_logA\res.pdf}}\\[1em]
    \caption{Spatial distribution of observed a) AIMS insitu, b) Satellite, c) eReefs and d) eReefs926 Chlorophyll-a (2009--2016) for the Wet Tropics Open Coastal Zone.}\label{fig:eda.spatial_chl_oc}
\end{figure}
\end{landscape}

\begin{landscape}
  \begin{figure}[ptbh]
    \begin{minipage}{0.5\linewidth}
      a) AIMS insitu\\\includegraphics[align=t,width=1\linewidth]{{figures/Exploratory_Data_Analysis/Insitu/eda.spatial.year.nap_Wet Tropics__Open Coastal_niskin_logA\res.png}}\\[1em]
      c) eReefs\\\includegraphics[align=t,width=1\linewidth]{{figures/Exploratory_Data_Analysis/eReefs/eda.spatial.year.nap_Wet Tropics__Open Coastal_eReefs_logA\res.png}}\\[1em]
    \end{minipage}
    \begin{minipage}{0.5\linewidth}
      b) Satellite\\\includegraphics[align=t,width=1\linewidth]{{figures/Exploratory_Data_Analysis/Satellite/eda.spatial.year.nap_Wet Tropics__Open Coastal__logA\res.png}}\\[1em]
      d) eReefs926\\\includegraphics[align=t,width=1\linewidth]{{figures/Exploratory_Data_Analysis/eReefs926/eda.spatial.year.nap_Wet Tropics__Open Coastal_eReefs926_logA\res.png}}\\[1em]
    \end{minipage}
  %b) AIMS FLNTU\\\includegraphics[align=t,width=0.95\linewidth]{{../Figures/Exploratory Data Analysis/FLNTU/eda.spatial.year.nap_Wet Tropics__Open Coastal_flntu_logA\res.pdf}}\\[1em]
  \caption{Spatial distribution of observed a) AIMS insitu, b) Satellite, c) eReefs and d) eReefs926 TSS (2009--2016) for the Wet Tropics Open Coastal Zone.}\label{fig:eda.spatial_nap_oc}
\end{figure}
\end{landscape}

\begin{landscape}
  \begin{figure}[ptbh]
    \begin{minipage}{0.5\linewidth}
      a) AIMS insitu\\\includegraphics[align=t,width=1\linewidth]{{figures/Exploratory_Data_Analysis/Insitu/eda.spatial.year.sd_Wet Tropics__Open Coastal_niskin_logA\res.png}}\\[1em]
      c) eReefs\\\includegraphics[align=t,width=1\linewidth]{{figures/Exploratory_Data_Analysis/eReefs/eda.spatial.year.sd_Wet Tropics__Open Coastal_eReefs_logA\res.png}}\\[1em]
    \end{minipage}
    \begin{minipage}{0.5\linewidth}
      b) Satellite\\\includegraphics[align=t,width=0.95\linewidth]{{figures/Exploratory_Data_Analysis/Satellite/eda.spatial.year.sd_Wet Tropics__Open Coastal__logA\res.png}}\\[1em]
      d) eReefs926\\\includegraphics[align=t,width=1\linewidth]{{figures/Exploratory_Data_Analysis/eReefs926/eda.spatial.year.sd_Wet Tropics__Open Coastal_eReefs926_logA\res.png}}\\[1em]
    \end{minipage}
  \caption{Spatial distribution of observed a) AIMS insitu, b) Satellite, c) eReefs and d) eReefs926 Secchi depth (2009--2016) for the Wet Tropics Open Coastal Zone.}\label{fig:eda.spatial_sd_oc}
\end{figure}
\end{landscape}

\begin{landscape}
  \begin{figure}[ptbh]
    \begin{minipage}[t]{0.5\linewidth}
      a) AIMS insitu\\\includegraphics[align=t,width=1\linewidth]{{figures/Exploratory_Data_Analysis/Insitu/eda.spatial.year.NOx_Wet Tropics__Open Coastal_niskin_logA\res.png}}\\[1em]
      c) eReefs926\\\includegraphics[align=t,width=1\linewidth]{{figures/Exploratory_Data_Analysis/eReefs926/eda.spatial.year.NOx_Wet Tropics__Open Coastal_eReefs926_logA\res.png}}\\[1em]      
    \end{minipage}
    \begin{minipage}[t]{0.5\linewidth}
      b) eReefs\\\includegraphics[align=t,width=1\linewidth]{{figures/Exploratory_Data_Analysis/eReefs/eda.spatial.year.NOx_Wet Tropics__Open Coastal_eReefs_logA\res.png}}\\[1em]
    \end{minipage}
  \caption{Spatial distribution of observed a) AIMS insitu, b) eReefs and c) eReefs926 NOx (2009--2016) for the Wet Tropics Open Coastal Zone.}\label{fig:eda.spatial_NOx_oc}
\end{figure}
\end{landscape}


\begin{landscape}
  \begin{figure}[ptbh]
    %\begin{center}
    %  \begin{parbox}{width=0.8\linewidth}
    a) AIMS insitu\begin{center}\includegraphics[align=t,width=0.8\linewidth]{{figures/Exploratory_Data_Analysis/Insitu/eda.spatial.year.chl_Dry Tropics__Midshelf_niskin_logA\res.png}}\end{center}
    b) Satellite\begin{center}\includegraphics[align=t,width=0.8\linewidth]{{figures/Exploratory_Data_Analysis/Satellite/eda.spatial.year.chl_Dry Tropics__Midshelf__logA\res.png}}\end{center}
    c) eReefs\begin{center}\includegraphics[align=t,width=0.8\linewidth]{{figures/Exploratory_Data_Analysis/eReefs/eda.spatial.year.chl_Dry Tropics__Midshelf_eReefs_logA\res.png}}\end{center}
    d) eReefs926\begin{center}\includegraphics[align=t,width=0.8\linewidth]{{figures/Exploratory_Data_Analysis/eReefs926/eda.spatial.year.chl_Dry Tropics__Midshelf_eReefs926_logA\res.png}}\end{center}
    %\end{parbox}
    %\end{center}
  \caption{Spatial distribution of observed a) AIMS insitu, b) Satellite, c) eReefs and d) eReefs926 Chlorophyll-a (2009--2016) for the Dry Tropics Midshelf Zone.}\label{fig:eda.spatial_chl_m}
\end{figure}
\end{landscape}
    
\begin{landscape}
  \begin{figure}[ptbh]
      a) AIMS insitu\begin{center}\includegraphics[align=t,width=0.8\linewidth]{{figures/Exploratory_Data_Analysis/Insitu/eda.spatial.year.nap_Dry Tropics__Midshelf_niskin_logA\res.png}}\end{center}
      b) Satellite\begin{center}\includegraphics[align=t,width=0.8\linewidth]{{figures/Exploratory_Data_Analysis/Satellite/eda.spatial.year.nap_Dry Tropics__Midshelf__logA\res.png}}\end{center}
      c) eReefs\begin{center}\includegraphics[align=t,width=0.8\linewidth]{{figures/Exploratory_Data_Analysis/eReefs/eda.spatial.year.nap_Dry Tropics__Midshelf_eReefs_logA\res.png}}\end{center}
      d) eReefs926\begin{center}\includegraphics[align=t,width=0.8\linewidth]{{figures/Exploratory_Data_Analysis/eReefs926/eda.spatial.year.nap_Dry Tropics__Midshelf_eReefs926_logA\res.png}}\end{center}
    \caption{Spatial distribution of observed a) AIMS insitu, b) Satellite, c) eReefs and d) eReefs926 TSS (2009--2016) for the Dry Tropics Midshelf Zone.}\label{fig:eda.spatial_nap_m}
  \end{figure}
\end{landscape}

\begin{landscape}
  \begin{figure}[ptbh]
      a) AIMS insitu\begin{center}\includegraphics[align=t,width=0.8\linewidth]{{figures/Exploratory_Data_Analysis/Insitu/eda.spatial.year.sd_Dry Tropics__Midshelf_niskin_logA\res.png}}\end{center}
      b) Satellite\begin{center}\includegraphics[align=t,width=0.8\linewidth]{{figures/Exploratory_Data_Analysis/Satellite/eda.spatial.year.sd_Dry Tropics__Midshelf__logA\res.png}}\end{center}
      c) Reefs\begin{center}\includegraphics[align=t,width=0.8\linewidth]{{figures/Exploratory_Data_Analysis/eReefs/eda.spatial.year.sd_Dry Tropics__Midshelf_eReefs_logA\res.png}}\end{center}
      d) eReefs926\begin{center}\includegraphics[align=t,width=0.8\linewidth]{{figures/Exploratory_Data_Analysis/eReefs926/eda.spatial.year.sd_Dry Tropics__Midshelf_eReefs926_logA\res.png}}\end{center}
    % b) AIMS FLNTU\\\includegraphics[align=t,width=0.95\linewidth]{{../Figures/Exploratory Data Analysis/FLNTU/eda.spatial.year.sd_Dry Tropics__Midshelf_flntu_logA\res.pdf}}\\[1em]
    \caption{Spatial distribution of observed a) AIMS insitu, b) Satellite, c) eReefs and d) eReefs926 Secchi depth (2009--2016) for the Dry Tropics Midshelf Zone.}\label{fig:eda.spatial_sd_m}
  \end{figure}
\end{landscape}

\begin{landscape}
  \begin{figure}[ptbh]
      a) AIMS insitu\begin{center}\includegraphics[align=t,width=0.8\linewidth]{{figures/Exploratory_Data_Analysis/Insitu/eda.spatial.year.NOx_Dry Tropics__Midshelf_niskin_logA\res.png}}\end{center}
      b) Reefs\begin{center}\includegraphics[align=t,width=0.8\linewidth]{{figures/Exploratory_Data_Analysis/eReefs/eda.spatial.year.NOx_Dry Tropics__Midshelf_eReefs_logA\res.png}}\end{center}
      c) eReefs926\begin{center}\includegraphics[align=t,width=0.8\linewidth]{{figures/Exploratory_Data_Analysis/eReefs926/eda.spatial.year.NOx_Dry Tropics__Midshelf_eReefs926_logA\res.png}}\end{center}
    % b) AIMS FLNTU\\\includegraphics[align=t,width=0.95\linewidth]{{figures/Exploratory_Data_Analysis/FLNTU/eda.spatial.year.NOx_Dry Tropics__Midshelf_flntu_logA\res.png}}\\[1em]
    \caption{Spatial distribution of observed a) AIMS insitu, b) eReefs and c) eReefs926 NOx (2009--2016) for the Dry Tropics Midshelf Zone.}\label{fig:eda.spatial_NOx_m}
  \end{figure}
\end{landscape} 
 
\subsection{Comparison of data sources}

Ensuring that the data underpinning the metric calculations are fit-for-purpose is a critical part
of the process, especially if multiple data sources for a specific indicator are to be aggregated as
part of these calculations.  For example, successful aggregation of Chlorophyll-a as modelled by the
eReefs BGC with Chlorophyll-a as extracted from satellite reflectance data (optical properties) will
largely depend on the underlying compatibility of these two sources.  Moreover, further combining
with far more sparse and irregular sources (such as AIMS insitu Chlorophyll-a samples) relies on
general patterns of spatial and temporal autocorrelation being present across the more dense data
sources so as to facilitate a contagious projection of sparse data across the denser layers.


Based on substantial inconsistencies in the magnitude and variation of the observations between
sources (AIMS insitu, Satellite and eReefs models), we recommend not to aggregate across the
streams of data.  Although it might be possible to normalize each source such that they do all have
the same basic characteristics prior to aggregation\footnote{indeed this is one of the functions of
indexing metrics (see section~\ref{sec:indexing})}, all the various approaches to achieve
normalization rely on the availability of independent estimates of either data reliability, accuracy
or biases present in each source.  Unfortunately, such information is not available.

Instead of aggregating the sources together, the preferred approach is to assimilate satellite
reflectance information into the eReefs BGC model and to rely on in situ measurements for
verification of the model performance.

It is worthwhile noting that there is no single point of truth as the sparse insitu sampling does
not account for the dynamic nature of the receiving environment, both temporally and spatially.  It
is however possible to compare different measurement methods at a high level.

The five different sources (Satellite, eReefs, eReefs926, AIMS Insitu and AIMS FLNTU loggers) were
all collected at different spatio-temporal resolutions.  Specifically:
\begin{itemize}
\item the Satellite data are collected on a 1km grid on a daily basis, however there are many gaps
in the time series of each cell due to cloud cover and other issues that affect the reliability of
observations.
\item the eReefs data are modelled and projected on to a 4km grid on a daily basis without any time series gaps
between 2013 and 2016
\item the eReefs926 data are modelled and projected  on to a 4km grid on a daily basis without any time series gaps
between 2011 and 2014
\item the AIMS Insitu samples are collected from specific sampling sites (28-32 throughout the GBR)
and on an infrequent basis (approx. 3-4 times per year although more frequently in later years).
Furthermore, apart from relatively recently, the majority of samples were collected in
the dry season and thus these samples could be biased towards long term water quality trends rather
than short-term pulses.
\item the AIMS FLNTU logger data are deployed at a subset (16) of the AIMS Insitu sampling locations
and record measurements every 10 minutes (although there are frequent gaps due to instrument
failure).
\end{itemize}

The AIMS Insitu sampling locations are strategically positioned so as to generally represent
transects away from major rivers discharging into the GBR.  As such, they likely represent biased
estimates of the water parameters of the surrounding water bodies.  Nevertheless, the observed data
are direct measurements of a range of parameters considered to be important measures of water
quality and are therefore considered to be relatively accurate estimates of the true state - albeit
for a potentially narrow (and biased) spatio-temporal window.  By contrast, the Satellite data represent indirect
proxies for some of these parameters (Chlorophyll-a, Total Suspended Solids and Secchi Depth) and
similarly, the eReefs data are indirect modelled estimates simulated from a deterministic
manifestation of a conceptual model.  Hence, to gauge the accuracy of the Satellite and eReefs data
(and thus inform qualitative confidence), time series and spatial patterns in the Satellite and
eReefs observations were compared to the AIMS Insitu observations.
 
The disparate spatio-temporal resolutions of the data sources present substantial challenges for
extracting comparable data.  For example, the proximity of AIMS Insitu samples to reefs and the
spatial resolution (1km or 4km grid) frequently results in an inability to obtain matching spatial
location for all three sources\footnote{Satellite data and eReefs models are of limited value
in shallow water}}.  Furthermore, gaps in the Satellite time series frequently prevent
matching Satellite data to the same day as AIMS Insitu sampling. Compounding these issues is the added
inherent complications and added noise associated with the inability to control exactly when sampling occurs in
throughout dynamic environments.  For example, Insitu samples are collected when (date as well as
time of day) based largely on logistics and availability of acceptable Satellite data are determined
by when the satellite passes over the GBR as whether the data are of sufficient quality\footnote{Effected by
light levels, viewing angle, cloud cover etc.}.

The degree to which the discrete AIMS Insitu samples reflect space and time around the actual
sampling sites/times is largely unknown.  That is, it is not clear how broadly representative the
direct observations are.  Consequently, it is difficult to estimate how broadly to filter the
Satellite and eReefs data in space and time around the AIMS Insitu sampling events in order to
generate comparable data.  The 'best' breadth is likely to be a compromise between data availability
(time limited for Satellite and space limited for eReefs) and data equivalence (the degree to
which samples from different sources are considered to represent the same spatio-temporal unit).

Figures \ref{fig:satellite_vs_niskin_locations_5km} and \ref{fig:eReefs_vs_niskin_locations_5km}
illustrate the spatial distribution of Satellite and eReefs grid cell centroid locations relative to
the AIMS Insitu sampling locations.  The different color spokes denote distance categories (red:
<1km, olive: <2km, aqua: <3km and purple: <4km) from the AIMS Insitu data.

The approach we took was to extract all observations within a specific series of spatio-temporal
windows or neighbourhoods from which we could calculate a range of association and correspondence
(such as RMSE and $R^2$) metrics (see Tables \ref{tab:association_metrics},
\ref{tab:comp.all.rmse.sum.max}, \ref{tab:comp.all.mae.sum.max} \& \ref{tab:comp.all.mpe.sum.max}).  Tables
\ref{tab:comp.all.rmse.sum.max}, \ref{tab:comp.all.mae.sum.max} \& \ref{tab:comp.all.mpe.sum.max} document the
top 5 ranked (according to RMSE, MAE and MAPE respectively) spatio-temporal lag associations between
Satellite/eReefs data and AIMS Insitu data.
   
 
\begin{table}[htb]
  \caption{Association and correspondence metrics between Satellite/eReefs observations ($\hat{\theta_i}$) and AIMS Niskin observations ($\theta_i$). Similar calculations can
  be performed on model residuals.}\label{tab:association_metrics}
  \begin{tabular}{llc}
    \toprule
    Metric & Description & Formulation \\
    \midrule\\[-0.5em]
    RMSE &
           \begin{minipage}[t]{9.5cm}Root Mean Square Error - is a measure of accuracy \end{minipage} &
           $RMSE(\hat{\theta}) = \sqrt{\frac{1}{n}\displaystyle\sum^{n}_{i=1} (\hat{\theta_i} - \theta_i)^2}$\\
    MAE &
           \begin{minipage}[t]{9.5cm}Mean Absolute Error - is a measure of accuracy \end{minipage} &
           $MAE(\hat{\theta}) = \frac{1}{n}\displaystyle\sum^{n}_{i=1} |(\hat{\theta_i} - \theta_i)|$ \\
    MAPE &
           \begin{minipage}[t]{9.5cm}Mean Absolute Percentage Error - is a measure of accuracy expressed as a percentage of AIMS insitu samples\end{minipage} &
           $MAPE(\hat{\theta}) = \frac{100}{n}\displaystyle\sum^{n}_{i=1} \middle|\frac{\hat{\theta_i} - \theta_i}{\theta_i}\middle|$ \\
    \bottomrule
  \end{tabular}
\end{table}


Whilst it is well established that water quality parameters can be highly varied over time and
space, even approximate degrees of spatio-temporal autocorrelation for these parameters remain
largely unknown.  Nevertheless, we might expect that observations from different sources collected
at similar locations and at similar times should be more similar to one another than they are to
more distal observations.  Furthermore, whilst the absolute values derived from different sources
might not be exactly the same, we should expect a reasonable degree of correlation between the
sources.  Given these two positions (that observations should be autocorrelated and that different
sources should be correlated), we should expect that the degree of correlation between the different
sources for a given measure should be strongest for observation pairs closer together in space and
time.
  
Tables~\ref{tab:comp.all.rmse.sum.max} -- \ref{tab:comp.all.mpe.sum.max} tabulate the association
and correspondence metrics between the AIMS insitu samples and either the Satellite or eReefs data
for each Measure.  Irrespective of the association metric (RMSE, MAE or MAPE), closest associations
with AIMS insitu observations tend to occur at shorter spatial distances for eReefs data than
Satellite data, yet the opposite is apparent for temporal lags.  We might have expected that
associations would be strongest proximal (in both time and space) to the AIMS insitu samples and
associations to weaken in some sort of multidimentional decaying pattern with increasing separation.
Such a pattern would permit relatively straight forward integration of the AIMS insitu observational
data into the Satellite or eReefs layers\footnote{Having a robust and consistent pattern of spatial
and temporal autocorrelation would allow us to model the expected value of AIMS insitu data at
unobserved locations.}  However this is not the case and thus it is very difficult to formulate an
integration routine that does more than just update a very limited number of points in space and
time.

The other rationale for exploring the spatio-temporal associations between AIMS insitu data and
Satellite/eReefs data is to be able to determine the optimal temporal lag and spatial distance
for making comparisons of trends.  Given that AIMS insitu data are in some respects considered the
more accurate (albeit limited in the degree to which they more broadly represent space and time
around the samples), a comparison of the general temporal trends of each source should give some
idea of the relative accuracy of the sources of indirect measurements (Satellite and eReefs).
Figures.~\ref{fig:chl_eReefs_vs_vs_Satellite_niskin_5km_natural} --
\ref{fig:NOx_eReefs_vs_vs_Satellite_niskin_5km_natural} illustrate the temporal patterns of
Chlorophyll-a, TSS, Secchi depth and NOx for each source (AIMS insitu, AIMS FLNTU, Satellite, eReefs
and eReefs926) for each of the AIMS insitu sampling locations.  The background fills of the site
titles are colored according to water body (Red: Enclosed Coastal, Green: Open Coastal, Blue:
Midshelf).

All sources of data are typically most variable at Enclosed Coastal sites and substantially less
variable at Midshelf sites.  Moreover, the alignment of trends also appears to be substantially
better at Midshore sites.  Enclosed Coastal and Open Coastal sites are closer to the coasts and in
particular, closer to major sources of discharge (as intended by the AIMS Water Quality MMP) whereby
water conditions are subject to more extreme fluctuations that result in conditions varying rapidly
in time and space.  Moreover, these sites are likely to be in shallower water or water whose depth
is relatively heterogeneous. As a result, data pooled within a 5km radius might represent a
substantially different body of water than that represented by the AIMS insitu point sources.  By
contrast, the conditions represented within a 5km radius at Midshelf sites are likely to be more
homogeneous and thereby resulting in a fairer comparison.

Notwithstanding the disparity in fairness between different water bodies as a result of how well the
various sources represent spatial and temporal envelopes, it is unlikely that either the eReefs 4km
models or Satellite data are going to provide accurate estimates for Enclosed Coastal water bodies.
%However, the accuracy for Midshelf and Offshore are likely to be sufficient.


\begin{landscape}
  \begin{table}[ht]
\centering
\caption{Top five ranked AIMS Niskin vs Satellite/eReefs observation association metrics (RMSE: root mean square error, MAE: mean absolute error, MAPE: mean percent error, Value: regression slope, residual.RMSE: residual root mean square error, residual.MAE: residual mean absolute error, R2.marginal: $R^2$ marginalized over sites, R2.conditional: $R^2$ conditional on sites) per Measure per source (Satellite, eReefs) for spatial/temporal lags.  Rows ranked and filtered based on RMSE. Dist and Lag represent spatial (km) and temporal (days) lags.} 
\label{tab:comp.all.rmse.sum.max}
\begingroup\scriptsize
\begin{tabular}{llrrrrrrrrrrrrrr}
  \toprule
Source & Measure & Dist & Lag & RMSE & MAE & MAPE & Value & Std.Error & DF & t.value & p.value & residual.RMSE & residual.MAE & R2.marginal & R2.conditional \\ 
  \midrule
Satellite & chl & 8.00 & 6.00 & 0.33 & 0.22 & 0.69 & 0.42 & 0.04 & 566.00 & 11.43 & 0.00 & 0.22 & 0.14 & 0.10 & 0.66 \\ 
  Satellite & chl & 9.00 & 6.00 & 0.33 & 0.22 & 0.69 & 0.42 & 0.04 & 566.00 & 11.37 & 0.00 & 0.22 & 0.14 & 0.09 & 0.67 \\ 
  Satellite & chl & 6.00 & 6.00 & 0.33 & 0.22 & 0.69 & 0.43 & 0.04 & 566.00 & 11.54 & 0.00 & 0.22 & 0.14 & 0.10 & 0.65 \\ 
  Satellite & chl & 10.00 & 6.00 & 0.33 & 0.22 & 0.69 & 0.42 & 0.04 & 566.00 & 11.30 & 0.00 & 0.22 & 0.13 & 0.09 & 0.67 \\ 
  Satellite & chl & 11.00 & 6.00 & 0.33 & 0.22 & 0.69 & 0.42 & 0.04 & 566.00 & 11.27 & 0.00 & 0.22 & 0.13 & 0.09 & 0.67 \\ 
  eReefs & chl & 1.00 & 5.00 & 0.34 & 0.24 & 0.44 & 0.13 & 0.03 & 96.00 & 3.67 & 0.00 & 0.10 & 0.08 & 0.08 & 0.48 \\ 
  eReefs & chl & 1.00 & 4.00 & 0.34 & 0.24 & 0.44 & 0.14 & 0.04 & 96.00 & 3.85 & 0.00 & 0.10 & 0.08 & 0.09 & 0.48 \\ 
  eReefs & chl & 1.00 & 6.00 & 0.34 & 0.24 & 0.45 & 0.12 & 0.03 & 96.00 & 3.63 & 0.00 & 0.09 & 0.08 & 0.08 & 0.49 \\ 
  eReefs & chl & 1.00 & 3.00 & 0.34 & 0.24 & 0.45 & 0.16 & 0.04 & 96.00 & 3.76 & 0.00 & 0.12 & 0.09 & 0.09 & 0.42 \\ 
  eReefs & chl & 1.00 & 7.00 & 0.34 & 0.24 & 0.45 & 0.11 & 0.03 & 96.00 & 3.46 & 0.00 & 0.09 & 0.07 & 0.07 & 0.50 \\ 
   \midrule
Satellite & nap & 4.00 & 1.00 & 1.65 & 0.90 & 1.02 & 0.48 & 0.03 & 432.00 & 16.60 & 0.00 & 1.15 & 0.54 & 0.40 & 0.45 \\ 
  Satellite & nap & 1.00 & 1.00 & 1.66 & 0.87 & 1.08 & 0.54 & 0.04 & 358.00 & 14.58 & 0.00 & 1.30 & 0.57 & 0.38 & 0.45 \\ 
  Satellite & nap & 4.00 & 0.00 & 1.67 & 0.87 & 1.21 & 0.51 & 0.04 & 225.00 & 13.99 & 0.00 & 1.17 & 0.52 & 0.45 & 0.49 \\ 
  Satellite & nap & 3.00 & 1.00 & 1.70 & 0.91 & 0.97 & 0.47 & 0.03 & 427.00 & 15.41 & 0.00 & 1.19 & 0.55 & 0.37 & 0.43 \\ 
  Satellite & nap & 3.00 & 0.00 & 1.73 & 0.90 & 1.11 & 0.54 & 0.04 & 214.00 & 13.28 & 0.00 & 1.23 & 0.57 & 0.43 & 0.53 \\ 
  eReefs & nap & 5.00 & 3.00 & 2.07 & 1.18 & 0.73 & 0.12 & 0.02 & 239.00 & 6.20 & 0.00 & 0.57 & 0.38 & 0.13 & 0.16 \\ 
  eReefs & nap & 5.00 & 4.00 & 2.07 & 1.17 & 0.73 & 0.11 & 0.02 & 239.00 & 5.51 & 0.00 & 0.56 & 0.39 & 0.11 & 0.16 \\ 
  eReefs & nap & 4.00 & 3.00 & 2.08 & 1.17 & 0.70 & 0.11 & 0.02 & 239.00 & 5.78 & 0.00 & 0.53 & 0.37 & 0.12 & 0.18 \\ 
  eReefs & nap & 4.00 & 4.00 & 2.08 & 1.16 & 0.70 & 0.09 & 0.02 & 239.00 & 5.03 & 0.00 & 0.54 & 0.39 & 0.09 & 0.16 \\ 
  eReefs & nap & 5.00 & 2.00 & 2.08 & 1.18 & 0.74 & 0.12 & 0.02 & 239.00 & 6.00 & 0.00 & 0.57 & 0.39 & 0.13 & 0.16 \\ 
   \midrule
Satellite & sd & 5.00 & 2.00 & 4.47 & 3.38 & 0.44 & 0.11 & 0.01 & 463.00 & 11.77 & 0.00 & 0.55 & 0.42 & 0.24 & 0.54 \\ 
  Satellite & sd & 4.00 & 2.00 & 4.48 & 3.38 & 0.44 & 0.11 & 0.01 & 462.00 & 11.71 & 0.00 & 0.56 & 0.42 & 0.24 & 0.52 \\ 
  Satellite & sd & 3.00 & 2.00 & 4.48 & 3.39 & 0.44 & 0.12 & 0.01 & 455.00 & 11.73 & 0.00 & 0.57 & 0.42 & 0.25 & 0.51 \\ 
  Satellite & sd & 11.00 & 2.00 & 4.48 & 3.37 & 0.44 & 0.11 & 0.01 & 470.00 & 11.65 & 0.00 & 0.53 & 0.41 & 0.20 & 0.61 \\ 
  Satellite & sd & 12.00 & 2.00 & 4.48 & 3.37 & 0.44 & 0.11 & 0.01 & 470.00 & 11.65 & 0.00 & 0.53 & 0.41 & 0.20 & 0.61 \\ 
  eReefs & sd & 4.00 & 1.00 & 13.13 & 11.31 & 2.37 & 1.23 & 0.12 & 196.00 & 10.39 & 0.00 & 6.47 & 4.92 & 0.35 & 0.37 \\ 
  eReefs & sd & 4.00 & 2.00 & 13.29 & 11.68 & 2.49 & 1.14 & 0.11 & 196.00 & 9.89 & 0.00 & 6.10 & 4.75 & 0.34 & 0.37 \\ 
  eReefs & sd & 5.00 & 1.00 & 13.46 & 11.62 & 2.36 & 1.29 & 0.12 & 185.00 & 10.81 & 0.00 & 6.61 & 5.12 & 0.38 & 0.39 \\ 
  eReefs & sd & 6.00 & 1.00 & 13.53 & 11.69 & 2.37 & 1.30 & 0.13 & 185.00 & 10.40 & 0.00 & 6.43 & 4.96 & 0.38 & 0.41 \\ 
  eReefs & sd & 5.00 & 2.00 & 13.66 & 12.02 & 2.48 & 1.18 & 0.12 & 185.00 & 10.20 & 0.00 & 6.30 & 5.02 & 0.36 & 0.37 \\ 
   \bottomrule
\end{tabular}
\endgroup
\end{table}
 
  \begin{table}[ht]
\centering
\caption{Top five ranked AIMS Niskin vs Satellite/eReefs observation association metrics (RMSE: root mean square error, MAE: mean absolute error, MAPE: mean percent error, Value: regression slope, residual.RMSE: residual root mean square error, residual.MAE: residual mean absolute error, R2.marginal: $R^2$ marginalized over sites, R2.conditional: $R^2$ conditional on sites) per Measure per source (Satellite, eReefs) for spatial/temporal lags.  Rows ranked and filtered based on MAE. Dist and Lag represent spatial (km) and temporal (days) lags.} 
\label{tab:comp.all.mae.sum.max}
\begingroup\scriptsize
\begin{tabular}{llrrrrrrrrrrrrrr}
  \toprule
Source & Measure & Dist & Lag & RMSE & MAE & MAPE & Value & Std.Error & DF & t.value & p.value & residual.RMSE & residual.MAE & R2.marginal & R2.conditional \\ 
  \midrule
Satellite & chl & 10.00 & 0.00 & 0.38 & 0.21 & 0.64 & 0.82 & 0.08 & 253.00 & 9.99 & 0.00 & 0.33 & 0.17 & 0.27 & 0.37 \\ 
  Satellite & chl & 11.00 & 0.00 & 0.38 & 0.21 & 0.65 & 0.81 & 0.08 & 254.00 & 9.89 & 0.00 & 0.33 & 0.17 & 0.26 & 0.38 \\ 
  Satellite & chl & 12.00 & 0.00 & 0.38 & 0.21 & 0.65 & 0.81 & 0.08 & 254.00 & 9.89 & 0.00 & 0.33 & 0.17 & 0.26 & 0.38 \\ 
  Satellite & chl & 4.00 & 0.00 & 0.38 & 0.21 & 0.65 & 0.91 & 0.08 & 226.00 & 10.82 & 0.00 & 0.33 & 0.17 & 0.32 & 0.44 \\ 
  Satellite & chl & 9.00 & 0.00 & 0.39 & 0.21 & 0.64 & 0.84 & 0.09 & 250.00 & 9.86 & 0.00 & 0.35 & 0.17 & 0.27 & 0.36 \\ 
  eReefs & chl & 3.00 & 5.00 & 0.34 & 0.23 & 0.43 & 0.14 & 0.02 & 221.00 & 6.09 & 0.00 & 0.09 & 0.08 & 0.11 & 0.46 \\ 
  eReefs & chl & 3.00 & 6.00 & 0.34 & 0.23 & 0.43 & 0.13 & 0.02 & 221.00 & 6.09 & 0.00 & 0.09 & 0.07 & 0.11 & 0.46 \\ 
  eReefs & chl & 3.00 & 4.00 & 0.34 & 0.23 & 0.43 & 0.14 & 0.02 & 221.00 & 6.09 & 0.00 & 0.10 & 0.08 & 0.11 & 0.45 \\ 
  eReefs & chl & 3.00 & 7.00 & 0.35 & 0.23 & 0.43 & 0.12 & 0.02 & 221.00 & 5.88 & 0.00 & 0.09 & 0.07 & 0.10 & 0.46 \\ 
  eReefs & chl & 4.00 & 5.00 & 0.34 & 0.23 & 0.43 & 0.13 & 0.02 & 239.00 & 5.98 & 0.00 & 0.09 & 0.07 & 0.10 & 0.46 \\ 
   \midrule
Satellite & nap & 4.00 & 0.00 & 1.67 & 0.87 & 1.21 & 0.51 & 0.04 & 225.00 & 13.99 & 0.00 & 1.17 & 0.52 & 0.45 & 0.49 \\ 
  Satellite & nap & 1.00 & 1.00 & 1.66 & 0.87 & 1.08 & 0.54 & 0.04 & 358.00 & 14.58 & 0.00 & 1.30 & 0.57 & 0.38 & 0.45 \\ 
  Satellite & nap & 4.00 & 1.00 & 1.65 & 0.90 & 1.02 & 0.48 & 0.03 & 432.00 & 16.60 & 0.00 & 1.15 & 0.54 & 0.40 & 0.45 \\ 
  Satellite & nap & 3.00 & 0.00 & 1.73 & 0.90 & 1.11 & 0.54 & 0.04 & 214.00 & 13.28 & 0.00 & 1.23 & 0.57 & 0.43 & 0.53 \\ 
  Satellite & nap & 3.00 & 1.00 & 1.70 & 0.91 & 0.97 & 0.47 & 0.03 & 427.00 & 15.41 & 0.00 & 1.19 & 0.55 & 0.37 & 0.43 \\ 
  eReefs & nap & 4.00 & 4.00 & 2.08 & 1.16 & 0.70 & 0.09 & 0.02 & 239.00 & 5.03 & 0.00 & 0.54 & 0.39 & 0.09 & 0.16 \\ 
  eReefs & nap & 4.00 & 3.00 & 2.08 & 1.17 & 0.70 & 0.11 & 0.02 & 239.00 & 5.78 & 0.00 & 0.53 & 0.37 & 0.12 & 0.18 \\ 
  eReefs & nap & 4.00 & 2.00 & 2.09 & 1.17 & 0.72 & 0.11 & 0.02 & 239.00 & 5.52 & 0.00 & 0.55 & 0.38 & 0.11 & 0.18 \\ 
  eReefs & nap & 5.00 & 4.00 & 2.07 & 1.17 & 0.73 & 0.11 & 0.02 & 239.00 & 5.51 & 0.00 & 0.56 & 0.39 & 0.11 & 0.16 \\ 
  eReefs & nap & 5.00 & 3.00 & 2.07 & 1.18 & 0.73 & 0.12 & 0.02 & 239.00 & 6.20 & 0.00 & 0.57 & 0.38 & 0.13 & 0.16 \\ 
   \midrule
Satellite & sd & 11.00 & 2.00 & 4.48 & 3.37 & 0.44 & 0.11 & 0.01 & 470.00 & 11.65 & 0.00 & 0.53 & 0.41 & 0.20 & 0.61 \\ 
  Satellite & sd & 12.00 & 2.00 & 4.48 & 3.37 & 0.44 & 0.11 & 0.01 & 470.00 & 11.65 & 0.00 & 0.53 & 0.41 & 0.20 & 0.61 \\ 
  Satellite & sd & 10.00 & 2.00 & 4.48 & 3.37 & 0.44 & 0.11 & 0.01 & 470.00 & 11.67 & 0.00 & 0.53 & 0.41 & 0.20 & 0.61 \\ 
  Satellite & sd & 4.00 & 2.00 & 4.48 & 3.38 & 0.44 & 0.11 & 0.01 & 462.00 & 11.71 & 0.00 & 0.56 & 0.42 & 0.24 & 0.52 \\ 
  Satellite & sd & 9.00 & 2.00 & 4.49 & 3.38 & 0.44 & 0.11 & 0.01 & 468.00 & 11.89 & 0.00 & 0.53 & 0.41 & 0.22 & 0.60 \\ 
  eReefs & sd & 4.00 & 1.00 & 13.13 & 11.31 & 2.37 & 1.23 & 0.12 & 196.00 & 10.39 & 0.00 & 6.47 & 4.92 & 0.35 & 0.37 \\ 
  eReefs & sd & 1.00 & 1.00 & 14.04 & 11.52 & 2.73 & 1.10 & 0.29 & 85.00 & 3.86 & 0.00 & 7.61 & 5.43 & 0.15 & 0.22 \\ 
  eReefs & sd & 1.00 & 2.00 & 13.71 & 11.58 & 2.79 & 1.12 & 0.26 & 85.00 & 4.31 & 0.00 & 6.87 & 5.36 & 0.18 & 0.26 \\ 
  eReefs & sd & 5.00 & 1.00 & 13.46 & 11.62 & 2.36 & 1.29 & 0.12 & 185.00 & 10.81 & 0.00 & 6.61 & 5.12 & 0.38 & 0.39 \\ 
  eReefs & sd & 4.00 & 2.00 & 13.29 & 11.68 & 2.49 & 1.14 & 0.11 & 196.00 & 9.89 & 0.00 & 6.10 & 4.75 & 0.34 & 0.37 \\ 
  \end{tabular}
\endgroup
\end{table}
   
  \begin{table}[ht]
\centering
\caption{Top five ranked AIMS Niskin vs Satellite/eReefs observation association metrics (RMSE: root mean square error, MAE: mean absolute error, MAPE: mean percent error, Value: regression slope, residual.RMSE: residual root mean square error, residual.MAE: residual mean absolute error, R2.marginal: $R^2$ marginalized over sites, R2.conditional: $R^2$ conditional on sites) per Measure per source (Satellite, eReefs) for spatial/temporal lags.  Rows ranked and filtered based on MAPE. Dist and Lag represent spatial (km) and temporal (days) lags.} 
\label{tab:comp.all.mpe.sum.max}
\begingroup\scriptsize
\begin{tabular}{llrrrrrrrrrrrrrr}
  \toprule
Source & Measure & Dist & Lag & RMSE & MAE & MAPE & Value & Std.Error & DF & t.value & p.value & residual.RMSE & residual.MAE & R2.marginal & R2.conditional \\ 
  \midrule
Satellite & chl & 4.00 & 2.00 & 0.37 & 0.21 & 0.62 & 0.64 & 0.05 & 508.00 & 12.12 & 0.00 & 0.30 & 0.15 & 0.18 & 0.48 \\ 
  Satellite & chl & 3.00 & 2.00 & 0.37 & 0.21 & 0.63 & 0.67 & 0.05 & 501.00 & 12.20 & 0.00 & 0.30 & 0.15 & 0.19 & 0.46 \\ 
  Satellite & chl & 2.00 & 2.00 & 0.35 & 0.21 & 0.63 & 0.63 & 0.05 & 492.00 & 12.64 & 0.00 & 0.27 & 0.15 & 0.19 & 0.54 \\ 
  Satellite & chl & 8.00 & 0.00 & 0.41 & 0.21 & 0.64 & 0.87 & 0.09 & 248.00 & 9.86 & 0.00 & 0.36 & 0.17 & 0.27 & 0.34 \\ 
  Satellite & chl & 10.00 & 0.00 & 0.38 & 0.21 & 0.64 & 0.82 & 0.08 & 253.00 & 9.99 & 0.00 & 0.33 & 0.17 & 0.27 & 0.37 \\ 
  eReefs & chl & 3.00 & 6.00 & 0.34 & 0.23 & 0.43 & 0.13 & 0.02 & 221.00 & 6.09 & 0.00 & 0.09 & 0.07 & 0.11 & 0.46 \\ 
  eReefs & chl & 4.00 & 6.00 & 0.34 & 0.23 & 0.43 & 0.12 & 0.02 & 239.00 & 6.03 & 0.00 & 0.09 & 0.07 & 0.10 & 0.47 \\ 
  eReefs & chl & 3.00 & 5.00 & 0.34 & 0.23 & 0.43 & 0.14 & 0.02 & 221.00 & 6.09 & 0.00 & 0.09 & 0.08 & 0.11 & 0.46 \\ 
  eReefs & chl & 2.00 & 6.00 & 0.35 & 0.23 & 0.43 & 0.13 & 0.02 & 195.00 & 5.72 & 0.00 & 0.09 & 0.07 & 0.11 & 0.45 \\ 
  eReefs & chl & 3.00 & 7.00 & 0.35 & 0.23 & 0.43 & 0.12 & 0.02 & 221.00 & 5.88 & 0.00 & 0.09 & 0.07 & 0.10 & 0.46 \\ 
   \midrule
Satellite & nap & 3.00 & 2.00 & 1.76 & 0.95 & 0.90 & 0.35 & 0.02 & 500.00 & 15.62 & 0.00 & 0.94 & 0.50 & 0.31 & 0.50 \\ 
  Satellite & nap & 2.00 & 2.00 & 1.81 & 0.96 & 0.91 & 0.35 & 0.02 & 491.00 & 14.78 & 0.00 & 0.97 & 0.50 & 0.27 & 0.52 \\ 
  Satellite & nap & 7.00 & 2.00 & 1.88 & 1.00 & 0.93 & 0.34 & 0.02 & 514.00 & 13.50 & 0.00 & 1.04 & 0.54 & 0.22 & 0.52 \\ 
  Satellite & nap & 8.00 & 2.00 & 1.88 & 1.01 & 0.93 & 0.33 & 0.02 & 514.00 & 13.35 & 0.00 & 1.03 & 0.54 & 0.21 & 0.54 \\ 
  Satellite & nap & 9.00 & 2.00 & 1.88 & 1.01 & 0.93 & 0.33 & 0.02 & 514.00 & 13.43 & 0.00 & 1.01 & 0.53 & 0.20 & 0.56 \\ 
  eReefs & nap & 1.00 & 4.00 & 2.34 & 1.36 & 0.68 & 0.10 & 0.03 & 96.00 & 3.12 & 0.00 & 0.76 & 0.50 & 0.08 & 0.08 \\ 
  eReefs & nap & 1.00 & 3.00 & 2.37 & 1.37 & 0.68 & 0.12 & 0.04 & 96.00 & 3.11 & 0.00 & 0.87 & 0.51 & 0.08 & 0.08 \\ 
  eReefs & nap & 11.00 & 4.00 & 2.57 & 1.28 & 0.69 & 0.07 & 0.02 & 246.00 & 4.48 & 0.00 & 0.55 & 0.39 & 0.07 & 0.17 \\ 
  eReefs & nap & 12.00 & 4.00 & 2.57 & 1.28 & 0.69 & 0.07 & 0.02 & 246.00 & 4.49 & 0.00 & 0.55 & 0.39 & 0.07 & 0.17 \\ 
  eReefs & nap & 10.00 & 4.00 & 2.57 & 1.28 & 0.69 & 0.07 & 0.02 & 246.00 & 4.45 & 0.00 & 0.56 & 0.39 & 0.07 & 0.16 \\ 
   \midrule
Satellite & sd & 6.00 & 0.00 & 4.64 & 3.50 & 0.43 & 0.16 & 0.02 & 217.00 & 10.16 & 0.00 & 0.74 & 0.54 & 0.34 & 0.42 \\ 
  Satellite & sd & 4.00 & 0.00 & 4.73 & 3.59 & 0.43 & 0.16 & 0.01 & 207.00 & 11.42 & 0.00 & 0.70 & 0.54 & 0.40 & 0.45 \\ 
  Satellite & sd & 7.00 & 0.00 & 4.63 & 3.51 & 0.43 & 0.15 & 0.02 & 224.00 & 10.00 & 0.00 & 0.73 & 0.55 & 0.33 & 0.41 \\ 
  Satellite & sd & 10.00 & 0.00 & 4.62 & 3.50 & 0.43 & 0.15 & 0.02 & 231.00 & 9.27 & 0.00 & 0.75 & 0.57 & 0.29 & 0.38 \\ 
  Satellite & sd & 5.00 & 0.00 & 4.70 & 3.56 & 0.43 & 0.16 & 0.01 & 211.00 & 11.05 & 0.00 & 0.70 & 0.53 & 0.38 & 0.44 \\ 
  eReefs & sd & 5.00 & 1.00 & 13.46 & 11.62 & 2.36 & 1.29 & 0.12 & 185.00 & 10.81 & 0.00 & 6.61 & 5.12 & 0.38 & 0.39 \\ 
  eReefs & sd & 4.00 & 1.00 & 13.13 & 11.31 & 2.37 & 1.23 & 0.12 & 196.00 & 10.39 & 0.00 & 6.47 & 4.92 & 0.35 & 0.37 \\ 
  eReefs & sd & 6.00 & 1.00 & 13.53 & 11.69 & 2.37 & 1.30 & 0.13 & 185.00 & 10.40 & 0.00 & 6.43 & 4.96 & 0.38 & 0.41 \\ 
  eReefs & sd & 8.00 & 1.00 & 13.91 & 12.00 & 2.38 & 1.38 & 0.13 & 185.00 & 10.31 & 0.00 & 6.39 & 4.97 & 0.40 & 0.45 \\ 
  eReefs & sd & 7.00 & 1.00 & 13.75 & 11.88 & 2.39 & 1.33 & 0.13 & 185.00 & 10.30 & 0.00 & 6.45 & 4.98 & 0.38 & 0.42 \\ 
   \bottomrule
\end{tabular}
\endgroup
\end{table}

\end{landscape}


\begin{figure}[htp]
  \includegraphics[width=0.9\linewidth]{{figures/Analyses at AIMS niskin sites/Observations/Satellite_vs_Niskin_locations_5km\res}.png}
\caption{Location of Satellite cells within 5km of AIMS niskin samples. Panel borders represent water bodies (Red: Enclosed Coastal, Green: Open Coastal, Blue: Midshelf).}\label{fig:satellite_vs_niskin_locations_5km}	
\end{figure}

\begin{figure}[htp] 
\includegraphics[width=0.9\linewidth]{{figures/Analyses at AIMS niskin sites/Observations/eReefs_vs_Niskin_locations_5km\res}.png}
\caption{Location of eReefs cells within 5km of AIMS niskin samples.  Panel borders represent water bodies (Red: Enclosed Coastal, Green: Open Coastal, Blue: Midshelf).}\label{fig:eReefs_vs_niskin_locations_5km}	
\end{figure}

\begin{landscape}
\begin{figure}[htp]
  %\includegraphics[width=0.9\linewidth]{{figures/Analyses at AIMS niskin sites/Observations/chl_eReefs_vs_Satellite_vs_Niskin_.Radius_5_natural\res}.png}
  \includegraphics[width=0.9\linewidth]{{figures/Analyses at AIMS niskin sites/Observations/chl_eReefs_vs_Satellite_vs_Niskin_.Radius_5_natural\res}.pdf}
\caption[Temporal patterns in Chlorophyll-a within 5km of each AIMS MMP sampling site for eReefs, Satellite and AIMS insitu and FLNTU logger sources]{Temporal patterns in Chlorophyll-a within 5km of each AIMS MMP sampling site for eReefs, Satellite and AIMS insitu and FLNTU logger sources. Horizontal dashed line represents the guideline value. Title backgrounds represent water bodies (Red: Enclosed Coastal, Green: Open Coastal, Blue: Midshelf).}\label{fig:chl_eReefs_vs_vs_Satellite_niskin_5km_natural}	
\end{figure}		
\end{landscape}

\begin{landscape}
\begin{figure}[htp]
  %\includegraphics[width=0.9\linewidth]{{figures/Analyses at AIMS niskin sites/Observations/nap_eReefs_vs_Satellite_vs_Niskin_.Radius_5_natural\res}.png}
  \includegraphics[width=0.9\linewidth]{{figures/Analyses at AIMS niskin sites/Observations/nap_eReefs_vs_Satellite_vs_Niskin_.Radius_5_natural\res}.pdf}
\caption[Temporal patterns in TSS within 5km of each AIMS MMP sampling site for eReefs, Satellite and AIMS insitu and FLNTU logger sources]{Temporal patterns in TSS within 5km of each AIMS MMP sampling site for eReefs, Satellite and AIMS insitu and FLNTU logger sources. Horizontal dashed line represents the guideline value. Title backgrounds represent water bodies (Red: Enclosed Coastal, Green: Open Coastal, Blue: Midshelf).}\label{fig:nap_eReefs_vs_vs_Satellite_niskin_5km_natural}	
\end{figure}		
\end{landscape}

\begin{landscape}
\begin{figure}[htp]
  %\includegraphics[width=0.9\linewidth]{{figures/Analyses at AIMS niskin sites/Observations/sd_eReefs_vs_Satellite_vs_Niskin_.Radius_5_natural\res}.png}
  \includegraphics[width=0.9\linewidth]{{figures/Analyses at AIMS niskin sites/Observations/sd_eReefs_vs_Satellite_vs_Niskin_.Radius_5_natural\res}.pdf}
\caption[Temporal patterns in Secchi depth within 5km of each AIMS MMP sampling site for eReefs, Satellite and AIMS insitu and FLNTU logger sources]{Temporal patterns in Secchi Depth within 5km of each AIMS MMP sampling site for eReefs, Satellite and AIMS insitu and FLNTU logger sources. Horizontal dashed line represents the guideline value. Title backgrounds represent water bodies (Red: Enclosed Coastal, Green: Open Coastal, Blue: Midshelf).}\label{fig:sd_eReefs_vs_vs_Satellite_niskin_5km_natural}	
\end{figure}		
\end{landscape}

\begin{landscape}
\begin{figure}[htp]
  %\includegraphics[width=0.9\linewidth]{{figures/Analyses at AIMS niskin sites/Observations/NOx_eReefs_vs_Satellite_vs_Niskin_.Radius_5_natural\res}.png}
  \includegraphics[width=0.9\linewidth]{{figures/Analyses at AIMS niskin sites/Observations/NOx_eReefs_vs_Satellite_vs_Niskin_.Radius_5_natural\res}.pdf}
\caption[Temporal patterns in NOx within 5km of each AIMS MMP sampling site for eReefs, Satellite and AIMS insitu and FLNTU logger sources]{Temporal patterns in NOx within 5km of each AIMS MMP sampling site for eReefs, Satellite and AIMS insitu and FLNTU logger sources. Horizontal dashed line represents the guideline value. Title backgrounds represent water bodies (Red: Enclosed Coastal, Green: Open Coastal, Blue: Midshelf).}\label{fig:NOx_eReefs_vs_vs_Satellite_niskin_5km_natural}	
\end{figure}		
\end{landscape}

