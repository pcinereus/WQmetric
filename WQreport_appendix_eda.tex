\subsection{Annual data}\label{appsec:EDA}
\subsubsection{Cape York, Enclosed Coastal}
\paragraph{Chlorophyll}

\begin{figure}[ptbh] 
  %a) AIMS insitu\\\includegraphics[align=t,width=0.95\linewidth]{{figures/Exploratory_Data_Analysis/Insitu/eda.year.chl_Wet Tropics__Open Coastal_niskin_log\res}.pdf}\\
  %b) AIMS FLNTU\\\includegraphics[align=t, width=0.95\linewidth]{{figures/Exploratory_Data_Analysis/FLNTU/eda.year.chl_Wet Tropics__Open Coastal_flntu_log\res}.pdf}\\
  c) Satellite\\\includegraphics[align=t, width=0.95\linewidth]{{figures/Exploratory_Data_Analysis/Satellite/eda.year.chl_Cape York__Enclosed Coastal__log\res}.png}}\\
  d) eReefs\\\includegraphics[align=t, width=0.95\linewidth]{{figures/Exploratory_Data_Analysis/eReefs/eda.year.chl_Cape York__Enclosed Coastal_eReefs_log\res}.png}}\\
  e) eReefs926\\\includegraphics[align=t, width=0.95\linewidth]{{figures/Exploratory_Data_Analysis/eReefs926/eda.year.chl_Cape York__Enclosed Coastal_eReefs926_log\res}.png}}\\
\caption[Observed Chlorophyll-a data for the Cape York Enclosed Coastal Zone (grouped annually)]{Observed (logarithmic axis with violin plot overlay) Chlorophyll-a data for the
  Cape York Enclosed Coastal
  Zone from a) AIMS insitu, b) AIMS FLNTU, c) Satellite, d) eReefs and e)
eReefs926.  Observations are ordered over time and colored conditional on season as Wet (blue
symbols) and Dry (red symbols).  Blue smoother represents Generalized Additive Mixed Model within a
water year and purple line represents average within the water year.  Horizontal red, black and
green dashed lines denote the twice threshold, threshold and half threshold values respectively.
Red and green background shading indicates the range (10\% shade: x4,/4; 30\% shade: x2,/2)
above and below threshold respectively.}\label{fig:violin_chl_ce}
\end{figure}

\paragraph{Total Suspended Solids}

\begin{figure}[ptbh] 
  %a) AIMS insitu\\\includegraphics[align=t,width=0.95\linewidth]{{figures/Exploratory_Data_Analysis/Insitu/eda.year.nap_Wet Tropics__Open Coastal_niskin_log\res}.pdf}\\
  %b) AIMS FLNTU\\\includegraphics[align=t, width=0.95\linewidth]{{figures/Exploratory_Data_Analysis/FLNTU/eda.year.nap_Wet Tropics__Open Coastal_flntu_log\res}.pdf}\\
  c) Satellite\\\includegraphics[align=t, width=0.95\linewidth]{{figures/Exploratory_Data_Analysis/Satellite/eda.year.nap_Cape York__Enclosed Coastal__log\res}.png}}\\
  d) eReefs\\\includegraphics[align=t, width=0.95\linewidth]{{figures/Exploratory_Data_Analysis/eReefs/eda.year.nap_Cape York__Enclosed Coastal_eReefs_log\res}.png}}\\
  e) eReefs926\\\includegraphics[align=t, width=0.95\linewidth]{{figures/Exploratory_Data_Analysis/eReefs926/eda.year.nap_Cape York__Enclosed Coastal_eReefs926_log\res}.png}}\\
\caption[Observed TSS data for the Cape York Enclosed Coastal Zone (grouped annually)]{Observed (logarithmic axis with violin plot overlay) Total Suspended Solids data for the
  Cape York Enclosed Coastal
  Zone from a) AIMS insitu, b) AIMS FLNTU, c) Satellite, d) eReefs and e)
eReefs926.  Observations are ordered over time and colored conditional on season as Wet (blue
symbols) and Dry (red symbols).  Blue smoother represents Generalized Additive Mixed Model within a
water year and purple line represents average within the water year.  Horizontal red, black and
green dashed lines denote the twice threshold, threshold and half threshold values respectively.
Red and green background shading indicates the range (10\% shade: x4,/4; 30\% shade: x2,/2)
above and below threshold respectively.}\label{fig:violin_nap_ce}
\end{figure}

\paragraph{Secchi Depth}

\begin{figure}[ptbh] 
  %a) AIMS insitu\\\includegraphics[align=t,width=0.95\linewidth]{{figures/Exploratory_Data_Analysis/Insitu/eda.year.sd_Wet Tropics__Open Coastal_niskin_log\res}.pdf}\\
  %b) AIMS FLNTU\\\includegraphics[align=t, width=0.95\linewidth]{{figures/Exploratory_Data_Analysis/FLNTU/eda.year.sd_Wet Tropics__Open Coastal_flntu_log\res}.pdf}\\
  c) Satellite\\\includegraphics[align=t, width=0.95\linewidth]{{figures/Exploratory_Data_Analysis/Satellite/eda.year.sd_Cape York__Enclosed Coastal__log\res}.png}}\\
  d) eReefs\\\includegraphics[align=t, width=0.95\linewidth]{{figures/Exploratory_Data_Analysis/eReefs/eda.year.sd_Cape York__Enclosed Coastal_eReefs_log\res}.png}}\\
  e) eReefs926\\\includegraphics[align=t, width=0.95\linewidth]{{figures/Exploratory_Data_Analysis/eReefs926/eda.year.sd_Cape York__Enclosed Coastal_eReefs926_log\res}.png}}\\
\caption[Observed Secchi depth data for the Cape York Enclosed Coastal Zone (grouped annually)]{Observed (logarithmic axis with violin plot overlay) Secchi Depth data for the
  Cape York Enclosed Coastal
  Zone from a) AIMS insitu, b) AIMS FLNTU, c) Satellite, d) eReefs and e)
eReefs926.  Observations are ordered over time and colored conditional on season as Wet (blue
symbols) and Dry (red symbols).  Blue smoother represents Generalized Additive Mixed Model within a
water year and purple line represents average within the water year.  Horizontal red, black and
green dashed lines denote the twice threshold, threshold and half threshold values respectively.
Red and green background shading indicates the range (10\% shade: x4,/4; 30\% shade: x2,/2)
above and below threshold respectively.}\label{fig:violin_sd_ce}
\end{figure}

\paragraph{NOx}

\begin{figure}[ptbh] 
  %a) AIMS insitu\\\includegraphics[align=t,width=0.95\linewidth]{{figures/Exploratory_Data_Analysis/Insitu/eda.year.NOx_Wet Tropics__Open Coastal_niskin_log\res}.pdf}\\
  %b) AIMS FLNTU\\\includegraphics[align=t, width=0.95\linewidth]{{figures/Exploratory_Data_Analysis/FLNTU/eda.year.NOx_Wet Tropics__Open Coastal_flntu_log\res}.pdf}\\
  c) Satellite\\\includegraphics[align=t, width=0.95\linewidth]{{figures/Exploratory_Data_Analysis/Satellite/eda.year.NOx_Cape York__Enclosed Coastal__log\res}.png}}\\
  d) eReefs\\\includegraphics[align=t, width=0.95\linewidth]{{figures/Exploratory_Data_Analysis/eReefs/eda.year.NOx_Cape York__Enclosed Coastal_eReefs_log\res}.png}}\\
  e) eReefs926\\\includegraphics[align=t, width=0.95\linewidth]{{figures/Exploratory_Data_Analysis/eReefs926/eda.year.NOx_Cape York__Enclosed Coastal_eReefs926_log\res}.png}}\\
\caption[Observed NOx data for the Cape York Enclosed Coastal Zone (grouped annually)]{Observed (logarithmic axis with violin plot overlay) NOx data for the
  Cape York Enclosed Coastal
  Zone from a) AIMS insitu, b) AIMS FLNTU, c) Satellite, d) eReefs and e)
eReefs926.  Observations are ordered over time and colored conditional on season as Wet (blue
symbols) and Dry (red symbols).  Blue smoother represents Generalized Additive Mixed Model within a
water year and purple line represents average within the water year.  Horizontal red, black and
green dashed lines denote the twice threshold, threshold and half threshold values respectively.
Red and green background shading indicates the range (10\% shade: x4,/4; 30\% shade: x2,/2)
above and below threshold respectively.}\label{fig:violin_NOx_ce}
\end{figure}


\subsubsection{Cape York, Open Coastal}
\paragraph{Chlorophyll}

\begin{figure}[ptbh] 
  %a) AIMS insitu\\\includegraphics[align=t,width=0.95\linewidth]{{figures/Exploratory_Data_Analysis/Insitu/eda.year.chl_Wet Tropics__Open Coastal_niskin_log\res}.pdf}\\
  %b) AIMS FLNTU\\\includegraphics[align=t, width=0.95\linewidth]{{figures/Exploratory_Data_Analysis/FLNTU/eda.year.chl_Wet Tropics__Open Coastal_flntu_log\res}.pdf}\\
  c) Satellite\\\includegraphics[align=t, width=0.95\linewidth]{{figures/Exploratory_Data_Analysis/Satellite/eda.year.chl_Cape York__Open Coastal__log\res}.png}}\\
  d) eReefs\\\includegraphics[align=t, width=0.95\linewidth]{{figures/Exploratory_Data_Analysis/eReefs/eda.year.chl_Cape York__Open Coastal_eReefs_log\res}.png}}\\
  e) eReefs926\\\includegraphics[align=t, width=0.95\linewidth]{{figures/Exploratory_Data_Analysis/eReefs926/eda.year.chl_Cape York__Open Coastal_eReefs926_log\res}.png}}\\
\caption[Observed Chlorophyll-a data for the Cape York Open Coastal Zone (grouped annually)]{Observed (logarithmic axis with violin plot overlay) Chlorophyll-a data for the
  Cape York Open Coastal
  Zone from a) AIMS insitu, b) AIMS FLNTU, c) Satellite, d) eReefs and e)
eReefs926.  Observations are ordered over time and colored conditional on season as Wet (blue
symbols) and Dry (red symbols).  Blue smoother represents Generalized Additive Mixed Model within a
water year and purple line represents average within the water year.  Horizontal red, black and
green dashed lines denote the twice threshold, threshold and half threshold values respectively.
Red and green background shading indicates the range (10\% shade: x4,/4; 30\% shade: x2,/2)
above and below threshold respectively.}\label{fig:violin_chl_co}
\end{figure}

\paragraph{Total Suspended Solids}

\begin{figure}[ptbh] 
  %a) AIMS insitu\\\includegraphics[align=t,width=0.95\linewidth]{{figures/Exploratory_Data_Analysis/Insitu/eda.year.nap_Wet Tropics__Open Coastal_niskin_log\res}.pdf}\\
  %b) AIMS FLNTU\\\includegraphics[align=t, width=0.95\linewidth]{{figures/Exploratory_Data_Analysis/FLNTU/eda.year.nap_Wet Tropics__Open Coastal_flntu_log\res}.pdf}\\
  c) Satellite\\\includegraphics[align=t, width=0.95\linewidth]{{figures/Exploratory_Data_Analysis/Satellite/eda.year.nap_Cape York__Open Coastal__log\res}.png}}\\
  d) eReefs\\\includegraphics[align=t, width=0.95\linewidth]{{figures/Exploratory_Data_Analysis/eReefs/eda.year.nap_Cape York__Open Coastal_eReefs_log\res}.png}}\\
  e) eReefs926\\\includegraphics[align=t, width=0.95\linewidth]{{figures/Exploratory_Data_Analysis/eReefs926/eda.year.nap_Cape York__Open Coastal_eReefs926_log\res}.png}}\\
\caption[Observed TSS data for the Cape York Open Coastal Zone (grouped annually)]{Observed (logarithmic axis with violin plot overlay) Total Suspended Solids data for the
  Cape York Open Coastal
  Zone from a) AIMS insitu, b) AIMS FLNTU, c) Satellite, d) eReefs and e)
eReefs926.  Observations are ordered over time and colored conditional on season as Wet (blue
symbols) and Dry (red symbols).  Blue smoother represents Generalized Additive Mixed Model within a
water year and purple line represents average within the water year.  Horizontal red, black and
green dashed lines denote the twice threshold, threshold and half threshold values respectively.
Red and green background shading indicates the range (10\% shade: x4,/4; 30\% shade: x2,/2)
above and below threshold respectively.}\label{fig:violin_nap_co}
\end{figure}

\paragraph{Secchi Depth}

\begin{figure}[ptbh] 
  %a) AIMS insitu\\\includegraphics[align=t,width=0.95\linewidth]{{figures/Exploratory_Data_Analysis/Insitu/eda.year.sd_Wet Tropics__Open Coastal_niskin_log\res}.pdf}\\
  %b) AIMS FLNTU\\\includegraphics[align=t, width=0.95\linewidth]{{figures/Exploratory_Data_Analysis/FLNTU/eda.year.sd_Wet Tropics__Open Coastal_flntu_log\res}.pdf}\\
  c) Satellite\\\includegraphics[align=t, width=0.95\linewidth]{{figures/Exploratory_Data_Analysis/Satellite/eda.year.sd_Cape York__Open Coastal__log\res}.png}}\\
  d) eReefs\\\includegraphics[align=t, width=0.95\linewidth]{{figures/Exploratory_Data_Analysis/eReefs/eda.year.sd_Cape York__Open Coastal_eReefs_log\res}.png}}\\
  e) eReefs926\\\includegraphics[align=t, width=0.95\linewidth]{{figures/Exploratory_Data_Analysis/eReefs926/eda.year.sd_Cape York__Open Coastal_eReefs926_log\res}.png}}\\
\caption[Observed Secchi depth data for the Cape York Open Coastal Zone (grouped annually)]{Observed (logarithmic axis with violin plot overlay) Secchi Depth data for the
  Cape York Open Coastal
  Zone from a) AIMS insitu, b) AIMS FLNTU, c) Satellite, d) eReefs and e)
eReefs926.  Observations are ordered over time and colored conditional on season as Wet (blue
symbols) and Dry (red symbols).  Blue smoother represents Generalized Additive Mixed Model within a
water year and purple line represents average within the water year.  Horizontal red, black and
green dashed lines denote the twice threshold, threshold and half threshold values respectively.
Red and green background shading indicates the range (10\% shade: x4,/4; 30\% shade: x2,/2)
above and below threshold respectively.}\label{fig:violin_sd_co}
\end{figure}

\paragraph{NOx}

\begin{figure}[ptbh] 
  %a) AIMS insitu\\\includegraphics[align=t,width=0.95\linewidth]{{figures/Exploratory_Data_Analysis/Insitu/eda.year.NOx_Wet Tropics__Open Coastal_niskin_log\res}.pdf}\\
  %b) AIMS FLNTU\\\includegraphics[align=t, width=0.95\linewidth]{{figures/Exploratory_Data_Analysis/FLNTU/eda.year.NOx_Wet Tropics__Open Coastal_flntu_log\res}.pdf}\\
  c) Satellite\\\includegraphics[align=t, width=0.95\linewidth]{{figures/Exploratory_Data_Analysis/Satellite/eda.year.NOx_Cape York__Open Coastal__log\res}.png}}\\
  d) eReefs\\\includegraphics[align=t, width=0.95\linewidth]{{figures/Exploratory_Data_Analysis/eReefs/eda.year.NOx_Cape York__Open Coastal_eReefs_log\res}.png}}\\
  e) eReefs926\\\includegraphics[align=t, width=0.95\linewidth]{{figures/Exploratory_Data_Analysis/eReefs926/eda.year.NOx_Cape York__Open Coastal_eReefs926_log\res}.png}}\\
\caption[Observed NOx data for the Cape York Open Coastal Zone (grouped annually)]{Observed (logarithmic axis with violin plot overlay) NOx data for the
  Cape York Open Coastal
  Zone from a) AIMS insitu, b) AIMS FLNTU, c) Satellite, d) eReefs and e)
eReefs926.  Observations are ordered over time and colored conditional on season as Wet (blue
symbols) and Dry (red symbols).  Blue smoother represents Generalized Additive Mixed Model within a
water year and purple line represents average within the water year.  Horizontal red, black and
green dashed lines denote the twice threshold, threshold and half threshold values respectively.
Red and green background shading indicates the range (10\% shade: x4,/4; 30\% shade: x2,/2)
above and below threshold respectively.}\label{fig:violin_NOx_co}
\end{figure}


\subsubsection{Cape York, Midshelf}
\paragraph{Chlorophyll}

\begin{figure}[ptbh] 
  %a) AIMS insitu\\\includegraphics[align=t,width=0.95\linewidth]{{figures/Exploratory_Data_Analysis/Insitu/eda.year.chl_Wet Tropics__Midshelf_niskin_log\res}.pdf}\\
  %b) AIMS FLNTU\\\includegraphics[align=t, width=0.95\linewidth]{{figures/Exploratory_Data_Analysis/FLNTU/eda.year.chl_Wet Tropics__Midshelf_flntu_log\res}.pdf}\\
  c) Satellite\\\includegraphics[align=t, width=0.95\linewidth]{{figures/Exploratory_Data_Analysis/Satellite/eda.year.chl_Cape York__Midshelf__log\res}.png}}\\
  d) eReefs\\\includegraphics[align=t, width=0.95\linewidth]{{figures/Exploratory_Data_Analysis/eReefs/eda.year.chl_Cape York__Midshelf_eReefs_log\res}.png}}\\
  e) eReefs926\\\includegraphics[align=t, width=0.95\linewidth]{{figures/Exploratory_Data_Analysis/eReefs926/eda.year.chl_Cape York__Midshelf_eReefs926_log\res}.png}}\\
\caption[Observed Chlorophyll-a data for the Cape York Midshelf Zone (grouped annually)]{Observed (logarithmic axis with violin plot overlay) Chlorophyll-a data for the
  Cape York Midshelf
  Zone from a) AIMS insitu, b) AIMS FLNTU, c) Satellite, d) eReefs and e)
eReefs926.  Observations are ordered over time and colored conditional on season as Wet (blue
symbols) and Dry (red symbols).  Blue smoother represents Generalized Additive Mixed Model within a
water year and purple line represents average within the water year.  Horizontal red, black and
green dashed lines denote the twice threshold, threshold and half threshold values respectively.
Red and green background shading indicates the range (10\% shade: x4,/4; 30\% shade: x2,/2)
above and below threshold respectively.}\label{fig:violin_chl_cm}
\end{figure}

\paragraph{Total Suspended Solids}

\begin{figure}[ptbh] 
  %a) AIMS insitu\\\includegraphics[align=t,width=0.95\linewidth]{{figures/Exploratory_Data_Analysis/Insitu/eda.year.nap_Wet Tropics__Midshelf_niskin_log\res}.pdf}\\
  %b) AIMS FLNTU\\\includegraphics[align=t, width=0.95\linewidth]{{figures/Exploratory_Data_Analysis/FLNTU/eda.year.nap_Wet Tropics__Midshelf_flntu_log\res}.pdf}\\
  c) Satellite\\\includegraphics[align=t, width=0.95\linewidth]{{figures/Exploratory_Data_Analysis/Satellite/eda.year.nap_Cape York__Midshelf__log\res}.png}}\\
  d) eReefs\\\includegraphics[align=t, width=0.95\linewidth]{{figures/Exploratory_Data_Analysis/eReefs/eda.year.nap_Cape York__Midshelf_eReefs_log\res}.png}}\\
  e) eReefs926\\\includegraphics[align=t, width=0.95\linewidth]{{figures/Exploratory_Data_Analysis/eReefs926/eda.year.nap_Cape York__Midshelf_eReefs926_log\res}.png}}\\
\caption[Observed TSS data for the Cape York Midshelf Zone (grouped annually)]{Observed (logarithmic axis with violin plot overlay) Total Suspended Solids data for the
  Cape York Midshelf
  Zone from a) AIMS insitu, b) AIMS FLNTU, c) Satellite, d) eReefs and e)
eReefs926.  Observations are ordered over time and colored conditional on season as Wet (blue
symbols) and Dry (red symbols).  Blue smoother represents Generalized Additive Mixed Model within a
water year and purple line represents average within the water year.  Horizontal red, black and
green dashed lines denote the twice threshold, threshold and half threshold values respectively.
Red and green background shading indicates the range (10\% shade: x4,/4; 30\% shade: x2,/2)
above and below threshold respectively.}\label{fig:violin_nap_cm}
\end{figure}

\paragraph{Secchi Depth}

\begin{figure}[ptbh] 
  %a) AIMS insitu\\\includegraphics[align=t,width=0.95\linewidth]{{figures/Exploratory_Data_Analysis/Insitu/eda.year.sd_Wet Tropics__Midshelf_niskin_log\res}.pdf}\\
  %b) AIMS FLNTU\\\includegraphics[align=t, width=0.95\linewidth]{{figures/Exploratory_Data_Analysis/FLNTU/eda.year.sd_Wet Tropics__Midshelf_flntu_log\res}.pdf}\\
  c) Satellite\\\includegraphics[align=t, width=0.95\linewidth]{{figures/Exploratory_Data_Analysis/Satellite/eda.year.sd_Cape York__Midshelf__log\res}.png}}\\
  d) eReefs\\\includegraphics[align=t, width=0.95\linewidth]{{figures/Exploratory_Data_Analysis/eReefs/eda.year.sd_Cape York__Midshelf_eReefs_log\res}.png}}\\
  e) eReefs926\\\includegraphics[align=t, width=0.95\linewidth]{{figures/Exploratory_Data_Analysis/eReefs926/eda.year.sd_Cape York__Midshelf_eReefs926_log\res}.png}}\\
\caption[Observed Secchi depth data for the Cape York Midshelf Zone (grouped annually)]{Observed (logarithmic axis with violin plot overlay) Secchi Depth data for the
  Cape York Midshelf
  Zone from a) AIMS insitu, b) AIMS FLNTU, c) Satellite, d) eReefs and e)
eReefs926.  Observations are ordered over time and colored conditional on season as Wet (blue
symbols) and Dry (red symbols).  Blue smoother represents Generalized Additive Mixed Model within a
water year and purple line represents average within the water year.  Horizontal red, black and
green dashed lines denote the twice threshold, threshold and half threshold values respectively.
Red and green background shading indicates the range (10\% shade: x4,/4; 30\% shade: x2,/2)
above and below threshold respectively.}\label{fig:violin_sd_cm}
\end{figure}

\paragraph{NOx}

\begin{figure}[ptbh] 
  %a) AIMS insitu\\\includegraphics[align=t,width=0.95\linewidth]{{figures/Exploratory_Data_Analysis/Insitu/eda.year.NOx_Wet Tropics__Midshelf_niskin_log\res}.pdf}\\
  %b) AIMS FLNTU\\\includegraphics[align=t, width=0.95\linewidth]{{figures/Exploratory_Data_Analysis/FLNTU/eda.year.NOx_Wet Tropics__Midshelf_flntu_log\res}.pdf}\\
  c) Satellite\\\includegraphics[align=t, width=0.95\linewidth]{{figures/Exploratory_Data_Analysis/Satellite/eda.year.NOx_Cape York__Midshelf__log\res}.png}}\\
  d) eReefs\\\includegraphics[align=t, width=0.95\linewidth]{{figures/Exploratory_Data_Analysis/eReefs/eda.year.NOx_Cape York__Midshelf_eReefs_log\res}.png}}\\
  e) eReefs926\\\includegraphics[align=t, width=0.95\linewidth]{{figures/Exploratory_Data_Analysis/eReefs926/eda.year.NOx_Cape York__Midshelf_eReefs926_log\res}.png}}\\
\caption[Observed NOx data for the Cape York Midshelf Zone (grouped annually)]{Observed (logarithmic axis with violin plot overlay) NOx data for the
  Cape York Midshelf
  Zone from a) AIMS insitu, b) AIMS FLNTU, c) Satellite, d) eReefs and e)
eReefs926.  Observations are ordered over time and colored conditional on season as Wet (blue
symbols) and Dry (red symbols).  Blue smoother represents Generalized Additive Mixed Model within a
water year and purple line represents average within the water year.  Horizontal red, black and
green dashed lines denote the twice threshold, threshold and half threshold values respectively.
Red and green background shading indicates the range (10\% shade: x4,/4; 30\% shade: x2,/2)
above and below threshold respectively.}\label{fig:violin_NOx_cm}
\end{figure}


\subsubsection{Cape York, Offshore}
\paragraph{Chlorophyll}

\begin{figure}[ptbh] 
  %a) AIMS insitu\\\includegraphics[align=t,width=0.95\linewidth]{{figures/Exploratory_Data_Analysis/Insitu/eda.year.chl_Wet Tropics__Offshore_niskin_log\res}.pdf}\\
  %b) AIMS FLNTU\\\includegraphics[align=t, width=0.95\linewidth]{{figures/Exploratory_Data_Analysis/FLNTU/eda.year.chl_Wet Tropics__Offshore_flntu_log\res}.pdf}\\
  c) Satellite\\\includegraphics[align=t, width=0.95\linewidth]{{figures/Exploratory_Data_Analysis/Satellite/eda.year.chl_Cape York__Offshore__log\res}.png}}\\
  d) eReefs\\\includegraphics[align=t, width=0.95\linewidth]{{figures/Exploratory_Data_Analysis/eReefs/eda.year.chl_Cape York__Offshore_eReefs_log\res}.png}}\\
  e) eReefs926\\\includegraphics[align=t, width=0.95\linewidth]{{figures/Exploratory_Data_Analysis/eReefs926/eda.year.chl_Cape York__Offshore_eReefs926_log\res}.png}}\\
\caption[Observed Chlorophyll-a data for the Cape York Offshore Zone (grouped annually)]{Observed (logarithmic axis with violin plot overlay) Chlorophyll-a data for the
  Cape York Offshore
  Zone from a) AIMS insitu, b) AIMS FLNTU, c) Satellite, d) eReefs and e)
eReefs926.  Observations are ordered over time and colored conditional on season as Wet (blue
symbols) and Dry (red symbols).  Blue smoother represents Generalized Additive Mixed Model within a
water year and purple line represents average within the water year.  Horizontal red, black and
green dashed lines denote the twice threshold, threshold and half threshold values respectively.
Red and green background shading indicates the range (10\% shade: x4,/4; 30\% shade: x2,/2)
above and below threshold respectively.}\label{fig:violin_chl_cof}
\end{figure}

\paragraph{Total Suspended Solids}

\begin{figure}[ptbh] 
  %a) AIMS insitu\\\includegraphics[align=t,width=0.95\linewidth]{{figures/Exploratory_Data_Analysis/Insitu/eda.year.nap_Wet Tropics__Offshore_niskin_log\res}.pdf}\\
  %b) AIMS FLNTU\\\includegraphics[align=t, width=0.95\linewidth]{{figures/Exploratory_Data_Analysis/FLNTU/eda.year.nap_Wet Tropics__Offshore_flntu_log\res}.pdf}\\
  c) Satellite\\\includegraphics[align=t, width=0.95\linewidth]{{figures/Exploratory_Data_Analysis/Satellite/eda.year.nap_Cape York__Offshore__log\res}.png}}\\
  d) eReefs\\\includegraphics[align=t, width=0.95\linewidth]{{figures/Exploratory_Data_Analysis/eReefs/eda.year.nap_Cape York__Offshore_eReefs_log\res}.png}}\\
  e) eReefs926\\\includegraphics[align=t, width=0.95\linewidth]{{figures/Exploratory_Data_Analysis/eReefs926/eda.year.nap_Cape York__Offshore_eReefs926_log\res}.png}}\\
\caption[Observed TSS data for the Cape York Offshore Zone (grouped annually)]{Observed (logarithmic axis with violin plot overlay) Total Suspended Solids data for the
  Cape York Offshore
  Zone from a) AIMS insitu, b) AIMS FLNTU, c) Satellite, d) eReefs and e)
eReefs926.  Observations are ordered over time and colored conditional on season as Wet (blue
symbols) and Dry (red symbols).  Blue smoother represents Generalized Additive Mixed Model within a
water year and purple line represents average within the water year.  Horizontal red, black and
green dashed lines denote the twice threshold, threshold and half threshold values respectively.
Red and green background shading indicates the range (10\% shade: x4,/4; 30\% shade: x2,/2)
above and below threshold respectively.}\label{fig:violin_nap_cof}
\end{figure}

\paragraph{Secchi Depth}

\begin{figure}[ptbh] 
  %a) AIMS insitu\\\includegraphics[align=t,width=0.95\linewidth]{{figures/Exploratory_Data_Analysis/Insitu/eda.year.sd_Wet Tropics__Offshore_niskin_log\res}.pdf}\\
  %b) AIMS FLNTU\\\includegraphics[align=t, width=0.95\linewidth]{{figures/Exploratory_Data_Analysis/FLNTU/eda.year.sd_Wet Tropics__Offshore_flntu_log\res}.pdf}\\
  c) Satellite\\\includegraphics[align=t, width=0.95\linewidth]{{figures/Exploratory_Data_Analysis/Satellite/eda.year.sd_Cape York__Offshore__log\res}.png}}\\
  d) eReefs\\\includegraphics[align=t, width=0.95\linewidth]{{figures/Exploratory_Data_Analysis/eReefs/eda.year.sd_Cape York__Offshore_eReefs_log\res}.png}}\\
  e) eReefs926\\\includegraphics[align=t, width=0.95\linewidth]{{figures/Exploratory_Data_Analysis/eReefs926/eda.year.sd_Cape York__Offshore_eReefs926_log\res}.png}}\\
\caption[Observed Secchi depth data for the Cape York Offshore Zone (grouped annually)]{Observed (logarithmic axis with violin plot overlay) Secchi Depth data for the
  Cape York Offshore
  Zone from a) AIMS insitu, b) AIMS FLNTU, c) Satellite, d) eReefs and e)
eReefs926.  Observations are ordered over time and colored conditional on season as Wet (blue
symbols) and Dry (red symbols).  Blue smoother represents Generalized Additive Mixed Model within a
water year and purple line represents average within the water year.  Horizontal red, black and
green dashed lines denote the twice threshold, threshold and half threshold values respectively.
Red and green background shading indicates the range (10\% shade: x4,/4; 30\% shade: x2,/2)
above and below threshold respectively.}\label{fig:violin_sd_cof}
\end{figure}

\paragraph{NOx}

\begin{figure}[ptbh] 
  %a) AIMS insitu\\\includegraphics[align=t,width=0.95\linewidth]{{figures/Exploratory_Data_Analysis/Insitu/eda.year.NOx_Wet Tropics__Offshore_niskin_log\res}.pdf}\\
  %b) AIMS FLNTU\\\includegraphics[align=t, width=0.95\linewidth]{{figures/Exploratory_Data_Analysis/FLNTU/eda.year.NOx_Wet Tropics__Offshore_flntu_log\res}.pdf}\\
  c) Satellite\\\includegraphics[align=t, width=0.95\linewidth]{{figures/Exploratory_Data_Analysis/Satellite/eda.year.NOx_Cape York__Offshore__log\res}.png}}\\
  d) eReefs\\\includegraphics[align=t, width=0.95\linewidth]{{figures/Exploratory_Data_Analysis/eReefs/eda.year.NOx_Cape York__Offshore_eReefs_log\res}.png}}\\
  e) eReefs926\\\includegraphics[align=t, width=0.95\linewidth]{{figures/Exploratory_Data_Analysis/eReefs926/eda.year.NOx_Cape York__Offshore_eReefs926_log\res}.png}}\\
\caption[Observed NOx data for the Cape York Offshore Zone (grouped annually)]{Observed (logarithmic axis with violin plot overlay) NOx data for the
  Cape York Offshore
  Zone from a) AIMS insitu, b) AIMS FLNTU, c) Satellite, d) eReefs and e)
eReefs926.  Observations are ordered over time and colored conditional on season as Wet (blue
symbols) and Dry (red symbols).  Blue smoother represents Generalized Additive Mixed Model within a
water year and purple line represents average within the water year.  Horizontal red, black and
green dashed lines denote the twice threshold, threshold and half threshold values respectively.
Red and green background shading indicates the range (10\% shade: x4,/4; 30\% shade: x2,/2)
above and below threshold respectively.}\label{fig:violin_NOx_cof}
\end{figure}


%% Wet Tropics
\subsubsection{Wet Tropics, Enclosed Coastal}
\paragraph{Chlorophyll}

\begin{figure}[ptbh] 
  a) AIMS insitu\\\includegraphics[align=t,width=0.95\linewidth]{{figures/Exploratory_Data_Analysis/Insitu/eda.year.chl_Wet Tropics__Open Coastal_niskin_log\res}.pdf}\\
  b) AIMS FLNTU\\\includegraphics[align=t, width=0.95\linewidth]{{figures/Exploratory_Data_Analysis/FLNTU/eda.year.chl_Wet Tropics__Open Coastal_flntu_log\res}.pdf}\\
  c) Satellite\\\includegraphics[align=t, width=0.95\linewidth]{{figures/Exploratory_Data_Analysis/Satellite/eda.year.chl_Wet Tropics__Enclosed Coastal__log\res}.png}}\\
  d) eReefs\\\includegraphics[align=t, width=0.95\linewidth]{{figures/Exploratory_Data_Analysis/eReefs/eda.year.chl_Wet Tropics__Enclosed Coastal_eReefs_log\res}.png}}\\
  e) eReefs926\\\includegraphics[align=t, width=0.95\linewidth]{{figures/Exploratory_Data_Analysis/eReefs926/eda.year.chl_Wet Tropics__Enclosed Coastal_eReefs926_log\res}.png}}\\
\caption[Observed Chlorophyll-a data for the Wet Tropics Enclosed Coastal Zone (grouped annually)]{Observed (logarithmic axis with violin plot overlay) Chlorophyll-a data for the
  Wet Tropics Enclosed Coastal
  Zone from a) AIMS insitu, b) AIMS FLNTU, c) Satellite, d) eReefs and e)
eReefs926.  Observations are ordered over time and colored conditional on season as Wet (blue
symbols) and Dry (red symbols).  Blue smoother represents Generalized Additive Mixed Model within a
water year and purple line represents average within the water year.  Horizontal red, black and
green dashed lines denote the twice threshold, threshold and half threshold values respectively.
Red and green background shading indicates the range (10\% shade: x4,/4; 30\% shade: x2,/2)
above and below threshold respectively.}\label{fig:violin_chl_we}
\end{figure}

\paragraph{Total Suspended Solids}

\begin{figure}[ptbh] 
  a) AIMS insitu\\\includegraphics[align=t,width=0.95\linewidth]{{figures/Exploratory_Data_Analysis/Insitu/eda.year.nap_Wet Tropics__Open Coastal_niskin_log\res}.pdf}\\
  b) AIMS FLNTU\\\includegraphics[align=t, width=0.95\linewidth]{{figures/Exploratory_Data_Analysis/FLNTU/eda.year.nap_Wet Tropics__Open Coastal_flntu_log\res}.pdf}\\
  c) Satellite\\\includegraphics[align=t, width=0.95\linewidth]{{figures/Exploratory_Data_Analysis/Satellite/eda.year.nap_Wet Tropics__Enclosed Coastal__log\res}.png}}\\
  d) eReefs\\\includegraphics[align=t, width=0.95\linewidth]{{figures/Exploratory_Data_Analysis/eReefs/eda.year.nap_Wet Tropics__Enclosed Coastal_eReefs_log\res}.png}}\\
  e) eReefs926\\\includegraphics[align=t, width=0.95\linewidth]{{figures/Exploratory_Data_Analysis/eReefs926/eda.year.nap_Wet Tropics__Enclosed Coastal_eReefs926_log\res}.png}}\\
\caption[Observed TSS data for the Wet Tropics Enclosed Coastal Zone (grouped annually)]{Observed (logarithmic axis with violin plot overlay) Total Suspended Solids data for the
  Wet Tropics Enclosed Coastal
  Zone from a) AIMS insitu, b) AIMS FLNTU, c) Satellite, d) eReefs and e)
eReefs926.  Observations are ordered over time and colored conditional on season as Wet (blue
symbols) and Dry (red symbols).  Blue smoother represents Generalized Additive Mixed Model within a
water year and purple line represents average within the water year.  Horizontal red, black and
green dashed lines denote the twice threshold, threshold and half threshold values respectively.
Red and green background shading indicates the range (10\% shade: x4,/4; 30\% shade: x2,/2)
above and below threshold respectively.}\label{fig:violin_nap_we}
\end{figure}

\paragraph{Secchi Depth}

\begin{figure}[ptbh] 
  a) AIMS insitu\\\includegraphics[align=t,width=0.95\linewidth]{{figures/Exploratory_Data_Analysis/Insitu/eda.year.sd_Wet Tropics__Open Coastal_niskin_log\res}.pdf}\\
  b) AIMS FLNTU\\\includegraphics[align=t, width=0.95\linewidth]{{figures/Exploratory_Data_Analysis/FLNTU/eda.year.sd_Wet Tropics__Open Coastal_flntu_log\res}.pdf}\\
  c) Satellite\\\includegraphics[align=t, width=0.95\linewidth]{{figures/Exploratory_Data_Analysis/Satellite/eda.year.sd_Wet Tropics__Enclosed Coastal__log\res}.png}}\\
  d) eReefs\\\includegraphics[align=t, width=0.95\linewidth]{{figures/Exploratory_Data_Analysis/eReefs/eda.year.sd_Wet Tropics__Enclosed Coastal_eReefs_log\res}.png}}\\
  e) eReefs926\\\includegraphics[align=t, width=0.95\linewidth]{{figures/Exploratory_Data_Analysis/eReefs926/eda.year.sd_Wet Tropics__Enclosed Coastal_eReefs926_log\res}.png}}\\
\caption[Observed Secchi depth data for the Wet Tropics Enclosed Coastal Zone (grouped annually)]{Observed (logarithmic axis with violin plot overlay) Secchi Depth data for the
  Wet Tropics Enclosed Coastal
  Zone from a) AIMS insitu, b) AIMS FLNTU, c) Satellite, d) eReefs and e)
eReefs926.  Observations are ordered over time and colored conditional on season as Wet (blue
symbols) and Dry (red symbols).  Blue smoother represents Generalized Additive Mixed Model within a
water year and purple line represents average within the water year.  Horizontal red, black and
green dashed lines denote the twice threshold, threshold and half threshold values respectively.
Red and green background shading indicates the range (10\% shade: x4,/4; 30\% shade: x2,/2)
above and below threshold respectively.}\label{fig:violin_sd_we}
\end{figure}

\paragraph{NOx}

\begin{figure}[ptbh] 
  a) AIMS insitu\\\includegraphics[align=t,width=0.95\linewidth]{{figures/Exploratory_Data_Analysis/Insitu/eda.year.NOx_Wet Tropics__Open Coastal_niskin_log\res}.pdf}\\
  b) AIMS FLNTU\\\includegraphics[align=t, width=0.95\linewidth]{{figures/Exploratory_Data_Analysis/FLNTU/eda.year.NOx_Wet Tropics__Open Coastal_flntu_log\res}.pdf}\\
  c) Satellite\\\includegraphics[align=t, width=0.95\linewidth]{{figures/Exploratory_Data_Analysis/Satellite/eda.year.NOx_Wet Tropics__Enclosed Coastal__log\res}.png}}\\
  d) eReefs\\\includegraphics[align=t, width=0.95\linewidth]{{figures/Exploratory_Data_Analysis/eReefs/eda.year.NOx_Wet Tropics__Enclosed Coastal_eReefs_log\res}.png}}\\
  e) eReefs926\\\includegraphics[align=t, width=0.95\linewidth]{{figures/Exploratory_Data_Analysis/eReefs926/eda.year.NOx_Wet Tropics__Enclosed Coastal_eReefs926_log\res}.png}}\\
\caption[Observed NOx data for the Wet Tropics Enclosed Coastal Zone (grouped annually)]{Observed (logarithmic axis with violin plot overlay) NOx data for the
  Wet Tropics Enclosed Coastal
  Zone from a) AIMS insitu, b) AIMS FLNTU, c) Satellite, d) eReefs and e)
eReefs926.  Observations are ordered over time and colored conditional on season as Wet (blue
symbols) and Dry (red symbols).  Blue smoother represents Generalized Additive Mixed Model within a
water year and purple line represents average within the water year.  Horizontal red, black and
green dashed lines denote the twice threshold, threshold and half threshold values respectively.
Red and green background shading indicates the range (10\% shade: x4,/4; 30\% shade: x2,/2)
above and below threshold respectively.}\label{fig:violin_NOx_we}
\end{figure}


\subsubsection{Wet Tropics, Open Coastal}
\paragraph{Chlorophyll}

\begin{figure}[ptbh] 
  a) AIMS insitu\\\includegraphics[align=t,width=0.95\linewidth]{{figures/Exploratory_Data_Analysis/Insitu/eda.year.chl_Wet Tropics__Open Coastal_niskin_log\res}.pdf}\\
  b) AIMS FLNTU\\\includegraphics[align=t, width=0.95\linewidth]{{figures/Exploratory_Data_Analysis/FLNTU/eda.year.chl_Wet Tropics__Open Coastal_flntu_log\res}.pdf}\\
  c) Satellite\\\includegraphics[align=t, width=0.95\linewidth]{{figures/Exploratory_Data_Analysis/Satellite/eda.year.chl_Wet Tropics__Open Coastal__log\res}.png}}\\
  d) eReefs\\\includegraphics[align=t, width=0.95\linewidth]{{figures/Exploratory_Data_Analysis/eReefs/eda.year.chl_Wet Tropics__Open Coastal_eReefs_log\res}.png}}\\
  e) eReefs926\\\includegraphics[align=t, width=0.95\linewidth]{{figures/Exploratory_Data_Analysis/eReefs926/eda.year.chl_Wet Tropics__Open Coastal_eReefs926_log\res}.png}}\\
\caption[Observed Chlorophyll-a data for the Wet Tropics Open Coastal Zone (grouped annually)]{Observed (logarithmic axis with violin plot overlay) Chlorophyll-a data for the
  Wet Tropics Open Coastal
  Zone from a) AIMS insitu, b) AIMS FLNTU, c) Satellite, d) eReefs and e)
eReefs926.  Observations are ordered over time and colored conditional on season as Wet (blue
symbols) and Dry (red symbols).  Blue smoother represents Generalized Additive Mixed Model within a
water year and purple line represents average within the water year.  Horizontal red, black and
green dashed lines denote the twice threshold, threshold and half threshold values respectively.
Red and green background shading indicates the range (10\% shade: x4,/4; 30\% shade: x2,/2)
above and below threshold respectively.}\label{fig:violin_chl_wo}
\end{figure}

\paragraph{Total Suspended Solids}

\begin{figure}[ptbh] 
  a) AIMS insitu\\\includegraphics[align=t,width=0.95\linewidth]{{figures/Exploratory_Data_Analysis/Insitu/eda.year.nap_Wet Tropics__Open Coastal_niskin_log\res}.pdf}\\
  b) AIMS FLNTU\\\includegraphics[align=t, width=0.95\linewidth]{{figures/Exploratory_Data_Analysis/FLNTU/eda.year.nap_Wet Tropics__Open Coastal_flntu_log\res}.pdf}\\
  c) Satellite\\\includegraphics[align=t, width=0.95\linewidth]{{figures/Exploratory_Data_Analysis/Satellite/eda.year.nap_Wet Tropics__Open Coastal__log\res}.png}}\\
  d) eReefs\\\includegraphics[align=t, width=0.95\linewidth]{{figures/Exploratory_Data_Analysis/eReefs/eda.year.nap_Wet Tropics__Open Coastal_eReefs_log\res}.png}}\\
  e) eReefs926\\\includegraphics[align=t, width=0.95\linewidth]{{figures/Exploratory_Data_Analysis/eReefs926/eda.year.nap_Wet Tropics__Open Coastal_eReefs926_log\res}.png}}\\
\captio[Observed TSS data for the Wet Tropics Open Coastal Zone (grouped annually)]{Observed (logarithmic axis with violin plot overlay) Total Suspended Solids data for the
  Wet Tropics Open Coastal
  Zone from a) AIMS insitu, b) AIMS FLNTU, c) Satellite, d) eReefs and e)
eReefs926.  Observations are ordered over time and colored conditional on season as Wet (blue
symbols) and Dry (red symbols).  Blue smoother represents Generalized Additive Mixed Model within a
water year and purple line represents average within the water year.  Horizontal red, black and
green dashed lines denote the twice threshold, threshold and half threshold values respectively.
Red and green background shading indicates the range (10\% shade: x4,/4; 30\% shade: x2,/2)
above and below threshold respectively.}\label{fig:violin_nap_wo}
\end{figure}

\paragraph{Secchi Depth}

\begin{figure}[ptbh] 
  a) AIMS insitu\\\includegraphics[align=t,width=0.95\linewidth]{{figures/Exploratory_Data_Analysis/Insitu/eda.year.sd_Wet Tropics__Open Coastal_niskin_log\res}.pdf}\\
  b) AIMS FLNTU\\\includegraphics[align=t, width=0.95\linewidth]{{figures/Exploratory_Data_Analysis/FLNTU/eda.year.sd_Wet Tropics__Open Coastal_flntu_log\res}.pdf}\\
  c) Satellite\\\includegraphics[align=t, width=0.95\linewidth]{{figures/Exploratory_Data_Analysis/Satellite/eda.year.sd_Wet Tropics__Open Coastal__log\res}.png}}\\
  d) eReefs\\\includegraphics[align=t, width=0.95\linewidth]{{figures/Exploratory_Data_Analysis/eReefs/eda.year.sd_Wet Tropics__Open Coastal_eReefs_log\res}.png}}\\
  e) eReefs926\\\includegraphics[align=t, width=0.95\linewidth]{{figures/Exploratory_Data_Analysis/eReefs926/eda.year.sd_Wet Tropics__Open Coastal_eReefs926_log\res}.png}}\\
\caption[Observed Secchi depth data for the Wet Tropics Open Coastal Zone (grouped annually)]{Observed (logarithmic axis with violin plot overlay) Secchi Depth data for the
  Wet Tropics Open Coastal
  Zone from a) AIMS insitu, b) AIMS FLNTU, c) Satellite, d) eReefs and e)
eReefs926.  Observations are ordered over time and colored conditional on season as Wet (blue
symbols) and Dry (red symbols).  Blue smoother represents Generalized Additive Mixed Model within a
water year and purple line represents average within the water year.  Horizontal red, black and
green dashed lines denote the twice threshold, threshold and half threshold values respectively.
Red and green background shading indicates the range (10\% shade: x4,/4; 30\% shade: x2,/2)
above and below threshold respectively.}\label{fig:violin_sd_wo}
\end{figure}

\paragraph{NOx}

\begin{figure}[ptbh] 
  a) AIMS insitu\\\includegraphics[align=t,width=0.95\linewidth]{{figures/Exploratory_Data_Analysis/Insitu/eda.year.NOx_Wet Tropics__Open Coastal_niskin_log\res}.pdf}\\
  b) AIMS FLNTU\\\includegraphics[align=t, width=0.95\linewidth]{{figures/Exploratory_Data_Analysis/FLNTU/eda.year.NOx_Wet Tropics__Open Coastal_flntu_log\res}.pdf}\\
  c) Satellite\\\includegraphics[align=t, width=0.95\linewidth]{{figures/Exploratory_Data_Analysis/Satellite/eda.year.NOx_Wet Tropics__Open Coastal__log\res}.png}}\\
  d) eReefs\\\includegraphics[align=t, width=0.95\linewidth]{{figures/Exploratory_Data_Analysis/eReefs/eda.year.NOx_Wet Tropics__Open Coastal_eReefs_log\res}.png}}\\
  e) eReefs926\\\includegraphics[align=t, width=0.95\linewidth]{{figures/Exploratory_Data_Analysis/eReefs926/eda.year.NOx_Wet Tropics__Open Coastal_eReefs926_log\res}.png}}\\
\caption[Observed NOx data for the Wet Tropics Open Coastal Zone (grouped annually)]{Observed (logarithmic axis with violin plot overlay) NOx data for the
  Wet Tropics Open Coastal
  Zone from a) AIMS insitu, b) AIMS FLNTU, c) Satellite, d) eReefs and e)
eReefs926.  Observations are ordered over time and colored conditional on season as Wet (blue
symbols) and Dry (red symbols).  Blue smoother represents Generalized Additive Mixed Model within a
water year and purple line represents average within the water year.  Horizontal red, black and
green dashed lines denote the twice threshold, threshold and half threshold values respectively.
Red and green background shading indicates the range (10\% shade: x4,/4; 30\% shade: x2,/2)
above and below threshold respectively.}\label{fig:violin_NOx_wo}
\end{figure}

\subsubsection{Wet Tropics, Midshelf}
\paragraph{Chlorophyll}

\begin{figure}[ptbh] 
  a) AIMS insitu\\\includegraphics[align=t,width=0.95\linewidth]{{figures/Exploratory_Data_Analysis/Insitu/eda.year.chl_Wet Tropics__Midshelf_niskin_log\res}.pdf}\\
  b) AIMS FLNTU\\\includegraphics[align=t, width=0.95\linewidth]{{figures/Exploratory_Data_Analysis/FLNTU/eda.year.chl_Wet Tropics__Midshelf_flntu_log\res}.pdf}\\
  c) Satellite\\\includegraphics[align=t, width=0.95\linewidth]{{figures/Exploratory_Data_Analysis/Satellite/eda.year.chl_Wet Tropics__Midshelf__log\res}.png}}\\
  d) eReefs\\\includegraphics[align=t, width=0.95\linewidth]{{figures/Exploratory_Data_Analysis/eReefs/eda.year.chl_Wet Tropics__Midshelf_eReefs_log\res}.png}}\\
  e) eReefs926\\\includegraphics[align=t, width=0.95\linewidth]{{figures/Exploratory_Data_Analysis/eReefs926/eda.year.chl_Wet Tropics__Midshelf_eReefs926_log\res}.png}}\\
\caption[Observed Chlorophyll-a data for the Wet Tropics Midshelf Zone (grouped annually)]{Observed (logarithmic axis with violin plot overlay) Chlorophyll-a data for the
  Wet Tropics Midshelf
  Zone from a) AIMS insitu, b) AIMS FLNTU, c) Satellite, d) eReefs and e)
eReefs926.  Observations are ordered over time and colored conditional on season as Wet (blue
symbols) and Dry (red symbols).  Blue smoother represents Generalized Additive Mixed Model within a
water year and purple line represents average within the water year.  Horizontal red, black and
green dashed lines denote the twice threshold, threshold and half threshold values respectively.
Red and green background shading indicates the range (10\% shade: x4,/4; 30\% shade: x2,/2)
above and below threshold respectively.}\label{fig:violin_chl_wm}
\end{figure}

\paragraph{Total Suspended Solids}

\begin{figure}[ptbh] 
  a) AIMS insitu\\\includegraphics[align=t,width=0.95\linewidth]{{figures/Exploratory_Data_Analysis/Insitu/eda.year.nap_Wet Tropics__Midshelf_niskin_log\res}.pdf}\\
  b) AIMS FLNTU\\\includegraphics[align=t, width=0.95\linewidth]{{figures/Exploratory_Data_Analysis/FLNTU/eda.year.nap_Wet Tropics__Midshelf_flntu_log\res}.pdf}\\
  c) Satellite\\\includegraphics[align=t, width=0.95\linewidth]{{figures/Exploratory_Data_Analysis/Satellite/eda.year.nap_Wet Tropics__Midshelf__log\res}.png}}\\
  d) eReefs\\\includegraphics[align=t, width=0.95\linewidth]{{figures/Exploratory_Data_Analysis/eReefs/eda.year.nap_Wet Tropics__Midshelf_eReefs_log\res}.png}}\\
  e) eReefs926\\\includegraphics[align=t, width=0.95\linewidth]{{figures/Exploratory_Data_Analysis/eReefs926/eda.year.nap_Wet Tropics__Midshelf_eReefs926_log\res}.png}}\\
\caption[Observed TSS data for the Wet Tropics Midshelf Zone (grouped annually)]{Observed (logarithmic axis with violin plot overlay) Total Suspended Solids data for the
  Wet Tropics Midshelf
  Zone from a) AIMS insitu, b) AIMS FLNTU, c) Satellite, d) eReefs and e)
eReefs926.  Observations are ordered over time and colored conditional on season as Wet (blue
symbols) and Dry (red symbols).  Blue smoother represents Generalized Additive Mixed Model within a
water year and purple line represents average within the water year.  Horizontal red, black and
green dashed lines denote the twice threshold, threshold and half threshold values respectively.
Red and green background shading indicates the range (10\% shade: x4,/4; 30\% shade: x2,/2)
above and below threshold respectively.}\label{fig:violin_nap_wm}
\end{figure}

\paragraph{Secchi Depth}

\begin{figure}[ptbh] 
  a) AIMS insitu\\\includegraphics[align=t,width=0.95\linewidth]{{figures/Exploratory_Data_Analysis/Insitu/eda.year.sd_Wet Tropics__Midshelf_niskin_log\res}.pdf}\\
  b) AIMS FLNTU\\\includegraphics[align=t, width=0.95\linewidth]{{figures/Exploratory_Data_Analysis/FLNTU/eda.year.sd_Wet Tropics__Midshelf_flntu_log\res}.pdf}\\
  c) Satellite\\\includegraphics[align=t, width=0.95\linewidth]{{figures/Exploratory_Data_Analysis/Satellite/eda.year.sd_Wet Tropics__Midshelf__log\res}.png}}\\
  d) eReefs\\\includegraphics[align=t, width=0.95\linewidth]{{figures/Exploratory_Data_Analysis/eReefs/eda.year.sd_Wet Tropics__Midshelf_eReefs_log\res}.png}}\\
  e) eReefs926\\\includegraphics[align=t, width=0.95\linewidth]{{figures/Exploratory_Data_Analysis/eReefs926/eda.year.sd_Wet Tropics__Midshelf_eReefs926_log\res}.png}}\\
\caption[Observed Secchi depth data for the Wet Tropics Midshelf Zone (grouped annually)]{Observed (logarithmic axis with violin plot overlay) Secchi Depth data for the
  Wet Tropics Midshelf
  Zone from a) AIMS insitu, b) AIMS FLNTU, c) Satellite, d) eReefs and e)
eReefs926.  Observations are ordered over time and colored conditional on season as Wet (blue
symbols) and Dry (red symbols).  Blue smoother represents Generalized Additive Mixed Model within a
water year and purple line represents average within the water year.  Horizontal red, black and
green dashed lines denote the twice threshold, threshold and half threshold values respectively.
Red and green background shading indicates the range (10\% shade: x4,/4; 30\% shade: x2,/2)
above and below threshold respectively.}\label{fig:violin_sd_wm}
\end{figure}

\paragraph{NOx}

\begin{figure}[ptbh] 
  a) AIMS insitu\\\includegraphics[align=t,width=0.95\linewidth]{{figures/Exploratory_Data_Analysis/Insitu/eda.year.NOx_Wet Tropics__Midshelf_niskin_log\res}.pdf}\\
  b) AIMS FLNTU\\\includegraphics[align=t, width=0.95\linewidth]{{figures/Exploratory_Data_Analysis/FLNTU/eda.year.NOx_Wet Tropics__Midshelf_flntu_log\res}.pdf}\\
  c) Satellite\\\includegraphics[align=t, width=0.95\linewidth]{{figures/Exploratory_Data_Analysis/Satellite/eda.year.NOx_Wet Tropics__Midshelf__log\res}.png}}\\
  d) eReefs\\\includegraphics[align=t, width=0.95\linewidth]{{figures/Exploratory_Data_Analysis/eReefs/eda.year.NOx_Wet Tropics__Midshelf_eReefs_log\res}.png}}\\
  e) eReefs926\\\includegraphics[align=t, width=0.95\linewidth]{{figures/Exploratory_Data_Analysis/eReefs926/eda.year.NOx_Wet Tropics__Midshelf_eReefs926_log\res}.png}}\\
\caption[Observed NOx data for the Wet Tropics Midshelf Zone (grouped annually)]{Observed (logarithmic axis with violin plot overlay) NOx data for the
  Wet Tropics Midshelf
  Zone from a) AIMS insitu, b) AIMS FLNTU, c) Satellite, d) eReefs and e)
eReefs926.  Observations are ordered over time and colored conditional on season as Wet (blue
symbols) and Dry (red symbols).  Blue smoother represents Generalized Additive Mixed Model within a
water year and purple line represents average within the water year.  Horizontal red, black and
green dashed lines denote the twice threshold, threshold and half threshold values respectively.
Red and green background shading indicates the range (10\% shade: x4,/4; 30\% shade: x2,/2)
above and below threshold respectively.}\label{fig:violin_NOx_wm}
\end{figure}

\subsubsection{Wet Tropics, Offshore}
\paragraph{Chlorophyll}

\begin{figure}[ptbh] 
  a) AIMS insitu\\\includegraphics[align=t,width=0.95\linewidth]{{figures/Exploratory_Data_Analysis/Insitu/eda.year.chl_Wet Tropics__Offshore_niskin_log\res}.pdf}\\
  b) AIMS FLNTU\\\includegraphics[align=t, width=0.95\linewidth]{{figures/Exploratory_Data_Analysis/FLNTU/eda.year.chl_Wet Tropics__Offshore_flntu_log\res}.pdf}\\
  c) Satellite\\\includegraphics[align=t, width=0.95\linewidth]{{figures/Exploratory_Data_Analysis/Satellite/eda.year.chl_Wet Tropics__Offshore__log\res}.png}}\\
  d) eReefs\\\includegraphics[align=t, width=0.95\linewidth]{{figures/Exploratory_Data_Analysis/eReefs/eda.year.chl_Wet Tropics__Offshore_eReefs_log\res}.png}}\\
  e) eReefs926\\\includegraphics[align=t, width=0.95\linewidth]{{figures/Exploratory_Data_Analysis/eReefs926/eda.year.chl_Wet Tropics__Offshore_eReefs926_log\res}.png}}\\
\caption[Observed Chlorophyll-a data for the Wet Tropics Offshore Zone (grouped annually)]{Observed (logarithmic axis with violin plot overlay) Chlorophyll-a data for the
  Wet Tropics Offshore
  Zone from a) AIMS insitu, b) AIMS FLNTU, c) Satellite, d) eReefs and e)
eReefs926.  Observations are ordered over time and colored conditional on season as Wet (blue
symbols) and Dry (red symbols).  Blue smoother represents Generalized Additive Mixed Model within a
water year and purple line represents average within the water year.  Horizontal red, black and
green dashed lines denote the twice threshold, threshold and half threshold values respectively.
Red and green background shading indicates the range (10\% shade: x4,/4; 30\% shade: x2,/2)
above and below threshold respectively.}\label{fig:violin_chl_wof}
\end{figure}

\paragraph{Total Suspended Solids}

\begin{figure}[ptbh] 
  a) AIMS insitu\\\includegraphics[align=t,width=0.95\linewidth]{{figures/Exploratory_Data_Analysis/Insitu/eda.year.nap_Wet Tropics__Offshore_niskin_log\res}.pdf}\\
  b) AIMS FLNTU\\\includegraphics[align=t, width=0.95\linewidth]{{figures/Exploratory_Data_Analysis/FLNTU/eda.year.nap_Wet Tropics__Offshore_flntu_log\res}.pdf}\\
  c) Satellite\\\includegraphics[align=t, width=0.95\linewidth]{{figures/Exploratory_Data_Analysis/Satellite/eda.year.nap_Wet Tropics__Offshore__log\res}.png}}\\
  d) eReefs\\\includegraphics[align=t, width=0.95\linewidth]{{figures/Exploratory_Data_Analysis/eReefs/eda.year.nap_Wet Tropics__Offshore_eReefs_log\res}.png}}\\
  e) eReefs926\\\includegraphics[align=t, width=0.95\linewidth]{{figures/Exploratory_Data_Analysis/eReefs926/eda.year.nap_Wet Tropics__Offshore_eReefs926_log\res}.png}}\\
\caption[Observed TSS data for the Wet Tropics Offshore Zone (grouped annually)]{Observed (logarithmic axis with violin plot overlay) Total Suspended Solids data for the
  Wet Tropics Offshore
  Zone from a) AIMS insitu, b) AIMS FLNTU, c) Satellite, d) eReefs and e)
eReefs926.  Observations are ordered over time and colored conditional on season as Wet (blue
symbols) and Dry (red symbols).  Blue smoother represents Generalized Additive Mixed Model within a
water year and purple line represents average within the water year.  Horizontal red, black and
green dashed lines denote the twice threshold, threshold and half threshold values respectively.
Red and green background shading indicates the range (10\% shade: x4,/4; 30\% shade: x2,/2)
above and below threshold respectively.}\label{fig:violin_nap_wof}
\end{figure}

\paragraph{Secchi Depth}

\begin{figure}[ptbh] 
  a) AIMS insitu\\\includegraphics[align=t,width=0.95\linewidth]{{figures/Exploratory_Data_Analysis/Insitu/eda.year.sd_Wet Tropics__Offshore_niskin_log\res}.pdf}\\
  b) AIMS FLNTU\\\includegraphics[align=t, width=0.95\linewidth]{{figures/Exploratory_Data_Analysis/FLNTU/eda.year.sd_Wet Tropics__Offshore_flntu_log\res}.pdf}\\
  c) Satellite\\\includegraphics[align=t, width=0.95\linewidth]{{figures/Exploratory_Data_Analysis/Satellite/eda.year.sd_Wet Tropics__Offshore__log\res}.png}}\\
  d) eReefs\\\includegraphics[align=t, width=0.95\linewidth]{{figures/Exploratory_Data_Analysis/eReefs/eda.year.sd_Wet Tropics__Offshore_eReefs_log\res}.png}}\\
  e) eReefs926\\\includegraphics[align=t, width=0.95\linewidth]{{figures/Exploratory_Data_Analysis/eReefs926/eda.year.sd_Wet Tropics__Offshore_eReefs926_log\res}.png}}\\
\caption[Observed Secchi depth data for the Wet Tropics Offshore Zone (grouped annually)]{Observed (logarithmic axis with violin plot overlay) Secchi Depth data for the
  Wet Tropics Offshore
  Zone from a) AIMS insitu, b) AIMS FLNTU, c) Satellite, d) eReefs and e)
eReefs926.  Observations are ordered over time and colored conditional on season as Wet (blue
symbols) and Dry (red symbols).  Blue smoother represents Generalized Additive Mixed Model within a
water year and purple line represents average within the water year.  Horizontal red, black and
green dashed lines denote the twice threshold, threshold and half threshold values respectively.
Red and green background shading indicates the range (10\% shade: x4,/4; 30\% shade: x2,/2)
above and below threshold respectively.}\label{fig:violin_sd_wof}
\end{figure}

\paragraph{NOx}

\begin{figure}[ptbh] 
  a) AIMS insitu\\\includegraphics[align=t,width=0.95\linewidth]{{figures/Exploratory_Data_Analysis/Insitu/eda.year.NOx_Wet Tropics__Offshore_niskin_log\res}.pdf}\\
  b) AIMS FLNTU\\\includegraphics[align=t, width=0.95\linewidth]{{figures/Exploratory_Data_Analysis/FLNTU/eda.year.NOx_Wet Tropics__Offshore_flntu_log\res}.pdf}\\
  c) Satellite\\\includegraphics[align=t, width=0.95\linewidth]{{figures/Exploratory_Data_Analysis/Satellite/eda.year.NOx_Wet Tropics__Offshore__log\res}.png}}\\
  d) eReefs\\\includegraphics[align=t, width=0.95\linewidth]{{figures/Exploratory_Data_Analysis/eReefs/eda.year.NOx_Wet Tropics__Offshore_eReefs_log\res}.png}}\\
  e) eReefs926\\\includegraphics[align=t, width=0.95\linewidth]{{figures/Exploratory_Data_Analysis/eReefs926/eda.year.NOx_Wet Tropics__Offshore_eReefs926_log\res}.png}}\\
\caption[Observed NOx data for the Wet Tropics Offshore Zone (grouped annually)]{Observed (logarithmic axis with violin plot overlay) NOx data for the
  Wet Tropics Offshore
  Zone from a) AIMS insitu, b) AIMS FLNTU, c) Satellite, d) eReefs and e)
eReefs926.  Observations are ordered over time and colored conditional on season as Wet (blue
symbols) and Dry (red symbols).  Blue smoother represents Generalized Additive Mixed Model within a
water year and purple line represents average within the water year.  Horizontal red, black and
green dashed lines denote the twice threshold, threshold and half threshold values respectively.
Red and green background shading indicates the range (10\% shade: x4,/4; 30\% shade: x2,/2)
above and below threshold respectively.}\label{fig:violin_NOx_wof}
\end{figure}


%% Dry Tropics
\subsubsection{Dry Tropics, Enclosed Coastal}
\paragraph{Chlorophyll}

\begin{figure}[ptbh] 
  a) AIMS insitu\\\includegraphics[align=t,width=0.95\linewidth]{{figures/Exploratory_Data_Analysis/Insitu/eda.year.chl_Dry Tropics__Enclosed Coastal_niskin_log\res}.pdf}\\
  b) AIMS FLNTU\\\includegraphics[align=t, width=0.95\linewidth]{{figures/Exploratory_Data_Analysis/FLNTU/eda.year.chl_Dry Tropics__Enclosed Coastal_flntu_log\res}.pdf}\\
  c) Satellite\\\includegraphics[align=t, width=0.95\linewidth]{{figures/Exploratory_Data_Analysis/Satellite/eda.year.chl_Dry Tropics__Enclosed Coastal__log\res}.png}}\\
  d) eReefs\\\includegraphics[align=t, width=0.95\linewidth]{{figures/Exploratory_Data_Analysis/eReefs/eda.year.chl_Dry Tropics__Enclosed Coastal_eReefs_log\res}.png}}\\
  e) eReefs926\\\includegraphics[align=t, width=0.95\linewidth]{{figures/Exploratory_Data_Analysis/eReefs926/eda.year.chl_Dry Tropics__Enclosed Coastal_eReefs926_log\res}.png}}\\
\caption[Observed Chlorophyll-a data for the Dry Tropics Enclosed Coastal Zone (grouped annually)]{Observed (logarithmic axis with violin plot overlay) Chlorophyll-a data for the
  Dry Tropics Enclosed Coastal
  Zone from a) AIMS insitu, b) AIMS FLNTU, c) Satellite, d) eReefs and e)
eReefs926.  Observations are ordered over time and colored conditional on season as Wet (blue
symbols) and Dry (red symbols).  Blue smoother represents Generalized Additive Mixed Model within a
water year and purple line represents average within the water year.  Horizontal red, black and
green dashed lines denote the twice threshold, threshold and half threshold values respectively.
Red and green background shading indicates the range (10\% shade: x4,/4; 30\% shade: x2,/2)
above and below threshold respectively.}\label{fig:violin_chl_de}
\end{figure}

\paragraph{Total Suspended Solids}

\begin{figure}[ptbh] 
  a) AIMS insitu\\\includegraphics[align=t,width=0.95\linewidth]{{figures/Exploratory_Data_Analysis/Insitu/eda.year.nap_Dry Tropics__Enclosed Coastal_niskin_log\res}.pdf}\\
  b) AIMS FLNTU\\\includegraphics[align=t, width=0.95\linewidth]{{figures/Exploratory_Data_Analysis/FLNTU/eda.year.nap_Dry Tropics__Enclosed Coastal_flntu_log\res}.pdf}\\
  c) Satellite\\\includegraphics[align=t, width=0.95\linewidth]{{figures/Exploratory_Data_Analysis/Satellite/eda.year.nap_Dry Tropics__Enclosed Coastal__log\res}.png}}\\
  d) eReefs\\\includegraphics[align=t, width=0.95\linewidth]{{figures/Exploratory_Data_Analysis/eReefs/eda.year.nap_Dry Tropics__Enclosed Coastal_eReefs_log\res}.png}}\\
  e) eReefs926\\\includegraphics[align=t, width=0.95\linewidth]{{figures/Exploratory_Data_Analysis/eReefs926/eda.year.nap_Dry Tropics__Enclosed Coastal_eReefs926_log\res}.png}}\\
\caption[Observed TSS data for the Dry Tropics Enclosed Coastal Zone (grouped annually)]{Observed (logarithmic axis with violin plot overlay) Total Suspended Solids data for the
  Dry Tropics Enclosed Coastal
  Zone from a) AIMS insitu, b) AIMS FLNTU, c) Satellite, d) eReefs and e)
eReefs926.  Observations are ordered over time and colored conditional on season as Wet (blue
symbols) and Dry (red symbols).  Blue smoother represents Generalized Additive Mixed Model within a
water year and purple line represents average within the water year.  Horizontal red, black and
green dashed lines denote the twice threshold, threshold and half threshold values respectively.
Red and green background shading indicates the range (10\% shade: x4,/4; 30\% shade: x2,/2)
above and below threshold respectively.}\label{fig:violin_nap_de}
\end{figure}

\paragraph{Secchi Depth}

\begin{figure}[ptbh] 
  a) AIMS insitu\\\includegraphics[align=t,width=0.95\linewidth]{{figures/Exploratory_Data_Analysis/Insitu/eda.year.sd_Dry Tropics__Enclosed Coastal_niskin_log\res}.pdf}\\
  b) AIMS FLNTU\\\includegraphics[align=t, width=0.95\linewidth]{{figures/Exploratory_Data_Analysis/FLNTU/eda.year.sd_Dry Tropics__Enclosed Coastal_flntu_log\res}.pdf}\\
  c) Satellite\\\includegraphics[align=t, width=0.95\linewidth]{{figures/Exploratory_Data_Analysis/Satellite/eda.year.sd_Dry Tropics__Enclosed Coastal__log\res}.png}}\\
  d) eReefs\\\includegraphics[align=t, width=0.95\linewidth]{{figures/Exploratory_Data_Analysis/eReefs/eda.year.sd_Dry Tropics__Enclosed Coastal_eReefs_log\res}.png}}\\
  e) eReefs926\\\includegraphics[align=t, width=0.95\linewidth]{{figures/Exploratory_Data_Analysis/eReefs926/eda.year.sd_Dry Tropics__Enclosed Coastal_eReefs926_log\res}.png}}\\
\caption[Observed Secchi depth data for the Dry Tropics Enclosed Coastal Zone (grouped annually)]{Observed (logarithmic axis with violin plot overlay) Secchi Depth data for the
  Dry Tropics Enclosed Coastal
  Zone from a) AIMS insitu, b) AIMS FLNTU, c) Satellite, d) eReefs and e)
eReefs926.  Observations are ordered over time and colored conditional on season as Wet (blue
symbols) and Dry (red symbols).  Blue smoother represents Generalized Additive Mixed Model within a
water year and purple line represents average within the water year.  Horizontal red, black and
green dashed lines denote the twice threshold, threshold and half threshold values respectively.
Red and green background shading indicates the range (10\% shade: x4,/4; 30\% shade: x2,/2)
above and below threshold respectively.}\label{fig:violin_sd_de}
\end{figure}

\paragraph{NOx}

\begin{figure}[ptbh] 
  a) AIMS insitu\\\includegraphics[align=t,width=0.95\linewidth]{{figures/Exploratory_Data_Analysis/Insitu/eda.year.NOx_Dry Tropics__Enclosed Coastal_niskin_log\res}.pdf}\\
  b) AIMS FLNTU\\\includegraphics[align=t, width=0.95\linewidth]{{figures/Exploratory_Data_Analysis/FLNTU/eda.year.NOx_Dry Tropics__Enclosed Coastal_flntu_log\res}.pdf}\\
  c) Satellite\\\includegraphics[align=t, width=0.95\linewidth]{{figures/Exploratory_Data_Analysis/Satellite/eda.year.NOx_Dry Tropics__Enclosed Coastal__log\res}.png}}\\
  d) eReefs\\\includegraphics[align=t, width=0.95\linewidth]{{figures/Exploratory_Data_Analysis/eReefs/eda.year.NOx_Dry Tropics__Enclosed Coastal_eReefs_log\res}.png}}\\
  e) eReefs926\\\includegraphics[align=t, width=0.95\linewidth]{{figures/Exploratory_Data_Analysis/eReefs926/eda.year.NOx_Dry Tropics__Enclosed Coastal_eReefs926_log\res}.png}}\\
\caption[Observed NOx data for the Dry Tropics Enclosed Coastal Zone (grouped annually)]{Observed (logarithmic axis with violin plot overlay) NOx data for the
  Dry Tropics Enclosed Coastal
  Zone from a) AIMS insitu, b) AIMS FLNTU, c) Satellite, d) eReefs and e)
eReefs926.  Observations are ordered over time and colored conditional on season as Wet (blue
symbols) and Dry (red symbols).  Blue smoother represents Generalized Additive Mixed Model within a
water year and purple line represents average within the water year.  Horizontal red, black and
green dashed lines denote the twice threshold, threshold and half threshold values respectively.
Red and green background shading indicates the range (10\% shade: x4,/4; 30\% shade: x2,/2)
above and below threshold respectively.}\label{fig:violin_NOx_de}
\end{figure}


\subsubsection{Dry Tropics, Open Coastal}
\paragraph{Chlorophyll}

\begin{figure}[ptbh] 
  a) AIMS insitu\\\includegraphics[align=t,width=0.95\linewidth]{{figures/Exploratory_Data_Analysis/Insitu/eda.year.chl_Dry Tropics__Open Coastal_niskin_log\res}.pdf}\\
  b) AIMS FLNTU\\\includegraphics[align=t, width=0.95\linewidth]{{figures/Exploratory_Data_Analysis/FLNTU/eda.year.chl_Dry Tropics__Open Coastal_flntu_log\res}.pdf}\\
  c) Satellite\\\includegraphics[align=t, width=0.95\linewidth]{{figures/Exploratory_Data_Analysis/Satellite/eda.year.chl_Dry Tropics__Open Coastal__log\res}.png}}\\
  d) eReefs\\\includegraphics[align=t, width=0.95\linewidth]{{figures/Exploratory_Data_Analysis/eReefs/eda.year.chl_Dry Tropics__Open Coastal_eReefs_log\res}.png}}\\
  e) eReefs926\\\includegraphics[align=t, width=0.95\linewidth]{{figures/Exploratory_Data_Analysis/eReefs926/eda.year.chl_Dry Tropics__Open Coastal_eReefs926_log\res}.png}}\\
\caption[Observed Chlorophyll-a data for the Dry Tropics Open Coastal Zone (grouped annually)]{Observed (logarithmic axis with violin plot overlay) Chlorophyll-a data for the
  Dry Tropics Open Coastal
  Zone from a) AIMS insitu, b) AIMS FLNTU, c) Satellite, d) eReefs and e)
eReefs926.  Observations are ordered over time and colored conditional on season as Wet (blue
symbols) and Dry (red symbols).  Blue smoother represents Generalized Additive Mixed Model within a
water year and purple line represents average within the water year.  Horizontal red, black and
green dashed lines denote the twice threshold, threshold and half threshold values respectively.
Red and green background shading indicates the range (10\% shade: x4,/4; 30\% shade: x2,/2)
above and below threshold respectively.}\label{fig:violin_chl_do}
\end{figure}

\paragraph{Total Suspended Solids}

\begin{figure}[ptbh] 
  a) AIMS insitu\\\includegraphics[align=t,width=0.95\linewidth]{{figures/Exploratory_Data_Analysis/Insitu/eda.year.nap_Dry Tropics__Open Coastal_niskin_log\res}.pdf}\\
  b) AIMS FLNTU\\\includegraphics[align=t, width=0.95\linewidth]{{figures/Exploratory_Data_Analysis/FLNTU/eda.year.nap_Dry Tropics__Open Coastal_flntu_log\res}.pdf}\\
  c) Satellite\\\includegraphics[align=t, width=0.95\linewidth]{{figures/Exploratory_Data_Analysis/Satellite/eda.year.nap_Dry Tropics__Open Coastal__log\res}.png}}\\
  d) eReefs\\\includegraphics[align=t, width=0.95\linewidth]{{figures/Exploratory_Data_Analysis/eReefs/eda.year.nap_Dry Tropics__Open Coastal_eReefs_log\res}.png}}\\
  e) eReefs926\\\includegraphics[align=t, width=0.95\linewidth]{{figures/Exploratory_Data_Analysis/eReefs926/eda.year.nap_Dry Tropics__Open Coastal_eReefs926_log\res}.png}}\\
\caption[Observed TSS data for the Dry Tropics Open Coastal Zone (grouped annually)]{Observed (logarithmic axis with violin plot overlay) Total Suspended Solids data for the
  Dry Tropics Open Coastal
  Zone from a) AIMS insitu, b) AIMS FLNTU, c) Satellite, d) eReefs and e)
eReefs926.  Observations are ordered over time and colored conditional on season as Wet (blue
symbols) and Dry (red symbols).  Blue smoother represents Generalized Additive Mixed Model within a
water year and purple line represents average within the water year.  Horizontal red, black and
green dashed lines denote the twice threshold, threshold and half threshold values respectively.
Red and green background shading indicates the range (10\% shade: x4,/4; 30\% shade: x2,/2)
above and below threshold respectively.}\label{fig:violin_nap_do}
\end{figure}

\paragraph{Secchi Depth}

\begin{figure}[ptbh] 
  a) AIMS insitu\\\includegraphics[align=t,width=0.95\linewidth]{{figures/Exploratory_Data_Analysis/Insitu/eda.year.sd_Dry Tropics__Open Coastal_niskin_log\res}.pdf}\\
  b) AIMS FLNTU\\\includegraphics[align=t, width=0.95\linewidth]{{figures/Exploratory_Data_Analysis/FLNTU/eda.year.sd_Dry Tropics__Open Coastal_flntu_log\res}.pdf}\\
  c) Satellite\\\includegraphics[align=t, width=0.95\linewidth]{{figures/Exploratory_Data_Analysis/Satellite/eda.year.sd_Dry Tropics__Open Coastal__log\res}.png}}\\
  d) eReefs\\\includegraphics[align=t, width=0.95\linewidth]{{figures/Exploratory_Data_Analysis/eReefs/eda.year.sd_Dry Tropics__Open Coastal_eReefs_log\res}.png}}\\
  e) eReefs926\\\includegraphics[align=t, width=0.95\linewidth]{{figures/Exploratory_Data_Analysis/eReefs926/eda.year.sd_Dry Tropics__Open Coastal_eReefs926_log\res}.png}}\\
\caption[Observed Secchi depth data for the Dry Tropics Open Coastal Zone (grouped annually)]{Observed (logarithmic axis with violin plot overlay) Secchi Depth data for the
  Dry Tropics Open Coastal
  Zone from a) AIMS insitu, b) AIMS FLNTU, c) Satellite, d) eReefs and e)
eReefs926.  Observations are ordered over time and colored conditional on season as Wet (blue
symbols) and Dry (red symbols).  Blue smoother represents Generalized Additive Mixed Model within a
water year and purple line represents average within the water year.  Horizontal red, black and
green dashed lines denote the twice threshold, threshold and half threshold values respectively.
Red and green background shading indicates the range (10\% shade: x4,/4; 30\% shade: x2,/2)
above and below threshold respectively.}\label{fig:violin_sd_do}
\end{figure}

\paragraph{NOx}

\begin{figure}[ptbh] 
  a) AIMS insitu\\\includegraphics[align=t,width=0.95\linewidth]{{figures/Exploratory_Data_Analysis/Insitu/eda.year.NOx_Dry Tropics__Open Coastal_niskin_log\res}.pdf}\\
  b) AIMS FLNTU\\\includegraphics[align=t, width=0.95\linewidth]{{figures/Exploratory_Data_Analysis/FLNTU/eda.year.NOx_Dry Tropics__Open Coastal_flntu_log\res}.pdf}\\
  c) Satellite\\\includegraphics[align=t, width=0.95\linewidth]{{figures/Exploratory_Data_Analysis/Satellite/eda.year.NOx_Dry Tropics__Open Coastal__log\res}.png}}\\
  d) eReefs\\\includegraphics[align=t, width=0.95\linewidth]{{figures/Exploratory_Data_Analysis/eReefs/eda.year.NOx_Dry Tropics__Open Coastal_eReefs_log\res}.png}}\\
  e) eReefs926\\\includegraphics[align=t, width=0.95\linewidth]{{figures/Exploratory_Data_Analysis/eReefs926/eda.year.NOx_Dry Tropics__Open Coastal_eReefs926_log\res}.png}}\\
\caption[Observed NOx data for the Dry Tropics Open Coastal Zone (grouped annually)]{Observed (logarithmic axis with violin plot overlay) NOx data for the
  Dry Tropics Open Coastal
  Zone from a) AIMS insitu, b) AIMS FLNTU, c) Satellite, d) eReefs and e)
eReefs926.  Observations are ordered over time and colored conditional on season as Wet (blue
symbols) and Dry (red symbols).  Blue smoother represents Generalized Additive Mixed Model within a
water year and purple line represents average within the water year.  Horizontal red, black and
green dashed lines denote the twice threshold, threshold and half threshold values respectively.
Red and green background shading indicates the range (10\% shade: x4,/4; 30\% shade: x2,/2)
above and below threshold respectively.}\label{fig:violin_NOx_do}
\end{figure}

\subsubsection{Dry Tropics, Midshelf}
\paragraph{Chlorophyll}

\begin{figure}[ptbh] 
  a) AIMS insitu\\\includegraphics[align=t,width=0.95\linewidth]{{figures/Exploratory_Data_Analysis/Insitu/eda.year.chl_Dry Tropics__Midshelf_niskin_log\res}.pdf}\\
  b) AIMS FLNTU\\\includegraphics[align=t, width=0.95\linewidth]{{figures/Exploratory_Data_Analysis/FLNTU/eda.year.chl_Dry Tropics__Midshelf_flntu_log\res}.pdf}\\
  c) Satellite\\\includegraphics[align=t, width=0.95\linewidth]{{figures/Exploratory_Data_Analysis/Satellite/eda.year.chl_Dry Tropics__Midshelf__log\res}.png}}\\
  d) eReefs\\\includegraphics[align=t, width=0.95\linewidth]{{figures/Exploratory_Data_Analysis/eReefs/eda.year.chl_Dry Tropics__Midshelf_eReefs_log\res}.png}}\\
  e) eReefs926\\\includegraphics[align=t, width=0.95\linewidth]{{figures/Exploratory_Data_Analysis/eReefs926/eda.year.chl_Dry Tropics__Midshelf_eReefs926_log\res}.png}}\\
\caption[Observed Chlorophyll-a data for the Dry Tropics Midshelf Zone (grouped annually)]{Observed (logarithmic axis with violin plot overlay) Chlorophyll-a data for the
  Dry Tropics Midshelf
  Zone from a) AIMS insitu, b) AIMS FLNTU, c) Satellite, d) eReefs and e)
eReefs926.  Observations are ordered over time and colored conditional on season as Wet (blue
symbols) and Dry (red symbols).  Blue smoother represents Generalized Additive Mixed Model within a
water year and purple line represents average within the water year.  Horizontal red, black and
green dashed lines denote the twice threshold, threshold and half threshold values respectively.
Red and green background shading indicates the range (10\% shade: x4,/4; 30\% shade: x2,/2)
above and below threshold respectively.}\label{fig:violin_chl_dm}
\end{figure}

\paragraph{Total Suspended Solids}

\begin{figure}[ptbh] 
  a) AIMS insitu\\\includegraphics[align=t,width=0.95\linewidth]{{figures/Exploratory_Data_Analysis/Insitu/eda.year.nap_Dry Tropics__Midshelf_niskin_log\res}.pdf}\\
  b) AIMS FLNTU\\\includegraphics[align=t, width=0.95\linewidth]{{figures/Exploratory_Data_Analysis/FLNTU/eda.year.nap_Dry Tropics__Midshelf_flntu_log\res}.pdf}\\
  c) Satellite\\\includegraphics[align=t, width=0.95\linewidth]{{figures/Exploratory_Data_Analysis/Satellite/eda.year.nap_Dry Tropics__Midshelf__log\res}.png}}\\
  d) eReefs\\\includegraphics[align=t, width=0.95\linewidth]{{figures/Exploratory_Data_Analysis/eReefs/eda.year.nap_Dry Tropics__Midshelf_eReefs_log\res}.png}}\\
  e) eReefs926\\\includegraphics[align=t, width=0.95\linewidth]{{figures/Exploratory_Data_Analysis/eReefs926/eda.year.nap_Dry Tropics__Midshelf_eReefs926_log\res}.png}}\\
\caption[Observed TSS data for the Dry Tropics Midshelf Zone (grouped annually)]{Observed (logarithmic axis with violin plot overlay) Total Suspended Solids data for the
  Dry Tropics Midshelf
  Zone from a) AIMS insitu, b) AIMS FLNTU, c) Satellite, d) eReefs and e)
eReefs926.  Observations are ordered over time and colored conditional on season as Wet (blue
symbols) and Dry (red symbols).  Blue smoother represents Generalized Additive Mixed Model within a
water year and purple line represents average within the water year.  Horizontal red, black and
green dashed lines denote the twice threshold, threshold and half threshold values respectively.
Red and green background shading indicates the range (10\% shade: x4,/4; 30\% shade: x2,/2)
above and below threshold respectively.}\label{fig:violin_nap_dm}
\end{figure}

\paragraph{Secchi Depth}

\begin{figure}[ptbh] 
  a) AIMS insitu\\\includegraphics[align=t,width=0.95\linewidth]{{figures/Exploratory_Data_Analysis/Insitu/eda.year.sd_Dry Tropics__Midshelf_niskin_log\res}.pdf}\\
  b) AIMS FLNTU\\\includegraphics[align=t, width=0.95\linewidth]{{figures/Exploratory_Data_Analysis/FLNTU/eda.year.sd_Dry Tropics__Midshelf_flntu_log\res}.pdf}\\
  c) Satellite\\\includegraphics[align=t, width=0.95\linewidth]{{figures/Exploratory_Data_Analysis/Satellite/eda.year.sd_Dry Tropics__Midshelf__log\res}.png}}\\
  d) eReefs\\\includegraphics[align=t, width=0.95\linewidth]{{figures/Exploratory_Data_Analysis/eReefs/eda.year.sd_Dry Tropics__Midshelf_eReefs_log\res}.png}}\\
  e) eReefs926\\\includegraphics[align=t, width=0.95\linewidth]{{figures/Exploratory_Data_Analysis/eReefs926/eda.year.sd_Dry Tropics__Midshelf_eReefs926_log\res}.png}}\\
\caption[Observed Secchi depth data for the Dry Tropics Midshelf Zone (grouped annually)]{Observed (logarithmic axis with violin plot overlay) Secchi Depth data for the
  Dry Tropics Midshelf
  Zone from a) AIMS insitu, b) AIMS FLNTU, c) Satellite, d) eReefs and e)
eReefs926.  Observations are ordered over time and colored conditional on season as Wet (blue
symbols) and Dry (red symbols).  Blue smoother represents Generalized Additive Mixed Model within a
water year and purple line represents average within the water year.  Horizontal red, black and
green dashed lines denote the twice threshold, threshold and half threshold values respectively.
Red and green background shading indicates the range (10\% shade: x4,/4; 30\% shade: x2,/2)
above and below threshold respectively.}\label{fig:violin_sd_dm}
\end{figure}

\paragraph{NOx}

\begin{figure}[ptbh] 
  a) AIMS insitu\\\includegraphics[align=t,width=0.95\linewidth]{{figures/Exploratory_Data_Analysis/Insitu/eda.year.NOx_Dry Tropics__Midshelf_niskin_log\res}.pdf}\\
  b) AIMS FLNTU\\\includegraphics[align=t, width=0.95\linewidth]{{figures/Exploratory_Data_Analysis/FLNTU/eda.year.NOx_Dry Tropics__Midshelf_flntu_log\res}.pdf}\\
  c) Satellite\\\includegraphics[align=t, width=0.95\linewidth]{{figures/Exploratory_Data_Analysis/Satellite/eda.year.NOx_Dry Tropics__Midshelf__log\res}.png}}\\
  d) eReefs\\\includegraphics[align=t, width=0.95\linewidth]{{figures/Exploratory_Data_Analysis/eReefs/eda.year.NOx_Dry Tropics__Midshelf_eReefs_log\res}.png}}\\
  e) eReefs926\\\includegraphics[align=t, width=0.95\linewidth]{{figures/Exploratory_Data_Analysis/eReefs926/eda.year.NOx_Dry Tropics__Midshelf_eReefs926_log\res}.png}}\\
\caption[Observed NOx data for the Dry Tropics Midshelf Zone (grouped annually)]{Observed (logarithmic axis with violin plot overlay) NOx data for the
  Dry Tropics Midshelf
  Zone from a) AIMS insitu, b) AIMS FLNTU, c) Satellite, d) eReefs and e)
eReefs926.  Observations are ordered over time and colored conditional on season as Wet (blue
symbols) and Dry (red symbols).  Blue smoother represents Generalized Additive Mixed Model within a
water year and purple line represents average within the water year.  Horizontal red, black and
green dashed lines denote the twice threshold, threshold and half threshold values respectively.
Red and green background shading indicates the range (10\% shade: x4,/4; 30\% shade: x2,/2)
above and below threshold respectively.}\label{fig:violin_NOx_dm}
\end{figure}

\subsubsection{Dry Tropics, Offshore}
\paragraph{Chlorophyll}

\begin{figure}[ptbh] 
  a) AIMS insitu\\\includegraphics[align=t,width=0.95\linewidth]{{figures/Exploratory_Data_Analysis/Insitu/eda.year.chl_Dry Tropics__Offshore_niskin_log\res}.pdf}\\
  b) AIMS FLNTU\\\includegraphics[align=t, width=0.95\linewidth]{{figures/Exploratory_Data_Analysis/FLNTU/eda.year.chl_Dry Tropics__Offshore_flntu_log\res}.pdf}\\
  c) Satellite\\\includegraphics[align=t, width=0.95\linewidth]{{figures/Exploratory_Data_Analysis/Satellite/eda.year.chl_Dry Tropics__Offshore__log\res}.png}}\\
  d) eReefs\\\includegraphics[align=t, width=0.95\linewidth]{{figures/Exploratory_Data_Analysis/eReefs/eda.year.chl_Dry Tropics__Offshore_eReefs_log\res}.png}}\\
  e) eReefs926\\\includegraphics[align=t, width=0.95\linewidth]{{figures/Exploratory_Data_Analysis/eReefs926/eda.year.chl_Dry Tropics__Offshore_eReefs926_log\res}.png}}\\
\caption[Observed Chlorophyll-a data for the Dry Tropics Offshore Zone (grouped annually)]{Observed (logarithmic axis with violin plot overlay) Chlorophyll-a data for the
  Dry Tropics Offshore
  Zone from a) AIMS insitu, b) AIMS FLNTU, c) Satellite, d) eReefs and e)
eReefs926.  Observations are ordered over time and colored conditional on season as Wet (blue
symbols) and Dry (red symbols).  Blue smoother represents Generalized Additive Mixed Model within a
water year and purple line represents average within the water year.  Horizontal red, black and
green dashed lines denote the twice threshold, threshold and half threshold values respectively.
Red and green background shading indicates the range (10\% shade: x4,/4; 30\% shade: x2,/2)
above and below threshold respectively.}\label{fig:violin_chl_dof}
\end{figure}

\paragraph{Total Suspended Solids}

\begin{figure}[ptbh] 
  a) AIMS insitu\\\includegraphics[align=t,width=0.95\linewidth]{{figures/Exploratory_Data_Analysis/Insitu/eda.year.nap_Dry Tropics__Offshore_niskin_log\res}.pdf}\\
  b) AIMS FLNTU\\\includegraphics[align=t, width=0.95\linewidth]{{figures/Exploratory_Data_Analysis/FLNTU/eda.year.nap_Dry Tropics__Offshore_flntu_log\res}.pdf}\\
  c) Satellite\\\includegraphics[align=t, width=0.95\linewidth]{{figures/Exploratory_Data_Analysis/Satellite/eda.year.nap_Dry Tropics__Offshore__log\res}.png}}\\
  d) eReefs\\\includegraphics[align=t, width=0.95\linewidth]{{figures/Exploratory_Data_Analysis/eReefs/eda.year.nap_Dry Tropics__Offshore_eReefs_log\res}.png}}\\
  e) eReefs926\\\includegraphics[align=t, width=0.95\linewidth]{{figures/Exploratory_Data_Analysis/eReefs926/eda.year.nap_Dry Tropics__Offshore_eReefs926_log\res}.png}}\\
\caption[Observed TSS data for the Dry Tropics Offshore Zone (grouped annually)]{Observed (logarithmic axis with violin plot overlay) Total Suspended Solids data for the
  Dry Tropics Offshore
  Zone from a) AIMS insitu, b) AIMS FLNTU, c) Satellite, d) eReefs and e)
eReefs926.  Observations are ordered over time and colored conditional on season as Wet (blue
symbols) and Dry (red symbols).  Blue smoother represents Generalized Additive Mixed Model within a
water year and purple line represents average within the water year.  Horizontal red, black and
green dashed lines denote the twice threshold, threshold and half threshold values respectively.
Red and green background shading indicates the range (10\% shade: x4,/4; 30\% shade: x2,/2)
above and below threshold respectively.}\label{fig:violin_nap_dof}
\end{figure}

\paragraph{Secchi Depth}

\begin{figure}[ptbh] 
  a) AIMS insitu\\\includegraphics[align=t,width=0.95\linewidth]{{figures/Exploratory_Data_Analysis/Insitu/eda.year.sd_Dry Tropics__Offshore_niskin_log\res}.pdf}\\
  b) AIMS FLNTU\\\includegraphics[align=t, width=0.95\linewidth]{{figures/Exploratory_Data_Analysis/FLNTU/eda.year.sd_Dry Tropics__Offshore_flntu_log\res}.pdf}\\
  c) Satellite\\\includegraphics[align=t, width=0.95\linewidth]{{figures/Exploratory_Data_Analysis/Satellite/eda.year.sd_Dry Tropics__Offshore__log\res}.png}}\\
  d) eReefs\\\includegraphics[align=t, width=0.95\linewidth]{{figures/Exploratory_Data_Analysis/eReefs/eda.year.sd_Dry Tropics__Offshore_eReefs_log\res}.png}}\\
  e) eReefs926\\\includegraphics[align=t, width=0.95\linewidth]{{figures/Exploratory_Data_Analysis/eReefs926/eda.year.sd_Dry Tropics__Offshore_eReefs926_log\res}.png}}\\
\caption[Observed Secchi depth data for the Dry Tropics Offshore Zone (grouped annually)]{Observed (logarithmic axis with violin plot overlay) Secchi Depth data for the
  Dry Tropics Offshore
  Zone from a) AIMS insitu, b) AIMS FLNTU, c) Satellite, d) eReefs and e)
eReefs926.  Observations are ordered over time and colored conditional on season as Wet (blue
symbols) and Dry (red symbols).  Blue smoother represents Generalized Additive Mixed Model within a
water year and purple line represents average within the water year.  Horizontal red, black and
green dashed lines denote the twice threshold, threshold and half threshold values respectively.
Red and green background shading indicates the range (10\% shade: x4,/4; 30\% shade: x2,/2)
above and below threshold respectively.}\label{fig:violin_sd_dof}
\end{figure}

\paragraph{NOx}

\begin{figure}[ptbh] 
  a) AIMS insitu\\\includegraphics[align=t,width=0.95\linewidth]{{figures/Exploratory_Data_Analysis/Insitu/eda.year.NOx_Dry Tropics__Offshore_niskin_log\res}.pdf}\\
  b) AIMS FLNTU\\\includegraphics[align=t, width=0.95\linewidth]{{figures/Exploratory_Data_Analysis/FLNTU/eda.year.NOx_Dry Tropics__Offshore_flntu_log\res}.pdf}\\
  c) Satellite\\\includegraphics[align=t, width=0.95\linewidth]{{figures/Exploratory_Data_Analysis/Satellite/eda.year.NOx_Dry Tropics__Offshore__log\res}.png}}\\
  d) eReefs\\\includegraphics[align=t, width=0.95\linewidth]{{figures/Exploratory_Data_Analysis/eReefs/eda.year.NOx_Dry Tropics__Offshore_eReefs_log\res}.png}}\\
  e) eReefs926\\\includegraphics[align=t, width=0.95\linewidth]{{figures/Exploratory_Data_Analysis/eReefs926/eda.year.NOx_Dry Tropics__Offshore_eReefs926_log\res}.png}}\\
\caption[Observed NOx data for the Dry Tropics Offshore Zone (grouped annually)]{Observed (logarithmic axis with violin plot overlay) NOx data for the
  Dry Tropics Offshore
  Zone from a) AIMS insitu, b) AIMS FLNTU, c) Satellite, d) eReefs and e)
eReefs926.  Observations are ordered over time and colored conditional on season as Wet (blue
symbols) and Dry (red symbols).  Blue smoother represents Generalized Additive Mixed Model within a
water year and purple line represents average within the water year.  Horizontal red, black and
green dashed lines denote the twice threshold, threshold and half threshold values respectively.
Red and green background shading indicates the range (10\% shade: x4,/4; 30\% shade: x2,/2)
above and below threshold respectively.}\label{fig:violin_NOx_dof}
\end{figure}


%% Mackay Whitsunday
\subsubsection{Mackay Whitsunday, Enclosed Coastal}
\paragraph{Chlorophyll}

\begin{figure}[ptbh] 
  a) AIMS insitu\\\includegraphics[align=t,width=0.95\linewidth]{{figures/Exploratory_Data_Analysis/Insitu/eda.year.chl_Mackay Whitsunday__Enclosed Coastal_niskin_log\res}.pdf}\\
  b) AIMS FLNTU\\\includegraphics[align=t, width=0.95\linewidth]{{figures/Exploratory_Data_Analysis/FLNTU/eda.year.chl_Mackay Whitsunday__Enclosed Coastal_flntu_log\res}.pdf}\\
  c) Satellite\\\includegraphics[align=t, width=0.95\linewidth]{{figures/Exploratory_Data_Analysis/Satellite/eda.year.chl_Mackay Whitsunday__Enclosed Coastal__log\res}.png}}\\
  d) eReefs\\\includegraphics[align=t, width=0.95\linewidth]{{figures/Exploratory_Data_Analysis/eReefs/eda.year.chl_Mackay Whitsunday__Enclosed Coastal_eReefs_log\res}.png}}\\
  e) eReefs926\\\includegraphics[align=t, width=0.95\linewidth]{{figures/Exploratory_Data_Analysis/eReefs926/eda.year.chl_Mackay Whitsunday__Enclosed Coastal_eReefs926_log\res}.png}}\\
\caption[Observed Chlorophyll-a data for the Mackay Whitsunday Enclosed Coastal Zone (grouped annually)]{Observed (logarithmic axis with violin plot overlay) Chlorophyll-a data for the
  Mackay Whitsunday Enclosed Coastal
  Zone from a) AIMS insitu, b) AIMS FLNTU, c) Satellite, d) eReefs and e)
eReefs926.  Observations are ordered over time and colored conditional on season as Wet (blue
symbols) and Dry (red symbols).  Blue smoother represents Generalized Additive Mixed Model within a
water year and purple line represents average within the water year.  Horizontal red, black and
green dashed lines denote the twice threshold, threshold and half threshold values respectively.
Red and green background shading indicates the range (10\% shade: x4,/4; 30\% shade: x2,/2)
above and below threshold respectively.}\label{fig:violin_chl_me}
\end{figure}

\paragraph{Total Suspended Solids}

\begin{figure}[ptbh] 
  a) AIMS insitu\\\includegraphics[align=t,width=0.95\linewidth]{{figures/Exploratory_Data_Analysis/Insitu/eda.year.nap_Mackay Whitsunday__Enclosed Coastal_niskin_log\res}.pdf}\\
  b) AIMS FLNTU\\\includegraphics[align=t, width=0.95\linewidth]{{figures/Exploratory_Data_Analysis/FLNTU/eda.year.nap_Mackay Whitsunday__Enclosed Coastal_flntu_log\res}.pdf}\\
  c) Satellite\\\includegraphics[align=t, width=0.95\linewidth]{{figures/Exploratory_Data_Analysis/Satellite/eda.year.nap_Mackay Whitsunday__Enclosed Coastal__log\res}.png}}\\
  d) eReefs\\\includegraphics[align=t, width=0.95\linewidth]{{figures/Exploratory_Data_Analysis/eReefs/eda.year.nap_Mackay Whitsunday__Enclosed Coastal_eReefs_log\res}.png}}\\
  e) eReefs926\\\includegraphics[align=t, width=0.95\linewidth]{{figures/Exploratory_Data_Analysis/eReefs926/eda.year.nap_Mackay Whitsunday__Enclosed Coastal_eReefs926_log\res}.png}}\\
\caption[Observed TSS data for the Mackay Whitsunday Enclosed Coastal Zone (grouped annually)]{Observed (logarithmic axis with violin plot overlay) Total Suspended Solids data for the
  Mackay Whitsunday Enclosed Coastal
  Zone from a) AIMS insitu, b) AIMS FLNTU, c) Satellite, d) eReefs and e)
eReefs926.  Observations are ordered over time and colored conditional on season as Wet (blue
symbols) and Dry (red symbols).  Blue smoother represents Generalized Additive Mixed Model within a
water year and purple line represents average within the water year.  Horizontal red, black and
green dashed lines denote the twice threshold, threshold and half threshold values respectively.
Red and green background shading indicates the range (10\% shade: x4,/4; 30\% shade: x2,/2)
above and below threshold respectively.}\label{fig:violin_nap_me}
\end{figure}

\paragraph{Secchi Depth}

\begin{figure}[ptbh] 
  a) AIMS insitu\\\includegraphics[align=t,width=0.95\linewidth]{{figures/Exploratory_Data_Analysis/Insitu/eda.year.sd_Mackay Whitsunday__Enclosed Coastal_niskin_log\res}.pdf}\\
  b) AIMS FLNTU\\\includegraphics[align=t, width=0.95\linewidth]{{figures/Exploratory_Data_Analysis/FLNTU/eda.year.sd_Mackay Whitsunday__Enclosed Coastal_flntu_log\res}.pdf}\\
  c) Satellite\\\includegraphics[align=t, width=0.95\linewidth]{{figures/Exploratory_Data_Analysis/Satellite/eda.year.sd_Mackay Whitsunday__Enclosed Coastal__log\res}.png}}\\
  d) eReefs\\\includegraphics[align=t, width=0.95\linewidth]{{figures/Exploratory_Data_Analysis/eReefs/eda.year.sd_Mackay Whitsunday__Enclosed Coastal_eReefs_log\res}.png}}\\
  e) eReefs926\\\includegraphics[align=t, width=0.95\linewidth]{{figures/Exploratory_Data_Analysis/eReefs926/eda.year.sd_Mackay Whitsunday__Enclosed Coastal_eReefs926_log\res}.png}}\\
\caption[Observed Secchi depth data for the Mackay Whitsunday Enclosed Coastal Zone (grouped annually)]{Observed (logarithmic axis with violin plot overlay) Secchi Depth data for the
  Mackay Whitsunday Enclosed Coastal
  Zone from a) AIMS insitu, b) AIMS FLNTU, c) Satellite, d) eReefs and e)
eReefs926.  Observations are ordered over time and colored conditional on season as Wet (blue
symbols) and Dry (red symbols).  Blue smoother represents Generalized Additive Mixed Model within a
water year and purple line represents average within the water year.  Horizontal red, black and
green dashed lines denote the twice threshold, threshold and half threshold values respectively.
Red and green background shading indicates the range (10\% shade: x4,/4; 30\% shade: x2,/2)
above and below threshold respectively.}\label{fig:violin_sd_me}
\end{figure}

\paragraph{NOx}

\begin{figure}[ptbh] 
  a) AIMS insitu\\\includegraphics[align=t,width=0.95\linewidth]{{figures/Exploratory_Data_Analysis/Insitu/eda.year.NOx_Mackay Whitsunday__Enclosed Coastal_niskin_log\res}.pdf}\\
  b) AIMS FLNTU\\\includegraphics[align=t, width=0.95\linewidth]{{figures/Exploratory_Data_Analysis/FLNTU/eda.year.NOx_Mackay Whitsunday__Enclosed Coastal_flntu_log\res}.pdf}\\
  c) Satellite\\\includegraphics[align=t, width=0.95\linewidth]{{figures/Exploratory_Data_Analysis/Satellite/eda.year.NOx_Mackay Whitsunday__Enclosed Coastal__log\res}.png}}\\
  d) eReefs\\\includegraphics[align=t, width=0.95\linewidth]{{figures/Exploratory_Data_Analysis/eReefs/eda.year.NOx_Mackay Whitsunday__Enclosed Coastal_eReefs_log\res}.png}}\\
  e) eReefs926\\\includegraphics[align=t, width=0.95\linewidth]{{figures/Exploratory_Data_Analysis/eReefs926/eda.year.NOx_Mackay Whitsunday__Enclosed Coastal_eReefs926_log\res}.png}}\\
\caption[Observed NOx data for the Mackay Whitsunday Enclosed Coastal Zone (grouped annually)]{Observed (logarithmic axis with violin plot overlay) NOx data for the
  Mackay Whitsunday Enclosed Coastal
  Zone from a) AIMS insitu, b) AIMS FLNTU, c) Satellite, d) eReefs and e)
eReefs926.  Observations are ordered over time and colored conditional on season as Wet (blue
symbols) and Dry (red symbols).  Blue smoother represents Generalized Additive Mixed Model within a
water year and purple line represents average within the water year.  Horizontal red, black and
green dashed lines denote the twice threshold, threshold and half threshold values respectively.
Red and green background shading indicates the range (10\% shade: x4,/4; 30\% shade: x2,/2)
above and below threshold respectively.}\label{fig:violin_NOx_me}
\end{figure}


\subsubsection{Mackay Whitsunday, Open Coastal}
\paragraph{Chlorophyll}

\begin{figure}[ptbh] 
  a) AIMS insitu\\\includegraphics[align=t,width=0.95\linewidth]{{figures/Exploratory_Data_Analysis/Insitu/eda.year.chl_Mackay Whitsunday__Open Coastal_niskin_log\res}.pdf}\\
  b) AIMS FLNTU\\\includegraphics[align=t, width=0.95\linewidth]{{figures/Exploratory_Data_Analysis/FLNTU/eda.year.chl_Mackay Whitsunday__Open Coastal_flntu_log\res}.pdf}\\
  c) Satellite\\\includegraphics[align=t, width=0.95\linewidth]{{figures/Exploratory_Data_Analysis/Satellite/eda.year.chl_Mackay Whitsunday__Open Coastal__log\res}.png}}\\
  d) eReefs\\\includegraphics[align=t, width=0.95\linewidth]{{figures/Exploratory_Data_Analysis/eReefs/eda.year.chl_Mackay Whitsunday__Open Coastal_eReefs_log\res}.png}}\\
  e) eReefs926\\\includegraphics[align=t, width=0.95\linewidth]{{figures/Exploratory_Data_Analysis/eReefs926/eda.year.chl_Mackay Whitsunday__Open Coastal_eReefs926_log\res}.png}}\\
\caption[Observed Chlorophyll-a data for the Mackay Whitsunday Open Coastal Zone (grouped annually)]{Observed (logarithmic axis with violin plot overlay) Chlorophyll-a data for the
  Mackay Whitsunday Open Coastal
  Zone from a) AIMS insitu, b) AIMS FLNTU, c) Satellite, d) eReefs and e)
eReefs926.  Observations are ordered over time and colored conditional on season as Wet (blue
symbols) and Dry (red symbols).  Blue smoother represents Generalized Additive Mixed Model within a
water year and purple line represents average within the water year.  Horizontal red, black and
green dashed lines denote the twice threshold, threshold and half threshold values respectively.
Red and green background shading indicates the range (10\% shade: x4,/4; 30\% shade: x2,/2)
above and below threshold respectively.}\label{fig:violin_chl_mo}
\end{figure}

\paragraph{Total Suspended Solids}

\begin{figure}[ptbh] 
  a) AIMS insitu\\\includegraphics[align=t,width=0.95\linewidth]{{figures/Exploratory_Data_Analysis/Insitu/eda.year.nap_Mackay Whitsunday__Open Coastal_niskin_log\res}.pdf}\\
  b) AIMS FLNTU\\\includegraphics[align=t, width=0.95\linewidth]{{figures/Exploratory_Data_Analysis/FLNTU/eda.year.nap_Mackay Whitsunday__Open Coastal_flntu_log\res}.pdf}\\
  c) Satellite\\\includegraphics[align=t, width=0.95\linewidth]{{figures/Exploratory_Data_Analysis/Satellite/eda.year.nap_Mackay Whitsunday__Open Coastal__log\res}.png}}\\
  d) eReefs\\\includegraphics[align=t, width=0.95\linewidth]{{figures/Exploratory_Data_Analysis/eReefs/eda.year.nap_Mackay Whitsunday__Open Coastal_eReefs_log\res}.png}}\\
  e) eReefs926\\\includegraphics[align=t, width=0.95\linewidth]{{figures/Exploratory_Data_Analysis/eReefs926/eda.year.nap_Mackay Whitsunday__Open Coastal_eReefs926_log\res}.png}}\\
\caption[Observed TSS data for the Mackay Whitsunday Open Coastal Zone (grouped annually)]{Observed (logarithmic axis with violin plot overlay) Total Suspended Solids data for the
  Mackay Whitsunday Open Coastal
  Zone from a) AIMS insitu, b) AIMS FLNTU, c) Satellite, d) eReefs and e)
eReefs926.  Observations are ordered over time and colored conditional on season as Wet (blue
symbols) and Dry (red symbols).  Blue smoother represents Generalized Additive Mixed Model within a
water year and purple line represents average within the water year.  Horizontal red, black and
green dashed lines denote the twice threshold, threshold and half threshold values respectively.
Red and green background shading indicates the range (10\% shade: x4,/4; 30\% shade: x2,/2)
above and below threshold respectively.}\label{fig:violin_nap_mo}
\end{figure}

\paragraph{Secchi Depth}

\begin{figure}[ptbh] 
  a) AIMS insitu\\\includegraphics[align=t,width=0.95\linewidth]{{figures/Exploratory_Data_Analysis/Insitu/eda.year.sd_Mackay Whitsunday__Open Coastal_niskin_log\res}.pdf}\\
  b) AIMS FLNTU\\\includegraphics[align=t, width=0.95\linewidth]{{figures/Exploratory_Data_Analysis/FLNTU/eda.year.sd_Mackay Whitsunday__Open Coastal_flntu_log\res}.pdf}\\
  c) Satellite\\\includegraphics[align=t, width=0.95\linewidth]{{figures/Exploratory_Data_Analysis/Satellite/eda.year.sd_Mackay Whitsunday__Open Coastal__log\res}.png}}\\
  d) eReefs\\\includegraphics[align=t, width=0.95\linewidth]{{figures/Exploratory_Data_Analysis/eReefs/eda.year.sd_Mackay Whitsunday__Open Coastal_eReefs_log\res}.png}}\\
  e) eReefs926\\\includegraphics[align=t, width=0.95\linewidth]{{figures/Exploratory_Data_Analysis/eReefs926/eda.year.sd_Mackay Whitsunday__Open Coastal_eReefs926_log\res}.png}}\\
\caption[Observed Secchi depth data for the Mackay Whitsunday Open Coastal Zone (grouped annually)]{Observed (logarithmic axis with violin plot overlay) Secchi Depth data for the
  Mackay Whitsunday Open Coastal
  Zone from a) AIMS insitu, b) AIMS FLNTU, c) Satellite, d) eReefs and e)
eReefs926.  Observations are ordered over time and colored conditional on season as Wet (blue
symbols) and Dry (red symbols).  Blue smoother represents Generalized Additive Mixed Model within a
water year and purple line represents average within the water year.  Horizontal red, black and
green dashed lines denote the twice threshold, threshold and half threshold values respectively.
Red and green background shading indicates the range (10\% shade: x4,/4; 30\% shade: x2,/2)
above and below threshold respectively.}\label{fig:violin_sd_mo}
\end{figure}

\paragraph{NOx}

\begin{figure}[ptbh] 
  a) AIMS insitu\\\includegraphics[align=t,width=0.95\linewidth]{{figures/Exploratory_Data_Analysis/Insitu/eda.year.NOx_Mackay Whitsunday__Open Coastal_niskin_log\res}.pdf}\\
  b) AIMS FLNTU\\\includegraphics[align=t, width=0.95\linewidth]{{figures/Exploratory_Data_Analysis/FLNTU/eda.year.NOx_Mackay Whitsunday__Open Coastal_flntu_log\res}.pdf}\\
  c) Satellite\\\includegraphics[align=t, width=0.95\linewidth]{{figures/Exploratory_Data_Analysis/Satellite/eda.year.NOx_Mackay Whitsunday__Open Coastal__log\res}.png}}\\
  d) eReefs\\\includegraphics[align=t, width=0.95\linewidth]{{figures/Exploratory_Data_Analysis/eReefs/eda.year.NOx_Mackay Whitsunday__Open Coastal_eReefs_log\res}.png}}\\
  e) eReefs926\\\includegraphics[align=t, width=0.95\linewidth]{{figures/Exploratory_Data_Analysis/eReefs926/eda.year.NOx_Mackay Whitsunday__Open Coastal_eReefs926_log\res}.png}}\\
\caption[Observed NOx data for the Mackay Whitsunday Open Coastal Zone (grouped annually)]{Observed (logarithmic axis with violin plot overlay) NOx data for the
  Mackay Whitsunday Open Coastal
  Zone from a) AIMS insitu, b) AIMS FLNTU, c) Satellite, d) eReefs and e)
eReefs926.  Observations are ordered over time and colored conditional on season as Wet (blue
symbols) and Dry (red symbols).  Blue smoother represents Generalized Additive Mixed Model within a
water year and purple line represents average within the water year.  Horizontal red, black and
green dashed lines denote the twice threshold, threshold and half threshold values respectively.
Red and green background shading indicates the range (10\% shade: x4,/4; 30\% shade: x2,/2)
above and below threshold respectively.}\label{fig:violin_NOx_mo}
\end{figure}

\subsubsection{Mackay Whitsunday, Midshelf}
\paragraph{Chlorophyll}

\begin{figure}[ptbh] 
  a) AIMS insitu\\\includegraphics[align=t,width=0.95\linewidth]{{figures/Exploratory_Data_Analysis/Insitu/eda.year.chl_Mackay Whitsunday__Midshelf_niskin_log\res}.pdf}\\
  b) AIMS FLNTU\\\includegraphics[align=t, width=0.95\linewidth]{{figures/Exploratory_Data_Analysis/FLNTU/eda.year.chl_Mackay Whitsunday__Midshelf_flntu_log\res}.pdf}\\
  c) Satellite\\\includegraphics[align=t, width=0.95\linewidth]{{figures/Exploratory_Data_Analysis/Satellite/eda.year.chl_Mackay Whitsunday__Midshelf__log\res}.png}}\\
  d) eReefs\\\includegraphics[align=t, width=0.95\linewidth]{{figures/Exploratory_Data_Analysis/eReefs/eda.year.chl_Mackay Whitsunday__Midshelf_eReefs_log\res}.png}}\\
  e) eReefs926\\\includegraphics[align=t, width=0.95\linewidth]{{figures/Exploratory_Data_Analysis/eReefs926/eda.year.chl_Mackay Whitsunday__Midshelf_eReefs926_log\res}.png}}\\
\caption[Observed Chlorophyll-a data for the Mackay Whitsunday Midshelf Zone (grouped annually)]{Observed (logarithmic axis with violin plot overlay) Chlorophyll-a data for the
  Mackay Whitsunday Midshelf
  Zone from a) AIMS insitu, b) AIMS FLNTU, c) Satellite, d) eReefs and e)
eReefs926.  Observations are ordered over time and colored conditional on season as Wet (blue
symbols) and Dry (red symbols).  Blue smoother represents Generalized Additive Mixed Model within a
water year and purple line represents average within the water year.  Horizontal red, black and
green dashed lines denote the twice threshold, threshold and half threshold values respectively.
Red and green background shading indicates the range (10\% shade: x4,/4; 30\% shade: x2,/2)
above and below threshold respectively.}\label{fig:violin_chl_mm}
\end{figure}

\paragraph{Total Suspended Solids}

\begin{figure}[ptbh] 
  a) AIMS insitu\\\includegraphics[align=t,width=0.95\linewidth]{{figures/Exploratory_Data_Analysis/Insitu/eda.year.nap_Mackay Whitsunday__Midshelf_niskin_log\res}.pdf}\\
  b) AIMS FLNTU\\\includegraphics[align=t, width=0.95\linewidth]{{figures/Exploratory_Data_Analysis/FLNTU/eda.year.nap_Mackay Whitsunday__Midshelf_flntu_log\res}.pdf}\\
  c) Satellite\\\includegraphics[align=t, width=0.95\linewidth]{{figures/Exploratory_Data_Analysis/Satellite/eda.year.nap_Mackay Whitsunday__Midshelf__log\res}.png}}\\
  d) eReefs\\\includegraphics[align=t, width=0.95\linewidth]{{figures/Exploratory_Data_Analysis/eReefs/eda.year.nap_Mackay Whitsunday__Midshelf_eReefs_log\res}.png}}\\
  e) eReefs926\\\includegraphics[align=t, width=0.95\linewidth]{{figures/Exploratory_Data_Analysis/eReefs926/eda.year.nap_Mackay Whitsunday__Midshelf_eReefs926_log\res}.png}}\\
\caption[Observed TSS data for the Mackay Whitsunday Midshelf Zone (grouped annually)]{Observed (logarithmic axis with violin plot overlay) Total Suspended Solids data for the
  Mackay Whitsunday Midshelf
  Zone from a) AIMS insitu, b) AIMS FLNTU, c) Satellite, d) eReefs and e)
eReefs926.  Observations are ordered over time and colored conditional on season as Wet (blue
symbols) and Dry (red symbols).  Blue smoother represents Generalized Additive Mixed Model within a
water year and purple line represents average within the water year.  Horizontal red, black and
green dashed lines denote the twice threshold, threshold and half threshold values respectively.
Red and green background shading indicates the range (10\% shade: x4,/4; 30\% shade: x2,/2)
above and below threshold respectively.}\label{fig:violin_nap_mm}
\end{figure}

\paragraph{Secchi Depth}

\begin{figure}[ptbh] 
  a) AIMS insitu\\\includegraphics[align=t,width=0.95\linewidth]{{figures/Exploratory_Data_Analysis/Insitu/eda.year.sd_Mackay Whitsunday__Midshelf_niskin_log\res}.pdf}\\
  b) AIMS FLNTU\\\includegraphics[align=t, width=0.95\linewidth]{{figures/Exploratory_Data_Analysis/FLNTU/eda.year.sd_Mackay Whitsunday__Midshelf_flntu_log\res}.pdf}\\
  c) Satellite\\\includegraphics[align=t, width=0.95\linewidth]{{figures/Exploratory_Data_Analysis/Satellite/eda.year.sd_Mackay Whitsunday__Midshelf__log\res}.png}}\\
  d) eReefs\\\includegraphics[align=t, width=0.95\linewidth]{{figures/Exploratory_Data_Analysis/eReefs/eda.year.sd_Mackay Whitsunday__Midshelf_eReefs_log\res}.png}}\\
  e) eReefs926\\\includegraphics[align=t, width=0.95\linewidth]{{figures/Exploratory_Data_Analysis/eReefs926/eda.year.sd_Mackay Whitsunday__Midshelf_eReefs926_log\res}.png}}\\
\caption[Observed Secchi depth data for the Mackay Whitsunday Midshelf Zone (grouped annually)]{Observed (logarithmic axis with violin plot overlay) Secchi Depth data for the
  Mackay Whitsunday Midshelf
  Zone from a) AIMS insitu, b) AIMS FLNTU, c) Satellite, d) eReefs and e)
eReefs926.  Observations are ordered over time and colored conditional on season as Wet (blue
symbols) and Dry (red symbols).  Blue smoother represents Generalized Additive Mixed Model within a
water year and purple line represents average within the water year.  Horizontal red, black and
green dashed lines denote the twice threshold, threshold and half threshold values respectively.
Red and green background shading indicates the range (10\% shade: x4,/4; 30\% shade: x2,/2)
above and below threshold respectively.}\label{fig:violin_sd_mm}
\end{figure}

\paragraph{NOx}

\begin{figure}[ptbh] 
  a) AIMS insitu\\\includegraphics[align=t,width=0.95\linewidth]{{figures/Exploratory_Data_Analysis/Insitu/eda.year.NOx_Mackay Whitsunday__Midshelf_niskin_log\res}.pdf}\\
  b) AIMS FLNTU\\\includegraphics[align=t, width=0.95\linewidth]{{figures/Exploratory_Data_Analysis/FLNTU/eda.year.NOx_Mackay Whitsunday__Midshelf_flntu_log\res}.pdf}\\
  c) Satellite\\\includegraphics[align=t, width=0.95\linewidth]{{figures/Exploratory_Data_Analysis/Satellite/eda.year.NOx_Mackay Whitsunday__Midshelf__log\res}.png}}\\
  d) eReefs\\\includegraphics[align=t, width=0.95\linewidth]{{figures/Exploratory_Data_Analysis/eReefs/eda.year.NOx_Mackay Whitsunday__Midshelf_eReefs_log\res}.png}}\\
  e) eReefs926\\\includegraphics[align=t, width=0.95\linewidth]{{figures/Exploratory_Data_Analysis/eReefs926/eda.year.NOx_Mackay Whitsunday__Midshelf_eReefs926_log\res}.png}}\\
\caption[Observed NOx data for the Mackay Whitsunday Midshelf Zone (grouped annually)]{Observed (logarithmic axis with violin plot overlay) NOx data for the
  Mackay Whitsunday Midshelf
  Zone from a) AIMS insitu, b) AIMS FLNTU, c) Satellite, d) eReefs and e)
eReefs926.  Observations are ordered over time and colored conditional on season as Wet (blue
symbols) and Dry (red symbols).  Blue smoother represents Generalized Additive Mixed Model within a
water year and purple line represents average within the water year.  Horizontal red, black and
green dashed lines denote the twice threshold, threshold and half threshold values respectively.
Red and green background shading indicates the range (10\% shade: x4,/4; 30\% shade: x2,/2)
above and below threshold respectively.}\label{fig:violin_NOx_mm}
\end{figure}

\subsubsection{Mackay Whitsunday, Offshore}
\paragraph{Chlorophyll}

\begin{figure}[ptbh] 
  a) AIMS insitu\\\includegraphics[align=t,width=0.95\linewidth]{{figures/Exploratory_Data_Analysis/Insitu/eda.year.chl_Mackay Whitsunday__Offshore_niskin_log\res}.pdf}\\
  b) AIMS FLNTU\\\includegraphics[align=t, width=0.95\linewidth]{{figures/Exploratory_Data_Analysis/FLNTU/eda.year.chl_Mackay Whitsunday__Offshore_flntu_log\res}.pdf}\\
  c) Satellite\\\includegraphics[align=t, width=0.95\linewidth]{{figures/Exploratory_Data_Analysis/Satellite/eda.year.chl_Mackay Whitsunday__Offshore__log\res}.png}}\\
  d) eReefs\\\includegraphics[align=t, width=0.95\linewidth]{{figures/Exploratory_Data_Analysis/eReefs/eda.year.chl_Mackay Whitsunday__Offshore_eReefs_log\res}.png}}\\
  e) eReefs926\\\includegraphics[align=t, width=0.95\linewidth]{{figures/Exploratory_Data_Analysis/eReefs926/eda.year.chl_Mackay Whitsunday__Offshore_eReefs926_log\res}.png}}\\
\caption[Observed Chlorophyll-a data for the Mackay Whitsunday Offshore Zone (grouped annually)]{Observed (logarithmic axis with violin plot overlay) Chlorophyll-a data for the
  Mackay Whitsunday Offshore
  Zone from a) AIMS insitu, b) AIMS FLNTU, c) Satellite, d) eReefs and e)
eReefs926.  Observations are ordered over time and colored conditional on season as Wet (blue
symbols) and Dry (red symbols).  Blue smoother represents Generalized Additive Mixed Model within a
water year and purple line represents average within the water year.  Horizontal red, black and
green dashed lines denote the twice threshold, threshold and half threshold values respectively.
Red and green background shading indicates the range (10\% shade: x4,/4; 30\% shade: x2,/2)
above and below threshold respectively.}\label{fig:violin_chl_mof}
\end{figure}

\paragraph{Total Suspended Solids}

\begin{figure}[ptbh] 
  a) AIMS insitu\\\includegraphics[align=t,width=0.95\linewidth]{{figures/Exploratory_Data_Analysis/Insitu/eda.year.nap_Mackay Whitsunday__Offshore_niskin_log\res}.pdf}\\
  b) AIMS FLNTU\\\includegraphics[align=t, width=0.95\linewidth]{{figures/Exploratory_Data_Analysis/FLNTU/eda.year.nap_Mackay Whitsunday__Offshore_flntu_log\res}.pdf}\\
  c) Satellite\\\includegraphics[align=t, width=0.95\linewidth]{{figures/Exploratory_Data_Analysis/Satellite/eda.year.nap_Mackay Whitsunday__Offshore__log\res}.png}}\\
  d) eReefs\\\includegraphics[align=t, width=0.95\linewidth]{{figures/Exploratory_Data_Analysis/eReefs/eda.year.nap_Mackay Whitsunday__Offshore_eReefs_log\res}.png}}\\
  e) eReefs926\\\includegraphics[align=t, width=0.95\linewidth]{{figures/Exploratory_Data_Analysis/eReefs926/eda.year.nap_Mackay Whitsunday__Offshore_eReefs926_log\res}.png}}\\
\caption[Observed TSS data for the Mackay Whitsunday Offshore Zone (grouped annually)]{Observed (logarithmic axis with violin plot overlay) Total Suspended Solids data for the
  Mackay Whitsunday Offshore
  Zone from a) AIMS insitu, b) AIMS FLNTU, c) Satellite, d) eReefs and e)
eReefs926.  Observations are ordered over time and colored conditional on season as Wet (blue
symbols) and Dry (red symbols).  Blue smoother represents Generalized Additive Mixed Model within a
water year and purple line represents average within the water year.  Horizontal red, black and
green dashed lines denote the twice threshold, threshold and half threshold values respectively.
Red and green background shading indicates the range (10\% shade: x4,/4; 30\% shade: x2,/2)
above and below threshold respectively.}\label{fig:violin_nap_mof}
\end{figure}

\paragraph{Secchi Depth}

\begin{figure}[ptbh] 
  a) AIMS insitu\\\includegraphics[align=t,width=0.95\linewidth]{{figures/Exploratory_mata_Analysis/Insitu/eda.year.sd_Mackay Whitsunday__Offshore_niskin_log\res}.pdf}\\
  b) AIMS FLNTU\\\includegraphics[align=t, width=0.95\linewidth]{{figures/Exploratory_Data_Analysis/FLNTU/eda.year.sd_Mackay Whitsunday__Offshore_flntu_log\res}.pdf}\\
  c) Satellite\\\includegraphics[align=t, width=0.95\linewidth]{{figures/Exploratory_Data_Analysis/Satellite/eda.year.sd_Mackay Whitsunday__Offshore__log\res}.png}}\\
  d) eReefs\\\includegraphics[align=t, width=0.95\linewidth]{{figures/Exploratory_Data_Analysis/eReefs/eda.year.sd_Mackay Whitsunday__Offshore_eReefs_log\res}.png}}\\
  e) eReefs926\\\includegraphics[align=t, width=0.95\linewidth]{{figures/Exploratory_Data_Analysis/eReefs926/eda.year.sd_Mackay Whitsunday__Offshore_eReefs926_log\res}.png}}\\
\caption[Observed Secchi depth data for the Mackay Whitsunday Offshore Zone (grouped annually)]{Observed (logarithmic axis with violin plot overlay) Secchi Depth data for the
  Mackay Whitsunday Offshore
  Zone from a) AIMS insitu, b) AIMS FLNTU, c) Satellite, d) eReefs and e)
eReefs926.  Observations are ordered over time and colored conditional on season as Wet (blue
symbols) and Dry (red symbols).  Blue smoother represents Generalized Additive Mixed Model within a
water year and purple line represents average within the water year.  Horizontal red, black and
green dashed lines denote the twice threshold, threshold and half threshold values respectively.
Red and green background shading indicates the range (10\% shade: x4,/4; 30\% shade: x2,/2)
above and below threshold respectively.}\label{fig:violin_sd_mof}
\end{figure}

\paragraph{NOx}

\begin{figure}[ptbh] 
  a) AIMS insitu\\\includegraphics[align=t,width=0.95\linewidth]{{figures/Exploratory_Data_Analysis/Insitu/eda.year.NOx_Mackay Whitsunday__Offshore_niskin_log\res}.pdf}\\
  b) AIMS FLNTU\\\includegraphics[align=t, width=0.95\linewidth]{{figures/Exploratory_Data_Analysis/FLNTU/eda.year.NOx_Mackay Whitsunday__Offshore_flntu_log\res}.pdf}\\
  c) Satellite\\\includegraphics[align=t, width=0.95\linewidth]{{figures/Exploratory_Data_Analysis/Satellite/eda.year.NOx_Mackay Whitsunday__Offshore__log\res}.png}}\\
  d) eReefs\\\includegraphics[align=t, width=0.95\linewidth]{{figures/Exploratory_Data_Analysis/eReefs/eda.year.NOx_Mackay Whitsunday__Offshore_eReefs_log\res}.png}}\\
  e) eReefs926\\\includegraphics[align=t, width=0.95\linewidth]{{figures/Exploratory_Data_Analysis/eReefs926/eda.year.NOx_Mackay Whitsunday__Offshore_eReefs926_log\res}.png}}\\
\caption[Observed NOx data for the Mackay Whitsunday Offshore Zone (grouped annually)]{Observed (logarithmic axis with violin plot overlay) NOx data for the
  Mackay Whitsunday Offshore
  Zone from a) AIMS insitu, b) AIMS FLNTU, c) Satellite, d) eReefs and e)
eReefs926.  Observations are ordered over time and colored conditional on season as Wet (blue
symbols) and Dry (red symbols).  Blue smoother represents Generalized Additive Mixed Model within a
water year and purple line represents average within the water year.  Horizontal red, black and
green dashed lines denote the twice threshold, threshold and half threshold values respectively.
Red and green background shading indicates the range (10\% shade: x4,/4; 30\% shade: x2,/2)
above and below threshold respectively.}\label{fig:violin_NOx_mof}
\end{figure}


%% Fitzroy
\subsubsection{Fitzroy, Enclosed Coastal}
\paragraph{Chlorophyll}

\begin{figure}[ptbh] 
  a) AIMS insitu\\\includegraphics[align=t,width=0.95\linewidth]{{figures/Exploratory_Data_Analysis/Insitu/eda.year.chl_Fitzroy__Enclosed Coastal_niskin_log\res}.pdf}\\
  b) AIMS FLNTU\\\includegraphics[align=t, width=0.95\linewidth]{{figures/Exploratory_Data_Analysis/FLNTU/eda.year.chl_Fitzroy__Enclosed Coastal_flntu_log\res}.pdf}\\
  c) Satellite\\\includegraphics[align=t, width=0.95\linewidth]{{figures/Exploratory_Data_Analysis/Satellite/eda.year.chl_Fitzroy__Enclosed Coastal__log\res}.png}}\\
  d) eReefs\\\includegraphics[align=t, width=0.95\linewidth]{{figures/Exploratory_Data_Analysis/eReefs/eda.year.chl_Fitzroy__Enclosed Coastal_eReefs_log\res}.png}}\\
  e) eReefs926\\\includegraphics[align=t, width=0.95\linewidth]{{figures/Exploratory_Data_Analysis/eReefs926/eda.year.chl_Fitzroy__Enclosed Coastal_eReefs926_log\res}.png}}\\
\caption[Observed Chlorophyll-a data for the Fitzroy Enclosed Coastal Zone (grouped annually)]{Observed (logarithmic axis with violin plot overlay) Chlorophyll-a data for the
  Fitzroy Enclosed Coastal
  Zone from a) AIMS insitu, b) AIMS FLNTU, c) Satellite, d) eReefs and e)
eReefs926.  Observations are ordered over time and colored conditional on season as Wet (blue
symbols) and Dry (red symbols).  Blue smoother represents Generalized Additive Mixed Model within a
water year and purple line represents average within the water year.  Horizontal red, black and
green dashed lines denote the twice threshold, threshold and half threshold values respectively.
Red and green background shading indicates the range (10\% shade: x4,/4; 30\% shade: x2,/2)
above and below threshold respectively.}\label{fig:violin_chl_fe}
\end{figure}

\paragraph{Total Suspended Solids}

\begin{figure}[ptbh] 
  a) AIMS insitu\\\includegraphics[align=t,width=0.95\linewidth]{{figures/Exploratory_Data_Analysis/Insitu/eda.year.nap_Fitzroy__Enclosed Coastal_niskin_log\res}.pdf}\\
  b) AIMS FLNTU\\\includegraphics[align=t, width=0.95\linewidth]{{figures/Exploratory_Data_Analysis/FLNTU/eda.year.nap_Fitzroy__Enclosed Coastal_flntu_log\res}.pdf}\\
  c) Satellite\\\includegraphics[align=t, width=0.95\linewidth]{{figures/Exploratory_Data_Analysis/Satellite/eda.year.nap_Fitzroy__Enclosed Coastal__log\res}.png}}\\
  d) eReefs\\\includegraphics[align=t, width=0.95\linewidth]{{figures/Exploratory_Data_Analysis/eReefs/eda.year.nap_Fitzroy__Enclosed Coastal_eReefs_log\res}.png}}\\
  e) eReefs926\\\includegraphics[align=t, width=0.95\linewidth]{{figures/Exploratory_Data_Analysis/eReefs926/eda.year.nap_Fitzroy__Enclosed Coastal_eReefs926_log\res}.png}}\\
\caption[Observed TSS data for the Fitzroy Enclosed Coastal Zone (grouped annually)]{Observed (logarithmic axis with violin plot overlay) Total Suspended Solids data for the
  Fitzroy Enclosed Coastal
  Zone from a) AIMS insitu, b) AIMS FLNTU, c) Satellite, d) eReefs and e)
eReefs926.  Observations are ordered over time and colored conditional on season as Wet (blue
symbols) and Dry (red symbols).  Blue smoother represents Generalized Additive Mixed Model within a
water year and purple line represents average within the water year.  Horizontal red, black and
green dashed lines denote the twice threshold, threshold and half threshold values respectively.
Red and green background shading indicates the range (10\% shade: x4,/4; 30\% shade: x2,/2)
above and below threshold respectively.}\label{fig:violin_nap_fe}
\end{figure}

\paragraph{Secchi Depth}

\begin{figure}[ptbh] 
  a) AIMS insitu\\\includegraphics[align=t,width=0.95\linewidth]{{figures/Exploratory_Data_Analysis/Insitu/eda.year.sd_Fitzroy__Enclosed Coastal_niskin_log\res}.pdf}\\
  b) AIMS FLNTU\\\includegraphics[align=t, width=0.95\linewidth]{{figures/Exploratory_Data_Analysis/FLNTU/eda.year.sd_Fitzroy__Enclosed Coastal_flntu_log\res}.pdf}\\
  c) Satellite\\\includegraphics[align=t, width=0.95\linewidth]{{figures/Exploratory_Data_Analysis/Satellite/eda.year.sd_Fitzroy__Enclosed Coastal__log\res}.png}}\\
  d) eReefs\\\includegraphics[align=t, width=0.95\linewidth]{{figures/Exploratory_Data_Analysis/eReefs/eda.year.sd_Fitzroy__Enclosed Coastal_eReefs_log\res}.png}}\\
  e) eReefs926\\\includegraphics[align=t, width=0.95\linewidth]{{figures/Exploratory_Data_Analysis/eReefs926/eda.year.sd_Fitzroy__Enclosed Coastal_eReefs926_log\res}.png}}\\
\caption[Observed Secchi depth data for the Fitzroy Enclosed Coastal Zone (grouped annually)]{Observed (logarithmic axis with violin plot overlay) Secchi Depth data for the
  Fitzroy Enclosed Coastal
  Zone from a) AIMS insitu, b) AIMS FLNTU, c) Satellite, d) eReefs and e)
eReefs926.  Observations are ordered over time and colored conditional on season as Wet (blue
symbols) and Dry (red symbols).  Blue smoother represents Generalized Additive Mixed Model within a
water year and purple line represents average within the water year.  Horizontal red, black and
green dashed lines denote the twice threshold, threshold and half threshold values respectively.
Red and green background shading indicates the range (10\% shade: x4,/4; 30\% shade: x2,/2)
above and below threshold respectively.}\label{fig:violin_sd_fe}
\end{figure}

\paragraph{NOx}

\begin{figure}[ptbh] 
  a) AIMS insitu\\\includegraphics[align=t,width=0.95\linewidth]{{figures/Exploratory_Data_Analysis/Insitu/eda.year.NOx_Fitzroy__Enclosed Coastal_niskin_log\res}.pdf}\\
  b) AIMS FLNTU\\\includegraphics[align=t, width=0.95\linewidth]{{figures/Exploratory_Data_Analysis/FLNTU/eda.year.NOx_Fitzroy__Enclosed Coastal_flntu_log\res}.pdf}\\
  c) Satellite\\\includegraphics[align=t, width=0.95\linewidth]{{figures/Exploratory_Data_Analysis/Satellite/eda.year.NOx_Fitzroy__Enclosed Coastal__log\res}.png}}\\
  d) eReefs\\\includegraphics[align=t, width=0.95\linewidth]{{figures/Exploratory_Data_Analysis/eReefs/eda.year.NOx_Fitzroy__Enclosed Coastal_eReefs_log\res}.png}}\\
  e) eReefs926\\\includegraphics[align=t, width=0.95\linewidth]{{figures/Exploratory_Data_Analysis/eReefs926/eda.year.NOx_Fitzroy__Enclosed Coastal_eReefs926_log\res}.png}}\\
\caption[Observed NOx data for the Fitzroy Enclosed Coastal Zone (grouped annually)]{Observed (logarithmic axis with violin plot overlay) NOx data for the
  Fitzroy Enclosed Coastal
  Zone from a) AIMS insitu, b) AIMS FLNTU, c) Satellite, d) eReefs and e)
eReefs926.  Observations are ordered over time and colored conditional on season as Wet (blue
symbols) and Dry (red symbols).  Blue smoother represents Generalized Additive Mixed Model within a
water year and purple line represents average within the water year.  Horizontal red, black and
green dashed lines denote the twice threshold, threshold and half threshold values respectively.
Red and green background shading indicates the range (10\% shade: x4,/4; 30\% shade: x2,/2)
above and below threshold respectively.}\label{fig:violin_NOx_fe}
\end{figure}


\subsubsection{Fitzroy, Open Coastal}
\paragraph{Chlorophyll}

\begin{figure}[ptbh] 
  a) AIMS insitu\\\includegraphics[align=t,width=0.95\linewidth]{{figures/Exploratory_Data_Analysis/Insitu/eda.year.chl_Fitzroy__Open Coastal_niskin_log\res}.pdf}\\
  b) AIMS FLNTU\\\includegraphics[align=t, width=0.95\linewidth]{{figures/Exploratory_Data_Analysis/FLNTU/eda.year.chl_Fitzroy__Open Coastal_flntu_log\res}.pdf}\\
  c) Satellite\\\includegraphics[align=t, width=0.95\linewidth]{{figures/Exploratory_Data_Analysis/Satellite/eda.year.chl_Fitzroy__Open Coastal__log\res}.png}}\\
  d) eReefs\\\includegraphics[align=t, width=0.95\linewidth]{{figures/Exploratory_Data_Analysis/eReefs/eda.year.chl_Fitzroy__Open Coastal_eReefs_log\res}.png}}\\
  e) eReefs926\\\includegraphics[align=t, width=0.95\linewidth]{{figures/Exploratory_Data_Analysis/eReefs926/eda.year.chl_Fitzroy__Open Coastal_eReefs926_log\res}.png}}\\
\caption[Observed Chlorophyll-a data for the Fitzroy Open Coastal Zone (grouped annually)]{Observed (logarithmic axis with violin plot overlay) Chlorophyll-a data for the
  Fitzroy Open Coastal
  Zone from a) AIMS insitu, b) AIMS FLNTU, c) Satellite, d) eReefs and e)
eReefs926.  Observations are ordered over time and colored conditional on season as Wet (blue
symbols) and Dry (red symbols).  Blue smoother represents Generalized Additive Mixed Model within a
water year and purple line represents average within the water year.  Horizontal red, black and
green dashed lines denote the twice threshold, threshold and half threshold values respectively.
Red and green background shading indicates the range (10\% shade: x4,/4; 30\% shade: x2,/2)
above and below threshold respectively.}\label{fig:violin_chl_fo}
\end{figure}

\paragraph{Total Suspended Solids}

\begin{figure}[ptbh] 
  a) AIMS insitu\\\includegraphics[align=t,width=0.95\linewidth]{{figures/Exploratory_Data_Analysis/Insitu/eda.year.nap_Fitzroy__Open Coastal_niskin_log\res}.pdf}\\
  b) AIMS FLNTU\\\includegraphics[align=t, width=0.95\linewidth]{{figures/Exploratory_Data_Analysis/FLNTU/eda.year.nap_Fitzroy__Open Coastal_flntu_log\res}.pdf}\\
  c) Satellite\\\includegraphics[align=t, width=0.95\linewidth]{{figures/Exploratory_Data_Analysis/Satellite/eda.year.nap_Fitzroy__Open Coastal__log\res}.png}}\\
  d) eReefs\\\includegraphics[align=t, width=0.95\linewidth]{{figures/Exploratory_Data_Analysis/eReefs/eda.year.nap_Fitzroy__Open Coastal_eReefs_log\res}.png}}\\
  e) eReefs926\\\includegraphics[align=t, width=0.95\linewidth]{{figures/Exploratory_Data_Analysis/eReefs926/eda.year.nap_Fitzroy__Open Coastal_eReefs926_log\res}.png}}\\
\caption[Observed TSS data for the Fitzroy Open Coastal Zone (grouped annually)]{Observed (logarithmic axis with violin plot overlay) Total Suspended Solids data for the
  Fitzroy Open Coastal
  Zone from a) AIMS insitu, b) AIMS FLNTU, c) Satellite, d) eReefs and e)
eReefs926.  Observations are ordered over time and colored conditional on season as Wet (blue
symbols) and Dry (red symbols).  Blue smoother represents Generalized Additive Mixed Model within a
water year and purple line represents average within the water year.  Horizontal red, black and
green dashed lines denote the twice threshold, threshold and half threshold values respectively.
Red and green background shading indicates the range (10\% shade: x4,/4; 30\% shade: x2,/2)
above and below threshold respectively.}\label{fig:violin_nap_fo}
\end{figure}

\paragraph{Secchi Depth}

\begin{figure}[ptbh] 
  a) AIMS insitu\\\includegraphics[align=t,width=0.95\linewidth]{{figures/Exploratory_Data_Analysis/Insitu/eda.year.sd_Fitzroy__Open Coastal_niskin_log\res}.pdf}\\
  b) AIMS FLNTU\\\includegraphics[align=t, width=0.95\linewidth]{{figures/Exploratory_Data_Analysis/FLNTU/eda.year.sd_Fitzroy__Open Coastal_flntu_log\res}.pdf}\\
  c) Satellite\\\includegraphics[align=t, width=0.95\linewidth]{{figures/Exploratory_Data_Analysis/Satellite/eda.year.sd_Fitzroy__Open Coastal__log\res}.png}}\\
  d) eReefs\\\includegraphics[align=t, width=0.95\linewidth]{{figures/Exploratory_Data_Analysis/eReefs/eda.year.sd_Fitzroy__Open Coastal_eReefs_log\res}.png}}\\
  e) eReefs926\\\includegraphics[align=t, width=0.95\linewidth]{{figures/Exploratory_Data_Analysis/eReefs926/eda.year.sd_Fitzroy__Open Coastal_eReefs926_log\res}.png}}\\
\caption[Observed Secchi depth data for the Fitzroy Open Coastal Zone (grouped annually)]{Observed (logarithmic axis with violin plot overlay) Secchi Depth data for the
  Fitzroy Open Coastal
  Zone from a) AIMS insitu, b) AIMS FLNTU, c) Satellite, d) eReefs and e)
eReefs926.  Observations are ordered over time and colored conditional on season as Wet (blue
symbols) and Dry (red symbols).  Blue smoother represents Generalized Additive Mixed Model within a
water year and purple line represents average within the water year.  Horizontal red, black and
green dashed lines denote the twice threshold, threshold and half threshold values respectively.
Red and green background shading indicates the range (10\% shade: x4,/4; 30\% shade: x2,/2)
above and below threshold respectively.}\label{fig:violin_sd_fo}
\end{figure}

\paragraph{NOx}

\begin{figure}[ptbh] 
  a) AIMS insitu\\\includegraphics[align=t,width=0.95\linewidth]{{figures/Exploratory_Data_Analysis/Insitu/eda.year.NOx_Fitzroy__Open Coastal_niskin_log\res}.pdf}\\
  b) AIMS FLNTU\\\includegraphics[align=t, width=0.95\linewidth]{{figures/Exploratory_Data_Analysis/FLNTU/eda.year.NOx_Fitzroy__Open Coastal_flntu_log\res}.pdf}\\
  c) Satellite\\\includegraphics[align=t, width=0.95\linewidth]{{figures/Exploratory_Data_Analysis/Satellite/eda.year.NOx_Fitzroy__Open Coastal__log\res}.png}}\\
  d) eReefs\\\includegraphics[align=t, width=0.95\linewidth]{{figures/Exploratory_Data_Analysis/eReefs/eda.year.NOx_Fitzroy__Open Coastal_eReefs_log\res}.png}}\\
  e) eReefs926\\\includegraphics[align=t, width=0.95\linewidth]{{figures/Exploratory_Data_Analysis/eReefs926/eda.year.NOx_Fitzroy__Open Coastal_eReefs926_log\res}.png}}\\
\caption[Observed NOx data for the Fitzroy Open Coastal Zone (grouped annually)]{Observed (logarithmic axis with violin plot overlay) NOx data for the
  Fitzroy Open Coastal
  Zone from a) AIMS insitu, b) AIMS FLNTU, c) Satellite, d) eReefs and e)
eReefs926.  Observations are ordered over time and colored conditional on season as Wet (blue
symbols) and Dry (red symbols).  Blue smoother represents Generalized Additive Mixed Model within a
water year and purple line represents average within the water year.  Horizontal red, black and
green dashed lines denote the twice threshold, threshold and half threshold values respectively.
Red and green background shading indicates the range (10\% shade: x4,/4; 30\% shade: x2,/2)
above and below threshold respectively.}\label{fig:violin_NOx_fo}
\end{figure}

\subsubsection{Fitzroy, Midshelf}
\paragraph{Chlorophyll}

\begin{figure}[ptbh] 
  a) AIMS insitu\\\includegraphics[align=t,width=0.95\linewidth]{{figures/Exploratory_Data_Analysis/Insitu/eda.year.chl_Fitzroy__Midshelf_niskin_log\res}.pdf}\\
  b) AIMS FLNTU\\\includegraphics[align=t, width=0.95\linewidth]{{figures/Exploratory_Data_Analysis/FLNTU/eda.year.chl_Fitzroy__Midshelf_flntu_log\res}.pdf}\\
  c) Satellite\\\includegraphics[align=t, width=0.95\linewidth]{{figures/Exploratory_Data_Analysis/Satellite/eda.year.chl_Fitzroy__Midshelf__log\res}.png}}\\
  d) eReefs\\\includegraphics[align=t, width=0.95\linewidth]{{figures/Exploratory_Data_Analysis/eReefs/eda.year.chl_Fitzroy__Midshelf_eReefs_log\res}.png}}\\
  e) eReefs926\\\includegraphics[align=t, width=0.95\linewidth]{{figures/Exploratory_Data_Analysis/eReefs926/eda.year.chl_Fitzroy__Midshelf_eReefs926_log\res}.png}}\\
\caption[Observed Chlorophyll-a data for the Fitzroy Midshelf Zone (grouped annually)]{Observed (logarithmic axis with violin plot overlay) Chlorophyll-a data for the
  Fitzroy Midshelf
  Zone from a) AIMS insitu, b) AIMS FLNTU, c) Satellite, d) eReefs and e)
eReefs926.  Observations are ordered over time and colored conditional on season as Wet (blue
symbols) and Dry (red symbols).  Blue smoother represents Generalized Additive Mixed Model within a
water year and purple line represents average within the water year.  Horizontal red, black and
green dashed lines denote the twice threshold, threshold and half threshold values respectively.
Red and green background shading indicates the range (10\% shade: x4,/4; 30\% shade: x2,/2)
above and below threshold respectively.}\label{fig:violin_chl_fm}
\end{figure}

\paragraph{Total Suspended Solids}

\begin{figure}[ptbh] 
  a) AIMS insitu\\\includegraphics[align=t,width=0.95\linewidth]{{figures/Exploratory_Data_Analysis/Insitu/eda.year.nap_Fitzroy__Midshelf_niskin_log\res}.pdf}\\
  b) AIMS FLNTU\\\includegraphics[align=t, width=0.95\linewidth]{{figures/Exploratory_Data_Analysis/FLNTU/eda.year.nap_Fitzroy__Midshelf_flntu_log\res}.pdf}\\
  c) Satellite\\\includegraphics[align=t, width=0.95\linewidth]{{figures/Exploratory_Data_Analysis/Satellite/eda.year.nap_Fitzroy__Midshelf__log\res}.png}}\\
  d) eReefs\\\includegraphics[align=t, width=0.95\linewidth]{{figures/Exploratory_Data_Analysis/eReefs/eda.year.nap_Fitzroy__Midshelf_eReefs_log\res}.png}}\\
  e) eReefs926\\\includegraphics[align=t, width=0.95\linewidth]{{figures/Exploratory_Data_Analysis/eReefs926/eda.year.nap_Fitzroy__Midshelf_eReefs926_log\res}.png}}\\
\caption[Observed TSS data for the Fitzroy Midshelf Zone (grouped annually)]{Observed (logarithmic axis with violin plot overlay) Total Suspended Solids data for the
  Fitzroy Midshelf
  Zone from a) AIMS insitu, b) AIMS FLNTU, c) Satellite, d) eReefs and e)
eReefs926.  Observations are ordered over time and colored conditional on season as Wet (blue
symbols) and Dry (red symbols).  Blue smoother represents Generalized Additive Mixed Model within a
water year and purple line represents average within the water year.  Horizontal red, black and
green dashed lines denote the twice threshold, threshold and half threshold values respectively.
Red and green background shading indicates the range (10\% shade: x4,/4; 30\% shade: x2,/2)
above and below threshold respectively.}\label{fig:violin_nap_fm}
\end{figure}

\paragraph{Secchi Depth}

\begin{figure}[ptbh] 
  a) AIMS insitu\\\includegraphics[align=t,width=0.95\linewidth]{{figures/Exploratory_Data_Analysis/Insitu/eda.year.sd_Fitzroy__Midshelf_niskin_log\res}.pdf}\\
  b) AIMS FLNTU\\\includegraphics[align=t, width=0.95\linewidth]{{figures/Exploratory_Data_Analysis/FLNTU/eda.year.sd_Fitzroy__Midshelf_flntu_log\res}.pdf}\\
  c) Satellite\\\includegraphics[align=t, width=0.95\linewidth]{{figures/Exploratory_Data_Analysis/Satellite/eda.year.sd_Fitzroy__Midshelf__log\res}.png}}\\
  d) eReefs\\\includegraphics[align=t, width=0.95\linewidth]{{figures/Exploratory_Data_Analysis/eReefs/eda.year.sd_Fitzroy__Midshelf_eReefs_log\res}.png}}\\
  e) eReefs926\\\includegraphics[align=t, width=0.95\linewidth]{{figures/Exploratory_Data_Analysis/eReefs926/eda.year.sd_Fitzroy__Midshelf_eReefs926_log\res}.png}}\\
\caption[Observed Secchi depth data for the Fitzroy Midshelf Zone (grouped annually)]{Observed (logarithmic axis with violin plot overlay) Secchi Depth data for the
  Fitzroy Midshelf
  Zone from a) AIMS insitu, b) AIMS FLNTU, c) Satellite, d) eReefs and e)
eReefs926.  Observations are ordered over time and colored conditional on season as Wet (blue
symbols) and Dry (red symbols).  Blue smoother represents Generalized Additive Mixed Model within a
water year and purple line represents average within the water year.  Horizontal red, black and
green dashed lines denote the twice threshold, threshold and half threshold values respectively.
Red and green background shading indicates the range (10\% shade: x4,/4; 30\% shade: x2,/2)
above and below threshold respectively.}\label{fig:violin_sd_fm}
\end{figure}

\paragraph{NOx}

\begin{figure}[ptbh] 
  a) AIMS insitu\\\includegraphics[align=t,width=0.95\linewidth]{{figures/Exploratory_Data_Analysis/Insitu/eda.year.NOx_Fitzroy__Midshelf_niskin_log\res}.pdf}\\
  b) AIMS FLNTU\\\includegraphics[align=t, width=0.95\linewidth]{{figures/Exploratory_Data_Analysis/FLNTU/eda.year.NOx_Fitzroy__Midshelf_flntu_log\res}.pdf}\\
  c) Satellite\\\includegraphics[align=t, width=0.95\linewidth]{{figures/Exploratory_Data_Analysis/Satellite/eda.year.NOx_Fitzroy__Midshelf__log\res}.png}}\\
  d) eReefs\\\includegraphics[align=t, width=0.95\linewidth]{{figures/Exploratory_Data_Analysis/eReefs/eda.year.NOx_Fitzroy__Midshelf_eReefs_log\res}.png}}\\
  e) eReefs926\\\includegraphics[align=t, width=0.95\linewidth]{{figures/Exploratory_Data_Analysis/eReefs926/eda.year.NOx_Fitzroy__Midshelf_eReefs926_log\res}.png}}\\
\caption[Observed NOx data for the Fitzroy Midshelf Zone (grouped annually)]{Observed (logarithmic axis with violin plot overlay) NOx data for the
  Fitzroy Midshelf
  Zone from a) AIMS insitu, b) AIMS FLNTU, c) Satellite, d) eReefs and e)
eReefs926.  Observations are ordered over time and colored conditional on season as Wet (blue
symbols) and Dry (red symbols).  Blue smoother represents Generalized Additive Mixed Model within a
water year and purple line represents average within the water year.  Horizontal red, black and
green dashed lines denote the twice threshold, threshold and half threshold values respectively.
Red and green background shading indicates the range (10\% shade: x4,/4; 30\% shade: x2,/2)
above and below threshold respectively.}\label{fig:violin_NOx_fm}
\end{figure}

\subsubsection{Fitzroy, Offshore}
\paragraph{Chlorophyll}

\begin{figure}[ptbh] 
  a) AIMS insitu\\\includegraphics[align=t,width=0.95\linewidth]{{figures/Exploratory_Data_Analysis/Insitu/eda.year.chl_Fitzroy__Offshore_niskin_log\res}.pdf}\\
  b) AIMS FLNTU\\\includegraphics[align=t, width=0.95\linewidth]{{figures/Exploratory_Data_Analysis/FLNTU/eda.year.chl_Fitzroy__Offshore_flntu_log\res}.pdf}\\
  c) Satellite\\\includegraphics[align=t, width=0.95\linewidth]{{figures/Exploratory_Data_Analysis/Satellite/eda.year.chl_Fitzroy__Offshore__log\res}.png}}\\
  d) eReefs\\\includegraphics[align=t, width=0.95\linewidth]{{figures/Exploratory_Data_Analysis/eReefs/eda.year.chl_Fitzroy__Offshore_eReefs_log\res}.png}}\\
  e) eReefs926\\\includegraphics[align=t, width=0.95\linewidth]{{figures/Exploratory_Data_Analysis/eReefs926/eda.year.chl_Fitzroy__Offshore_eReefs926_log\res}.png}}\\
\caption[Observed Chlorophyll-a data for the Fitzroy Offshore Zone (grouped annually)]{Observed (logarithmic axis with violin plot overlay) Chlorophyll-a data for the
  Fitzroy Offshore
  Zone from a) AIMS insitu, b) AIMS FLNTU, c) Satellite, d) eReefs and e)
eReefs926.  Observations are ordered over time and colored conditional on season as Wet (blue
symbols) and Dry (red symbols).  Blue smoother represents Generalized Additive Mixed Model within a
water year and purple line represents average within the water year.  Horizontal red, black and
green dashed lines denote the twice threshold, threshold and half threshold values respectively.
Red and green background shading indicates the range (10\% shade: x4,/4; 30\% shade: x2,/2)
above and below threshold respectively.}\label{fig:violin_chl_fof}
\end{figure}

\paragraph{Total Suspended Solids}

\begin{figure}[ptbh] 
  a) AIMS insitu\\\includegraphics[align=t,width=0.95\linewidth]{{figures/Exploratory_Data_Analysis/Insitu/eda.year.nap_Fitzroy__Offshore_niskin_log\res}.pdf}\\
  b) AIMS FLNTU\\\includegraphics[align=t, width=0.95\linewidth]{{figures/Exploratory_Data_Analysis/FLNTU/eda.year.nap_Fitzroy__Offshore_flntu_log\res}.pdf}\\
  c) Satellite\\\includegraphics[align=t, width=0.95\linewidth]{{figures/Exploratory_Data_Analysis/Satellite/eda.year.nap_Fitzroy__Offshore__log\res}.png}}\\
  d) eReefs\\\includegraphics[align=t, width=0.95\linewidth]{{figures/Exploratory_Data_Analysis/eReefs/eda.year.nap_Fitzroy__Offshore_eReefs_log\res}.png}}\\
  e) eReefs926\\\includegraphics[align=t, width=0.95\linewidth]{{figures/Exploratory_Data_Analysis/eReefs926/eda.year.nap_Fitzroy__Offshore_eReefs926_log\res}.png}}\\
\caption[Observed TSS data for the Fitzroy Offshore Zone (grouped annually)]{Observed (logarithmic axis with violin plot overlay) Total Suspended Solids data for the
  Fitzroy Offshore
  Zone from a) AIMS insitu, b) AIMS FLNTU, c) Satellite, d) eReefs and e)
eReefs926.  Observations are ordered over time and colored conditional on season as Wet (blue
symbols) and Dry (red symbols).  Blue smoother represents Generalized Additive Mixed Model within a
water year and purple line represents average within the water year.  Horizontal red, black and
green dashed lines denote the twice threshold, threshold and half threshold values respectively.
Red and green background shading indicates the range (10\% shade: x4,/4; 30\% shade: x2,/2)
above and below threshold respectively.}\label{fig:violin_nap_fof}
\end{figure}

\paragraph{Secchi Depth}

\begin{figure}[ptbh] 
  a) AIMS insitu\\\includegraphics[align=t,width=0.95\linewidth]{{figures/Exploratory_mata_Analysis/Insitu/eda.year.sd_Fitzroy__Offshore_niskin_log\res}.pdf}\\
  b) AIMS FLNTU\\\includegraphics[align=t, width=0.95\linewidth]{{figures/Exploratory_Data_Analysis/FLNTU/eda.year.sd_Fitzroy__Offshore_flntu_log\res}.pdf}\\
  c) Satellite\\\includegraphics[align=t, width=0.95\linewidth]{{figures/Exploratory_Data_Analysis/Satellite/eda.year.sd_Fitzroy__Offshore__log\res}.png}}\\
  d) eReefs\\\includegraphics[align=t, width=0.95\linewidth]{{figures/Exploratory_Data_Analysis/eReefs/eda.year.sd_Fitzroy__Offshore_eReefs_log\res}.png}}\\
  e) eReefs926\\\includegraphics[align=t, width=0.95\linewidth]{{figures/Exploratory_Data_Analysis/eReefs926/eda.year.sd_Fitzroy__Offshore_eReefs926_log\res}.png}}\\
\caption[Observed Secchi depth data for the Fitzroy Offshore Zone (grouped annually)]{Observed (logarithmic axis with violin plot overlay) Secchi Depth data for the
  Fitzroy Offshore
  Zone from a) AIMS insitu, b) AIMS FLNTU, c) Satellite, d) eReefs and e)
eReefs926.  Observations are ordered over time and colored conditional on season as Wet (blue
symbols) and Dry (red symbols).  Blue smoother represents Generalized Additive Mixed Model within a
water year and purple line represents average within the water year.  Horizontal red, black and
green dashed lines denote the twice threshold, threshold and half threshold values respectively.
Red and green background shading indicates the range (10\% shade: x4,/4; 30\% shade: x2,/2)
above and below threshold respectively.}\label{fig:violin_sd_fof}
\end{figure}

\paragraph{NOx}

\begin{figure}[ptbh] 
  a) AIMS insitu\\\includegraphics[align=t,width=0.95\linewidth]{{figures/Exploratory_Data_Analysis/Insitu/eda.year.NOx_Fitzroy__Offshore_niskin_log\res}.pdf}\\
  b) AIMS FLNTU\\\includegraphics[align=t, width=0.95\linewidth]{{figures/Exploratory_Data_Analysis/FLNTU/eda.year.NOx_Fitzroy__Offshore_flntu_log\res}.pdf}\\
  c) Satellite\\\includegraphics[align=t, width=0.95\linewidth]{{figures/Exploratory_Data_Analysis/Satellite/eda.year.NOx_Fitzroy__Offshore__log\res}.png}}\\
  d) eReefs\\\includegraphics[align=t, width=0.95\linewidth]{{figures/Exploratory_Data_Analysis/eReefs/eda.year.NOx_Fitzroy__Offshore_eReefs_log\res}.png}}\\
  e) eReefs926\\\includegraphics[align=t, width=0.95\linewidth]{{figures/Exploratory_Data_Analysis/eReefs926/eda.year.NOx_Fitzroy__Offshore_eReefs926_log\res}.png}}\\
\caption[Observed NOx data for the Fitzroy Offshore Zone (grouped annually)]{Observed (logarithmic axis with violin plot overlay) NOx data for the
  Fitzroy Offshore
  Zone from a) AIMS insitu, b) AIMS FLNTU, c) Satellite, d) eReefs and e)
eReefs926.  Observations are ordered over time and colored conditional on season as Wet (blue
symbols) and Dry (red symbols).  Blue smoother represents Generalized Additive Mixed Model within a
water year and purple line represents average within the water year.  Horizontal red, black and
green dashed lines denote the twice threshold, threshold and half threshold values respectively.
Red and green background shading indicates the range (10\% shade: x4,/4; 30\% shade: x2,/2)
above and below threshold respectively.}\label{fig:violin_NOx_fof}
\end{figure}


%% Burnett Mary
\subsubsection{Burnett Mary, Enclosed Coastal}
\paragraph{Chlorophyll}

\begin{figure}[ptbh] 
  a) AIMS insitu\\\includegraphics[align=t,width=0.95\linewidth]{{figures/Exploratory_Data_Analysis/Insitu/eda.year.chl_Burnett Mary__Enclosed Coastal_niskin_log\res}.pdf}\\
  b) AIMS FLNTU\\\includegraphics[align=t, width=0.95\linewidth]{{figures/Exploratory_Data_Analysis/FLNTU/eda.year.chl_Burnett Mary__Enclosed Coastal_flntu_log\res}.pdf}\\
  c) Satellite\\\includegraphics[align=t, width=0.95\linewidth]{{figures/Exploratory_Data_Analysis/Satellite/eda.year.chl_Burnett Mary__Enclosed Coastal__log\res}.png}}\\
  d) eReefs\\\includegraphics[align=t, width=0.95\linewidth]{{figures/Exploratory_Data_Analysis/eReefs/eda.year.chl_Burnett Mary__Enclosed Coastal_eReefs_log\res}.png}}\\
  e) eReefs926\\\includegraphics[align=t, width=0.95\linewidth]{{figures/Exploratory_Data_Analysis/eReefs926/eda.year.chl_Burnett Mary__Enclosed Coastal_eReefs926_log\res}.png}}\\
\caption[Observed Chlorophyll-a data for the Burnett Mary Enclosed Coastal Zone (grouped annually)]{Observed (logarithmic axis with violin plot overlay) Chlorophyll-a data for the
  Burnett Mary Enclosed Coastal
  Zone from a) AIMS insitu, b) AIMS FLNTU, c) Satellite, d) eReefs and e)
eReefs926.  Observations are ordered over time and colored conditional on season as Wet (blue
symbols) and Dry (red symbols).  Blue smoother represents Generalized Additive Mixed Model within a
water year and purple line represents average within the water year.  Horizontal red, black and
green dashed lines denote the twice threshold, threshold and half threshold values respectively.
Red and green background shading indicates the range (10\% shade: x4,/4; 30\% shade: x2,/2)
above and below threshold respectively.}\label{fig:violin_chl_be}
\end{figure}

\paragraph{Total Suspended Solids}

\begin{figure}[ptbh] 
  a) AIMS insitu\\\includegraphics[align=t,width=0.95\linewidth]{{figures/Exploratory_Data_Analysis/Insitu/eda.year.nap_Burnett Mary__Enclosed Coastal_niskin_log\res}.pdf}\\
  b) AIMS FLNTU\\\includegraphics[align=t, width=0.95\linewidth]{{figures/Exploratory_Data_Analysis/FLNTU/eda.year.nap_Burnett Mary__Enclosed Coastal_flntu_log\res}.pdf}\\
  c) Satellite\\\includegraphics[align=t, width=0.95\linewidth]{{figures/Exploratory_Data_Analysis/Satellite/eda.year.nap_Burnett Mary__Enclosed Coastal__log\res}.png}}\\
  d) eReefs\\\includegraphics[align=t, width=0.95\linewidth]{{figures/Exploratory_Data_Analysis/eReefs/eda.year.nap_Burnett Mary__Enclosed Coastal_eReefs_log\res}.png}}\\
  e) eReefs926\\\includegraphics[align=t, width=0.95\linewidth]{{figures/Exploratory_Data_Analysis/eReefs926/eda.year.nap_Burnett Mary__Enclosed Coastal_eReefs926_log\res}.png}}\\
\caption[Observed TSS data for the Burnett Mary Enclosed Coastal Zone (grouped annually)]{Observed (logarithmic axis with violin plot overlay) Total Suspended Solids data for the
  Burnett Mary Enclosed Coastal
  Zone from a) AIMS insitu, b) AIMS FLNTU, c) Satellite, d) eReefs and e)
eReefs926.  Observations are ordered over time and colored conditional on season as Wet (blue
symbols) and Dry (red symbols).  Blue smoother represents Generalized Additive Mixed Model within a
water year and purple line represents average within the water year.  Horizontal red, black and
green dashed lines denote the twice threshold, threshold and half threshold values respectively.
Red and green background shading indicates the range (10\% shade: x4,/4; 30\% shade: x2,/2)
above and below threshold respectively.}\label{fig:violin_nap_be}
\end{figure}

\paragraph{Secchi Depth}

\begin{figure}[ptbh] 
  a) AIMS insitu\\\includegraphics[align=t,width=0.95\linewidth]{{figures/Exploratory_Data_Analysis/Insitu/eda.year.sd_Burnett Mary__Enclosed Coastal_niskin_log\res}.pdf}\\
  b) AIMS FLNTU\\\includegraphics[align=t, width=0.95\linewidth]{{figures/Exploratory_Data_Analysis/FLNTU/eda.year.sd_Burnett Mary__Enclosed Coastal_flntu_log\res}.pdf}\\
  c) Satellite\\\includegraphics[align=t, width=0.95\linewidth]{{figures/Exploratory_Data_Analysis/Satellite/eda.year.sd_Burnett Mary__Enclosed Coastal__log\res}.png}}\\
  d) eReefs\\\includegraphics[align=t, width=0.95\linewidth]{{figures/Exploratory_Data_Analysis/eReefs/eda.year.sd_Burnett Mary__Enclosed Coastal_eReefs_log\res}.png}}\\
  e) eReefs926\\\includegraphics[align=t, width=0.95\linewidth]{{figures/Exploratory_Data_Analysis/eReefs926/eda.year.sd_Burnett Mary__Enclosed Coastal_eReefs926_log\res}.png}}\\
\caption[Observed Secchi depth data for the Burnett Mary Enclosed Coastal Zone (grouped annually)]{Observed (logarithmic axis with violin plot overlay) Secchi Depth data for the
  Burnett Mary Enclosed Coastal
  Zone from a) AIMS insitu, b) AIMS FLNTU, c) Satellite, d) eReefs and e)
eReefs926.  Observations are ordered over time and colored conditional on season as Wet (blue
symbols) and Dry (red symbols).  Blue smoother represents Generalized Additive Mixed Model within a
water year and purple line represents average within the water year.  Horizontal red, black and
green dashed lines denote the twice threshold, threshold and half threshold values respectively.
Red and green background shading indicates the range (10\% shade: x4,/4; 30\% shade: x2,/2)
above and below threshold respectively.}\label{fig:violin_sd_be}
\end{figure}

\paragraph{NOx}

\begin{figure}[ptbh] 
  a) AIMS insitu\\\includegraphics[align=t,width=0.95\linewidth]{{figures/Exploratory_Data_Analysis/Insitu/eda.year.NOx_Burnett Mary__Open Coastal_niskin_log\res}.pdf}\\
  b) AIMS FLNTU\\\includegraphics[align=t, width=0.95\linewidth]{{figures/Exploratory_Data_Analysis/FLNTU/eda.year.NOx_Burnett Mary__Open Coastal_flntu_log\res}.pdf}\\
  c) Satellite\\\includegraphics[align=t, width=0.95\linewidth]{{figures/Exploratory_Data_Analysis/Satellite/eda.year.NOx_Burnett Mary__Enclosed Coastal__log\res}.png}}\\
  d) eReefs\\\includegraphics[align=t, width=0.95\linewidth]{{figures/Exploratory_Data_Analysis/eReefs/eda.year.NOx_Burnett Mary__Enclosed Coastal_eReefs_log\res}.png}}\\
  e) eReefs926\\\includegraphics[align=t, width=0.95\linewidth]{{figures/Exploratory_Data_Analysis/eReefs926/eda.year.NOx_Burnett Mary__Enclosed Coastal_eReefs926_log\res}.png}}\\
\caption[Observed NOx data for the Burnett Mary Enclosed Coastal Zone (grouped annually)]{Observed (logarithmic axis with violin plot overlay) NOx data for the
  Burnett Mary Enclosed Coastal
  Zone from a) AIMS insitu, b) AIMS FLNTU, c) Satellite, d) eReefs and e)
eReefs926.  Observations are ordered over time and colored conditional on season as Wet (blue
symbols) and Dry (red symbols).  Blue smoother represents Generalized Additive Mixed Model within a
water year and purple line represents average within the water year.  Horizontal red, black and
green dashed lines denote the twice threshold, threshold and half threshold values respectively.
Red and green background shading indicates the range (10\% shade: x4,/4; 30\% shade: x2,/2)
above and below threshold respectively.}\label{fig:violin_NOx_be}
\end{figure}


\subsubsection{Burnett Mary, Open Coastal}
\paragraph{Chlorophyll}

\begin{figure}[ptbh] 
  a) AIMS insitu\\\includegraphics[align=t,width=0.95\linewidth]{{figures/Exploratory_Data_Analysis/Insitu/eda.year.chl_Burnett Mary__Open Coastal_niskin_log\res}.pdf}\\
  b) AIMS FLNTU\\\includegraphics[align=t, width=0.95\linewidth]{{figures/Exploratory_Data_Analysis/FLNTU/eda.year.chl_Burnett Mary__Open Coastal_flntu_log\res}.pdf}\\
  c) Satellite\\\includegraphics[align=t, width=0.95\linewidth]{{figures/Exploratory_Data_Analysis/Satellite/eda.year.chl_Burnett Mary__Open Coastal__log\res}.png}}\\
  d) eReefs\\\includegraphics[align=t, width=0.95\linewidth]{{figures/Exploratory_Data_Analysis/eReefs/eda.year.chl_Burnett Mary__Open Coastal_eReefs_log\res}.png}}\\
  e) eReefs926\\\includegraphics[align=t, width=0.95\linewidth]{{figures/Exploratory_Data_Analysis/eReefs926/eda.year.chl_Burnett Mary__Open Coastal_eReefs926_log\res}.png}}\\
\caption[Observed Chlorophyll-a data for the Burnett Mary Open Coastal Zone (grouped annually)]{Observed (logarithmic axis with violin plot overlay) Chlorophyll-a data for the
  Burnett Mary Open Coastal
  Zone from a) AIMS insitu, b) AIMS FLNTU, c) Satellite, d) eReefs and e)
eReefs926.  Observations are ordered over time and colored conditional on season as Wet (blue
symbols) and Dry (red symbols).  Blue smoother represents Generalized Additive Mixed Model within a
water year and purple line represents average within the water year.  Horizontal red, black and
green dashed lines denote the twice threshold, threshold and half threshold values respectively.
Red and green background shading indicates the range (10\% shade: x4,/4; 30\% shade: x2,/2)
above and below threshold respectively.}\label{fig:violin_chl_bo}
\end{figure}

\paragraph{Total Suspended Solids}

\begin{figure}[ptbh] 
  a) AIMS insitu\\\includegraphics[align=t,width=0.95\linewidth]{{figures/Exploratory_Data_Analysis/Insitu/eda.year.nap_Burnett Mary__Open Coastal_niskin_log\res}.pdf}\\
  b) AIMS FLNTU\\\includegraphics[align=t, width=0.95\linewidth]{{figures/Exploratory_Data_Analysis/FLNTU/eda.year.nap_Burnett Mary__Open Coastal_flntu_log\res}.pdf}\\
  c) Satellite\\\includegraphics[align=t, width=0.95\linewidth]{{figures/Exploratory_Data_Analysis/Satellite/eda.year.nap_Burnett Mary__Open Coastal__log\res}.png}}\\
  d) eReefs\\\includegraphics[align=t, width=0.95\linewidth]{{figures/Exploratory_Data_Analysis/eReefs/eda.year.nap_Burnett Mary__Open Coastal_eReefs_log\res}.png}}\\
  e) eReefs926\\\includegraphics[align=t, width=0.95\linewidth]{{figures/Exploratory_Data_Analysis/eReefs926/eda.year.nap_Burnett Mary__Open Coastal_eReefs926_log\res}.png}}\\
\caption[Observed TSS data for the Burnett Mary Open Coastal Zone (grouped annually)]{Observed (logarithmic axis with violin plot overlay) Total Suspended Solids data for the
  Burnett Mary Open Coastal
  Zone from a) AIMS insitu, b) AIMS FLNTU, c) Satellite, d) eReefs and e)
eReefs926.  Observations are ordered over time and colored conditional on season as Wet (blue
symbols) and Dry (red symbols).  Blue smoother represents Generalized Additive Mixed Model within a
water year and purple line represents average within the water year.  Horizontal red, black and
green dashed lines denote the twice threshold, threshold and half threshold values respectively.
Red and green background shading indicates the range (10\% shade: x4,/4; 30\% shade: x2,/2)
above and below threshold respectively.}\label{fig:violin_nap_bo}
\end{figure}

\paragraph{Secchi Depth}

\begin{figure}[ptbh] 
  a) AIMS insitu\\\includegraphics[align=t,width=0.95\linewidth]{{figures/Exploratory_Data_Analysis/Insitu/eda.year.sd_Burnett Mary__Open Coastal_niskin_log\res}.pdf}\\
  b) AIMS FLNTU\\\includegraphics[align=t, width=0.95\linewidth]{{figures/Exploratory_Data_Analysis/FLNTU/eda.year.sd_Burnett Mary__Open Coastal_flntu_log\res}.pdf}\\
  c) Satellite\\\includegraphics[align=t, width=0.95\linewidth]{{figures/Exploratory_Data_Analysis/Satellite/eda.year.sd_Burnett Mary__Open Coastal__log\res}.png}}\\
  d) eReefs\\\includegraphics[align=t, width=0.95\linewidth]{{figures/Exploratory_Data_Analysis/eReefs/eda.year.sd_Burnett Mary__Open Coastal_eReefs_log\res}.png}}\\
  e) eReefs926\\\includegraphics[align=t, width=0.95\linewidth]{{figures/Exploratory_Data_Analysis/eReefs926/eda.year.sd_Burnett Mary__Open Coastal_eReefs926_log\res}.png}}\\
\caption[Observed Secchi depth data for the Burnett Mary Open Coastal Zone (grouped annually)]{Observed (logarithmic axis with violin plot overlay) Secchi Depth data for the
  Burnett Mary Open Coastal
  Zone from a) AIMS insitu, b) AIMS FLNTU, c) Satellite, d) eReefs and e)
eReefs926.  Observations are ordered over time and colored conditional on season as Wet (blue
symbols) and Dry (red symbols).  Blue smoother represents Generalized Additive Mixed Model within a
water year and purple line represents average within the water year.  Horizontal red, black and
green dashed lines denote the twice threshold, threshold and half threshold values respectively.
Red and green background shading indicates the range (10\% shade: x4,/4; 30\% shade: x2,/2)
above and below threshold respectively.}\label{fig:violin_sd_bo}
\end{figure}

\paragraph{NOx}

\begin{figure}[ptbh] 
  a) AIMS insitu\\\includegraphics[align=t,width=0.95\linewidth]{{figures/Exploratory_Data_Analysis/Insitu/eda.year.NOx_Burnett Mary__Open Coastal_niskin_log\res}.pdf}\\
  b) AIMS FLNTU\\\includegraphics[align=t, width=0.95\linewidth]{{figures/Exploratory_Data_Analysis/FLNTU/eda.year.NOx_Burnett Mary__Open Coastal_flntu_log\res}.pdf}\\
  c) Satellite\\\includegraphics[align=t, width=0.95\linewidth]{{figures/Exploratory_Data_Analysis/Satellite/eda.year.NOx_Burnett Mary__Open Coastal__log\res}.png}}\\
  d) eReefs\\\includegraphics[align=t, width=0.95\linewidth]{{figures/Exploratory_Data_Analysis/eReefs/eda.year.NOx_Burnett Mary__Open Coastal_eReefs_log\res}.png}}\\
  e) eReefs926\\\includegraphics[align=t, width=0.95\linewidth]{{figures/Exploratory_Data_Analysis/eReefs926/eda.year.NOx_Burnett Mary__Open Coastal_eReefs926_log\res}.png}}\\
\caption[Observed NOx data for the Burnett Mary Open Coastal Zone (grouped annually)]{Observed (logarithmic axis with violin plot overlay) NOx data for the
  Burnett Mary Open Coastal
  Zone from a) AIMS insitu, b) AIMS FLNTU, c) Satellite, d) eReefs and e)
eReefs926.  Observations are ordered over time and colored conditional on season as Wet (blue
symbols) and Dry (red symbols).  Blue smoother represents Generalized Additive Mixed Model within a
water year and purple line represents average within the water year.  Horizontal red, black and
green dashed lines denote the twice threshold, threshold and half threshold values respectively.
Red and green background shading indicates the range (10\% shade: x4,/4; 30\% shade: x2,/2)
above and below threshold respectively.}\label{fig:violin_NOx_bo}
\end{figure}

\subsubsection{Burnett Mary, Midshelf}
\paragraph{Chlorophyll}

\begin{figure}[ptbh] 
  a) AIMS insitu\\\includegraphics[align=t,width=0.95\linewidth]{{figures/Exploratory_Data_Analysis/Insitu/eda.year.chl_Burnett Mary__Midshelf_niskin_log\res}.pdf}\\
  b) AIMS FLNTU\\\includegraphics[align=t, width=0.95\linewidth]{{figures/Exploratory_Data_Analysis/FLNTU/eda.year.chl_Burnett Mary__Midshelf_flntu_log\res}.pdf}\\
  c) Satellite\\\includegraphics[align=t, width=0.95\linewidth]{{figures/Exploratory_Data_Analysis/Satellite/eda.year.chl_Burnett Mary__Midshelf__log\res}.png}}\\
  d) eReefs\\\includegraphics[align=t, width=0.95\linewidth]{{figures/Exploratory_Data_Analysis/eReefs/eda.year.chl_Burnett Mary__Midshelf_eReefs_log\res}.png}}\\
  e) eReefs926\\\includegraphics[align=t, width=0.95\linewidth]{{figures/Exploratory_Data_Analysis/eReefs926/eda.year.chl_Burnett Mary__Midshelf_eReefs926_log\res}.png}}\\
\caption[Observed Chlorophyll-a data for the Burnett Mary Midshelf Zone (grouped annually)]{Observed (logarithmic axis with violin plot overlay) Chlorophyll-a data for the
  Burnett Mary Midshelf
  Zone from a) AIMS insitu, b) AIMS FLNTU, c) Satellite, d) eReefs and e)
eReefs926.  Observations are ordered over time and colored conditional on season as Wet (blue
symbols) and Dry (red symbols).  Blue smoother represents Generalized Additive Mixed Model within a
water year and purple line represents average within the water year.  Horizontal red, black and
green dashed lines denote the twice threshold, threshold and half threshold values respectively.
Red and green background shading indicates the range (10\% shade: x4,/4; 30\% shade: x2,/2)
above and below threshold respectively.}\label{fig:violin_chl_bm}
\end{figure}

\paragraph{Total Suspended Solids}

\begin{figure}[ptbh] 
  a) AIMS insitu\\\includegraphics[align=t,width=0.95\linewidth]{{figures/Exploratory_Data_Analysis/Insitu/eda.year.nap_Burnett Mary__Midshelf_niskin_log\res}.pdf}\\
  b) AIMS FLNTU\\\includegraphics[align=t, width=0.95\linewidth]{{figures/Exploratory_Data_Analysis/FLNTU/eda.year.nap_Burnett Mary__Midshelf_flntu_log\res}.pdf}\\
  c) Satellite\\\includegraphics[align=t, width=0.95\linewidth]{{figures/Exploratory_Data_Analysis/Satellite/eda.year.nap_Burnett Mary__Midshelf__log\res}.png}}\\
  d) eReefs\\\includegraphics[align=t, width=0.95\linewidth]{{figures/Exploratory_Data_Analysis/eReefs/eda.year.nap_Burnett Mary__Midshelf_eReefs_log\res}.png}}\\
  e) eReefs926\\\includegraphics[align=t, width=0.95\linewidth]{{figures/Exploratory_Data_Analysis/eReefs926/eda.year.nap_Burnett Mary__Midshelf_eReefs926_log\res}.png}}\\
\caption[Observed TSS data for the Burnett Mary Midshelf Zone (grouped annually)]{Observed (logarithmic axis with violin plot overlay) Total Suspended Solids data for the
  Burnett Mary Midshelf
  Zone from a) AIMS insitu, b) AIMS FLNTU, c) Satellite, d) eReefs and e)
eReefs926.  Observations are ordered over time and colored conditional on season as Wet (blue
symbols) and Dry (red symbols).  Blue smoother represents Generalized Additive Mixed Model within a
water year and purple line represents average within the water year.  Horizontal red, black and
green dashed lines denote the twice threshold, threshold and half threshold values respectively.
Red and green background shading indicates the range (10\% shade: x4,/4; 30\% shade: x2,/2)
above and below threshold respectively.}\label{fig:violin_nap_bm}
\end{figure}

\paragraph{Secchi Depth}

\begin{figure}[ptbh] 
  a) AIMS insitu\\\includegraphics[align=t,width=0.95\linewidth]{{figures/Exploratory_Data_Analysis/Insitu/eda.year.sd_Burnett Mary__Midshelf_niskin_log\res}.pdf}\\
  b) AIMS FLNTU\\\includegraphics[align=t, width=0.95\linewidth]{{figures/Exploratory_Data_Analysis/FLNTU/eda.year.sd_Burnett Mary__Midshelf_flntu_log\res}.pdf}\\
  c) Satellite\\\includegraphics[align=t, width=0.95\linewidth]{{figures/Exploratory_Data_Analysis/Satellite/eda.year.sd_Burnett Mary__Midshelf__log\res}.png}}\\
  d) eReefs\\\includegraphics[align=t, width=0.95\linewidth]{{figures/Exploratory_Data_Analysis/eReefs/eda.year.sd_Burnett Mary__Midshelf_eReefs_log\res}.png}}\\
  e) eReefs926\\\includegraphics[align=t, width=0.95\linewidth]{{figures/Exploratory_Data_Analysis/eReefs926/eda.year.sd_Burnett Mary__Midshelf_eReefs926_log\res}.png}}\\
\caption[Observed Secchi depth data for the Burnett Mary Midshelf Zone (grouped annually)]{Observed (logarithmic axis with violin plot overlay) Secchi Depth data for the
  Burnett Mary Midshelf
  Zone from a) AIMS insitu, b) AIMS FLNTU, c) Satellite, d) eReefs and e)
eReefs926.  Observations are ordered over time and colored conditional on season as Wet (blue
symbols) and Dry (red symbols).  Blue smoother represents Generalized Additive Mixed Model within a
water year and purple line represents average within the water year.  Horizontal red, black and
green dashed lines denote the twice threshold, threshold and half threshold values respectively.
Red and green background shading indicates the range (10\% shade: x4,/4; 30\% shade: x2,/2)
above and below threshold respectively.}\label{fig:violin_sd_bm}
\end{figure}

\paragraph{NOx}

\begin{figure}[ptbh] 
  a) AIMS insitu\\\includegraphics[align=t,width=0.95\linewidth]{{figures/Exploratory_Data_Analysis/Insitu/eda.year.NOx_Burnett Mary__Midshelf_niskin_log\res}.pdf}\\
  b) AIMS FLNTU\\\includegraphics[align=t, width=0.95\linewidth]{{figures/Exploratory_Data_Analysis/FLNTU/eda.year.NOx_Burnett Mary__Midshelf_flntu_log\res}.pdf}\\
  c) Satellite\\\includegraphics[align=t, width=0.95\linewidth]{{figures/Exploratory_Data_Analysis/Satellite/eda.year.NOx_Burnett Mary__Midshelf__log\res}.png}}\\
  d) eReefs\\\includegraphics[align=t, width=0.95\linewidth]{{figures/Exploratory_Data_Analysis/eReefs/eda.year.NOx_Burnett Mary__Midshelf_eReefs_log\res}.png}}\\
  e) eReefs926\\\includegraphics[align=t, width=0.95\linewidth]{{figures/Exploratory_Data_Analysis/eReefs926/eda.year.NOx_Burnett Mary__Midshelf_eReefs926_log\res}.png}}\\
\caption[Observed NOx data for the Burnett Mary Midshelf Zone (grouped annually)]{Observed (logarithmic axis with violin plot overlay) NOx data for the
  Burnett Mary Midshelf
  Zone from a) AIMS insitu, b) AIMS FLNTU, c) Satellite, d) eReefs and e)
eReefs926.  Observations are ordered over time and colored conditional on season as Wet (blue
symbols) and Dry (red symbols).  Blue smoother represents Generalized Additive Mixed Model within a
water year and purple line represents average within the water year.  Horizontal red, black and
green dashed lines denote the twice threshold, threshold and half threshold values respectively.
Red and green background shading indicates the range (10\% shade: x4,/4; 30\% shade: x2,/2)
above and below threshold respectively.}\label{fig:violin_NOx_bm}
\end{figure}

\subsubsection{Burnett Mary, Offshore}
\paragraph{Chlorophyll}

\begin{figure}[ptbh] 
  a) AIMS insitu\\\includegraphics[align=t,width=0.95\linewidth]{{figures/Exploratory_Data_Analysis/Insitu/eda.year.chl_Burnett Mary__Offshore_niskin_log\res}.pdf}\\
  b) AIMS FLNTU\\\includegraphics[align=t, width=0.95\linewidth]{{figures/Exploratory_Data_Analysis/FLNTU/eda.year.chl_Burnett Mary__Offshore_flntu_log\res}.pdf}\\
  c) Satellite\\\includegraphics[align=t, width=0.95\linewidth]{{figures/Exploratory_Data_Analysis/Satellite/eda.year.chl_Burnett Mary__Offshore__log\res}.png}}\\
  d) eReefs\\\includegraphics[align=t, width=0.95\linewidth]{{figures/Exploratory_Data_Analysis/eReefs/eda.year.chl_Burnett Mary__Offshore_eReefs_log\res}.png}}\\
  e) eReefs926\\\includegraphics[align=t, width=0.95\linewidth]{{figures/Exploratory_Data_Analysis/eReefs926/eda.year.chl_Burnett Mary__Offshore_eReefs926_log\res}.png}}\\
\caption[Observed Chlorophyll-a data for the Burnett Mary Offshore Zone (grouped annually)]{Observed (logarithmic axis with violin plot overlay) Chlorophyll-a data for the
  Burnett Mary Offshore
  Zone from a) AIMS insitu, b) AIMS FLNTU, c) Satellite, d) eReefs and e)
eReefs926.  Observations are ordered over time and colored conditional on season as Wet (blue
symbols) and Dry (red symbols).  Blue smoother represents Generalized Additive Mixed Model within a
water year and purple line represents average within the water year.  Horizontal red, black and
green dashed lines denote the twice threshold, threshold and half threshold values respectively.
Red and green background shading indicates the range (10\% shade: x4,/4; 30\% shade: x2,/2)
above and below threshold respectively.}\label{fig:violin_chl_bof}
\end{figure}

\paragraph{Total Suspended Solids}

\begin{figure}[ptbh] 
  a) AIMS insitu\\\includegraphics[align=t,width=0.95\linewidth]{{figures/Exploratory_Data_Analysis/Insitu/eda.year.nap_Burnett Mary__Offshore_niskin_log\res}.pdf}\\
  b) AIMS FLNTU\\\includegraphics[align=t, width=0.95\linewidth]{{figures/Exploratory_Data_Analysis/FLNTU/eda.year.nap_Burnett Mary__Offshore_flntu_log\res}.pdf}\\
  c) Satellite\\\includegraphics[align=t, width=0.95\linewidth]{{figures/Exploratory_Data_Analysis/Satellite/eda.year.nap_Burnett Mary__Offshore__log\res}.png}}\\
  d) eReefs\\\includegraphics[align=t, width=0.95\linewidth]{{figures/Exploratory_Data_Analysis/eReefs/eda.year.nap_Burnett Mary__Offshore_eReefs_log\res}.png}}\\
  e) eReefs926\\\includegraphics[align=t, width=0.95\linewidth]{{figures/Exploratory_Data_Analysis/eReefs926/eda.year.nap_Burnett Mary__Offshore_eReefs926_log\res}.png}}\\
\caption[Observed TSS data for the Burnett Mary Offshore Zone (grouped annually)]{Observed (logarithmic axis with violin plot overlay) Total Suspended Solids data for the
  Burnett Mary Offshore
  Zone from a) AIMS insitu, b) AIMS FLNTU, c) Satellite, d) eReefs and e)
eReefs926.  Observations are ordered over time and colored conditional on season as Wet (blue
symbols) and Dry (red symbols).  Blue smoother represents Generalized Additive Mixed Model within a
water year and purple line represents average within the water year.  Horizontal red, black and
green dashed lines denote the twice threshold, threshold and half threshold values respectively.
Red and green background shading indicates the range (10\% shade: x4,/4; 30\% shade: x2,/2)
above and below threshold respectively.}\label{fig:violin_nap_bof}
\end{figure}

\paragraph{Secchi Depth}

\begin{figure}[ptbh] 
  a) AIMS insitu\\\includegraphics[align=t,width=0.95\linewidth]{{figures/Exploratory_mata_Analysis/Insitu/eda.year.sd_Burnett Mary__Offshore_niskin_log\res}.pdf}\\
  b) AIMS FLNTU\\\includegraphics[align=t, width=0.95\linewidth]{{figures/Exploratory_Data_Analysis/FLNTU/eda.year.sd_Burnett Mary__Offshore_flntu_log\res}.pdf}\\
  c) Satellite\\\includegraphics[align=t, width=0.95\linewidth]{{figures/Exploratory_Data_Analysis/Satellite/eda.year.sd_Burnett Mary__Offshore__log\res}.png}}\\
  d) eReefs\\\includegraphics[align=t, width=0.95\linewidth]{{figures/Exploratory_Data_Analysis/eReefs/eda.year.sd_Burnett Mary__Offshore_eReefs_log\res}.png}}\\
  e) eReefs926\\\includegraphics[align=t, width=0.95\linewidth]{{figures/Exploratory_Data_Analysis/eReefs926/eda.year.sd_Burnett Mary__Offshore_eReefs926_log\res}.png}}\\
\caption[Observed Secchi depth data for the Burnett Mary Offshore Zone (grouped annually)]{Observed (logarithmic axis with violin plot overlay) Secchi Depth data for the
  Burnett Mary Offshore
  Zone from a) AIMS insitu, b) AIMS FLNTU, c) Satellite, d) eReefs and e)
eReefs926.  Observations are ordered over time and colored conditional on season as Wet (blue
symbols) and Dry (red symbols).  Blue smoother represents Generalized Additive Mixed Model within a
water year and purple line represents average within the water year.  Horizontal red, black and
green dashed lines denote the twice threshold, threshold and half threshold values respectively.
Red and green background shading indicates the range (10\% shade: x4,/4; 30\% shade: x2,/2)
above and below threshold respectively.}\label{fig:violin_sd_bof}
\end{figure}

\paragraph{NOx}

\begin{figure}[ptbh] 
  a) AIMS insitu\\\includegraphics[align=t,width=0.95\linewidth]{{figures/Exploratory_Data_Analysis/Insitu/eda.year.NOx_Burnett Mary__Offshore_niskin_log\res}.pdf}\\
  b) AIMS FLNTU\\\includegraphics[align=t, width=0.95\linewidth]{{figures/Exploratory_Data_Analysis/FLNTU/eda.year.NOx_Burnett Mary__Offshore_flntu_log\res}.pdf}\\
  c) Satellite\\\includegraphics[align=t, width=0.95\linewidth]{{figures/Exploratory_Data_Analysis/Satellite/eda.year.NOx_Burnett Mary__Offshore__log\res}.png}}\\
  d) eReefs\\\includegraphics[align=t, width=0.95\linewidth]{{figures/Exploratory_Data_Analysis/eReefs/eda.year.NOx_Burnett Mary__Offshore_eReefs_log\res}.png}}\\
  e) eReefs926\\\includegraphics[align=t, width=0.95\linewidth]{{figures/Exploratory_Data_Analysis/eReefs926/eda.year.NOx_Burnett Mary__Offshore_eReefs926_log\res}.png}}\\
\caption[Observed NOx data for the Burnett Mary Offshore Zone (grouped annually)]{Observed (logarithmic axis with violin plot overlay) NOx data for the
  Burnett Mary Offshore
  Zone from a) AIMS insitu, b) AIMS FLNTU, c) Satellite, d) eReefs and e)
eReefs926.  Observations are ordered over time and colored conditional on season as Wet (blue
symbols) and Dry (red symbols).  Blue smoother represents Generalized Additive Mixed Model within a
water year and purple line represents average within the water year.  Horizontal red, black and
green dashed lines denote the twice threshold, threshold and half threshold values respectively.
Red and green background shading indicates the range (10\% shade: x4,/4; 30\% shade: x2,/2)
above and below threshold respectively.}\label{fig:violin_NOx_bof}
\end{figure}
