%%%%%%%%%%%%%%%%%%%%%%%%%%%%%%%%%%%%%%%%%%%%%%%%%%%
%% Compile with                                  %%
%% xelatex -interaction=nonstopmode WEreport.tex %%
%% or via the Makefile with                      %%
%% make pdf -i                                   %%
%%%%%%%%%%%%%%%%%%%%%%%%%%%%%%%%%%%%%%%%%%%%%%%%%%%

%%%%%%%%%%%%%%%%%%%%%%%%%%%%%%%%%%%%%%%%%%%%%%%%%%%%%%%%%%%%%%%%%%%%%%%
%% Prior to running this, make sure that WQreport_aux.R has been run %%
%% this will generate all the dynamic tables                         %%
%%%%%%%%%%%%%%%%%%%%%%%%%%%%%%%%%%%%%%%%%%%%%%%%%%%%%%%%%%%%%%%%%%%%%%%

\documentclass[a4paper,8pt]{AIMSreport}
{\let\cleardoublepage\clearpage\begin{titlepage}

\thispagestyle{firststyle}
\newcommand{\HRule}{\rule{\linewidth}{0.5mm}} % Defines a new command for the horizontal lines, change thickness here


\graphicspath{{\string~/Work/Resources/Images/}}
\begin{picture}(100,60)(2.54cm,0cm)
    \parbox[b]{\paperwidth}{%
     \centering\includegraphics[width=186.2mm, height=36.1mm]{AIMSbanner.png}%
    }
\end{picture}


~\\[4em]


\begin{raggedleft}{\fontsize{24}{24}\titlefont \color{AIMSblue}NESP.3.2.5\par}\\[0.4cm] % Title of your document
\end{raggedleft}

{ \hfill\fontsize{12}{12}\color{AIMSblue}\uppercase{NESP.3.2.5}}\\[3em]

\begin{picture}(0,0)(2.54cm,0cm)
    \parbox[b]{\paperwidth}{%
     \centering\includegraphics[width=186.2mm, height=1.3mm]{AIMSline.png}%
    }
\end{picture} \\[1em]

\hfill{\fontsize{14}{14}\titlefont{Author: }\textsc{Murray Logan}} % Your name
~\\[25em]

{\hfill\fontsize{14}{14}\color{AIMSblue}\titlefont{AIMS: Australia's tropical marine research agency}}\\ % Your name

{\hfill\large \today}\\ % Date, change the \today to a set date if you want to be precise

\vfill % Fill the rest of the page with whitespace
\end{titlepage}}





%% Disclaimer page------------------------------------------------------------
\thispagestyle{firststyle}
Australian Institute of Marine Science\\
\begin{tabularx}{\linewidth}{llX}
PMB No 3                & PO Box 41775      & The UWA Oceans Institute (M096)\\
Townsville MC QLD 4810  & Casuarina NT 0811 & Crawley WA 6009\\
\end{tabularx}
\\[3em]

This report should be cited as:

%Logan, M (2016). {Murray Logan}. Jiaozhou Bay Report Card prepared for the Institute of Oceanology, Chinese Academy of Sciences. \today, (\pageref{LastPage} pp).
%~\\[1em]
Enquires should be directed to:\\[1em]
Murray Logan\\
m.logan@aims.gov.au  \\[2em]

\textcopyright~Copyright: Australian Institute of Marine Science (AIMS) \the\year

All rights are reserved and no part of this document may be reproduced, stored or copied in any form or by any means whatsoever except with the prior written permission of AIMS


\part*{DISCLAIMER}
While reasonable efforts have been made to ensure that the contents of this document are factually correct, AIMS does not make any representation or give any warranty regarding the accuracy, completeness, currency or suitability for any particular purpose of the information or statements contained in this document. To the extent permitted by law AIMS shall not be liable for any loss, damage, cost or expense that may be occasioned directly or indirectly through the use of or reliance on the contents of this document.


\part*{Revision History}

\begin{tabular}{|l|l|l|p{4cm}|p{4cm}|p{3cm}|}
\hline
\multicolumn{2}{|l|}{Version} & Title & Name & Date & Comments\\
\hline
\multirow{2}{*}{1} & Author &Dr&Murray Logan&\today&\\
\cline{2-6}
 & Approved by & &&&\\
\hline
\multirow{2}{*}{2} & Author &&&&\\
\cline{2-6}
 & Approved by & &&&\\
\hline
\multirow{2}{*}{3} & Author &&&&\\
\cline{2-6}
 & Approved by & &&&\\
\hline
\multirow{2}{*}{4} & Author &&&&\\
\cline{2-6}
 & Approved by & &&&\\
\hline

\end{tabular}

\newpage

\tableofcontents
\newpage
\listoffigures
\newpage
\listoftables
\newpage


\begin{document}

\section{Executive Summary}

\section{Introduction}

\section{Data sources}

Report cards are typically compiled and communicated annually.
However, the time window that constitutes a year differs from report card to report card.
Many environmental report cards communicate on data collected within a financial year.
This schedule provides a reporting window that is consistent with other management and governmental
considerations.  Others use a time window that naturally aligns with the cycle of some major underlying environmental
gradient - such as wet/dry season. For this project, we are adopting using the same water year (1st Oct -- 31 Sept) definition as the AIMS inshore Water
Quality Marine Monitoring Program \citep{Lonborg-MMP-2015}.

\subsection{Indicators}
\subsection{AIMS insitu samples}
\subsection{AIMS FLNTU samples}
\subsection{Remote sensing (BOM satellite)}
\subsection{eReefs}
\subsection{eReefs926}
\subsection{Thresholds}
\section{Exploratory data analysis}
\subsection{All data}
\subsection{Annual data}
\subsection{Monthly data}
\subsection{Spatial data}
\subsection{Comparison of data sources}

\section{Index metrics}

\subsection{Theoretical framework}
\subsection{Multivariate health indicators}
\subsection{Thresholds}
\subsection{Unifying indices}
\subsection{Hierarchical indices}
\subsection{Summary of adopted methodologies}
\subsection{Index sensitivity}
\subsection{Index explorations}
\subsection{Indices}

\subsection{Sources}
\subsection{Exploration of Measures}
\subsection{Measure/Site}
\subsection{Summary of recommendations}

\section{Hierarchical aggregations}
\subsection{Theoretical framework}
\subsection{Bootstrap aggregation}                   
\subsection{Beta approximation}
\subsection{Weights}           
\subsection{Expert interventions}
\subsection{Scores and Grades}
\subsection{Certainty rating}
\subsection{Confidence intervals}

\subsection{Summary of adopted methodologies}
\subsection{Aggregation summaries}
\subsubsection{Measure/Zone}
\subsubsection{Indicator/Site}
\subsubsection{Indicator/Zone}

\subsection{Summary of recommendations}
 
~\\[2em]\addcontentsline{toc}{section}{References}
\bibliographystyle{ecology}
\bibliography{References}

\clearpage

\end{document}  